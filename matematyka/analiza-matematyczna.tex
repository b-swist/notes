\chapter{Analiza matematyczna}

\section{Granica funkcji w punkcie}

\subsection{Działania na granicach}
\begin{gather*}
  \lim_{x \to x_0} \left[c \cdot f(x)\right] = c \cdot \lim_{x \to x_0} f(x)\\
  \lim_{x \to x_0} \left[f(x) \pm g(x)\right] = \lim_{x \to x_0} f(x) \pm \lim_{x \to x_0} g(x)\\
  \lim_{x \to x_0} \left[f(x) \cdot g(x)\right] = \lim_{x \to x_0} f(x) \cdot \lim_{x \to x_0} g(x)\\
  \lim_{x \to x_0} \left[\frac{f(x)}{g(x)}\right] =
  \frac{\displaystyle\lim_{x \to x_0} f(x)}{\displaystyle\lim_{x \to x_0} g(x)}
\end{gather*}

\subsection{Granica funkcji wielomianowej}
\begin{equation*}
  \lim_{x \to a} W(x) = W(a)
\end{equation*}

\subsection{Granice jednostronne}
\begin{equation*}
  \lim_{x \to x_0} f(x) = g \Leftrightarrow \lim_{x \to x_0^+} f(x) = \lim_{x \to x_0^-} f(x) = g
\end{equation*}

\subsection{Granice niewłaściwe}
\begin{equation*}
  \lim_{x \to x_0} f(x) = \pm\infty
\end{equation*}

\section{Ciągłość funkcji}
\begin{theorem}
  Funkcje wielomianowe, wymierne, potęgowe, wykładnicze, logarytmiczne i~trygonometryczne oraz ich
  sumy, różnice, iloczyny i ilorazy są ciągłe w każdym punkcie, w~którym są określone.
\end{theorem}

\begin{theorem}[Twierdzenie Darboux]
  Jeśli funkcja $f$ jest ciągła w przedziale domkniętym $\langle a, b\rangle$ oraz $f(a) \cdot f(b)
  < 0$, to istnieje taka liczba $c$, $c \in \langle a, b\rangle$, dla której $f(c) = 0$.
\end{theorem}

\section{Pochodna funkcji}
\subsection{Pochodna funkcji w punkcie}
\begin{equation*}
  f'(x_0) = \lim_{h \to 0} \frac{f(x_0 + h) - f(x)}{h}
\end{equation*}

\subsection{Wybrane wzory pochodnych}
\begin{gather*}
  (c)' = 0\\
  \left(\frac 1 x\right)' = -\frac{1}{x^2}\\
  \left(\sqrt x\right)' = \frac{1}{2 \sqrt x}\\
  \left(x^n\right)' = n \cdot x^{n-1}
\end{gather*}

\subsection{Działania na pochodnych}
\begin{gather*}
  \left[c \cdot f(x)\right]' = c \cdot f'(x)\\
  \left[f(x) \pm g(x)\right]' = f'(x) \pm g'(x)\\
  \left[f(x) \cdot g(x)\right]' = f'(x) \cdot g(x) + f(x) \cdot g'(x)\\
  \left[\frac{f(x)}{g(x)}\right]' = \frac{f'(x) \cdot g(x) + f(x) \cdot  g'(x)}{\left[g(x)\right]^2}
\end{gather*}

\subsection{Pochodne funkcji złożonych}
\begin{gather*}
  (g \circ f)(x) = g(f(x))\\
  (g \circ f)'(x) = g'(f(x)) \cdot f'(x)
\end{gather*}

\section{Styczna do wykresu funkcji}
\begin{gather*}
  y = f'(x_0)(x - x_0) + f(x_0)\\
  \tg \alpha = f'(x_0)
\end{gather*}

% vim:spell:spl=pl
