\documentclass{article}
\title{Matematyka}
\date{2025-02-25}
\author{Bartosz Świst}

\usepackage[utf8]{inputenc}
\usepackage[T1]{fontenc}
\usepackage{amsmath, amssymb, amsthm}

\numberwithin{equation}{section}
\newtheorem*{definition}{Definicja}
\newtheorem{theorem}{Twierdzenie}[section]

\begin{document}
  \maketitle
  \newpage

  \section{Zależności w trójkątach}
    \begin{theorem}
      Jeżeli trójkąt jest prostokątny, to kwadrat długości przeciwprostokątnej jest równy sumie kwadratów długości przyprostokątnych.
      \begin{equation}
        a^2 + b^2 = c^2
      \end{equation}
    \end{theorem}
    \begin{theorem}
      W dowolnym trójkącie odcinek łączący środki dwóch boków jest równoległy do boku trzeciego i jego długość jest równa połowie długości boku trzeciego.
      \begin{equation}
        \begin{aligned}
          DE &\parallel AB\\
          |DE| &= \frac 12|AB|
        \end{aligned}
      \end{equation}
    \end{theorem}
    \begin{definition}
      \textbf{Wysokością trójkąta} nazywamy odcinek łączący wierzchołek z prostązawierającą przeciwległy bok.
    \end{definition}
    \begin{theorem}
      W dowolnym trójkącie wysokości lub ich przedłużenia przecinają się w jednym punkcie. Ten punkt to \textbf{ortocentrum}.
    \end{theorem}
    \begin{align}
      h = \frac{a\sqrt3}{2} && h = \sqrt{c_1\cdot c_2}
    \end{align}
    \begin{definition}
      \textbf{Środkową trójkąta} nazywamy odcinek łączący wierzchołek\\ trójkąta ze środkiem przeciwległego boku.
    \end{definition}
    \begin{theorem}
      W dowolnym trójkącie jego środkowe przecinają sie w jednym punkcie, który dzieli każdą z nich w stosunku 1:2. Ten punkt to \textbf{środek ciężkości trójkąta}.
    \end{theorem}
    \begin{align}
      s = h && s = \frac 12c
    \end{align}

  \section{Zależności związane z okręgami}
  %%% TO DO: %%%
  %
  % - kąty w okręgu
    \begin{theorem}
      Jeżeli przez punkt P, którego odległość od środka danego okręgu jest większa niż promień, poprowadzimy styczną do okręgu w punkcie A i sieczną przecinającą okrąg w punktach B i C, to:
      \begin{equation}
        |PA|^2 = |PB| \cdot |PC|
      \end{equation}
    \end{theorem}
    \begin{theorem}
      Jeżeli dwie proste przetną okrąg odpowiednio w punktach A i B oraz C i D, a także przecinają się w punkcie P, którego odległość od środka danego okręgu jest większa niż promień, to:
      \begin{equation}
        |PA| \cdot |PB| = |PC| \cdot |PD|
      \end{equation}
    \end{theorem}
    \begin{theorem}
      Jeżeli cięciwy AB i CD okręgu przecinają się w punkcie P, to:
      \begin{equation}
        |PA| \cdot |PB| = |PC| \cdot |PD|
      \end{equation}
    \end{theorem}
    \begin{theorem}
      W dowolnym trójkącie dwusieczna kąta dzieli przeciwległy bok na odcinki, których długość jest proporcjonalna do długości pozostałych boków.
      \begin{equation}
        \frac{|AC|}{|CD|} = \frac{|AB|}{|BD|}
      \end{equation}
    \end{theorem}
    \begin{theorem}
      Środek okręgu \textbf{opisanego} na danym trójkącie jest punktem przecięcia \textbf{symetralnych} boków trójkąta.
    \end{theorem}
    \begin{theorem}
      Środek okręgu \textbf{wpisanego} w dany trójkąt jest punktem przecięcia \textbf{dwusiecznych} kątów trójkąta.
    \end{theorem}
    \begin{equation}
      \begin{aligned}
        &r = \frac 13h = \frac{a\sqrt3}{6}\qquad&r = \frac{a+b-c}{2}\\
        &r + R = h
      \end{aligned}
    \end{equation}

  \section{Zależności w czworokątach}
  %%% TO DO: %%%
  %
  % - Deltoidy
  % - Trapezy
  % - Równoległoboki
  % - Podobieństwo
    \begin{theorem}
      Środek okręgu \textbf{opisanego} na czworokącie jest punktem przecięcia się jego symetralnych.
      \begin{equation}
        \alpha + \gamma = \beta + \delta = 180^\circ
      \end{equation}
    \end{theorem}
    \begin{theorem}
      Środek okręgu \textbf{wpisanego} w czworokąt jest punktem przecięcia się dwusiecznych jego kątów.
      \begin{equation}
        \left.
          \begin{aligned}
            |AD| + |BC| &= w+x+y+z\\
            |AB| + |CD| &= w+x+y+z
          \end{aligned}
        \right\}
        \Rightarrow
        |AD| + |BC| = |AB| + |CD|
      \end{equation}
    \end{theorem}

  \section{Pole czworokątów}
  %%% TO DO: %%%
  %
  % - Podobieństwo
  % - Wzór Herona
    \subsection{Kwadrat}
      \begin{gather}
        P = a^2 = \frac{d^2}{2}\\
        R = \frac 12d = \frac 12a\sqrt2\\
        r = \frac 12a
      \end{gather}
    \subsection{Prostokąt}
      \begin{align}
        &P = ab\\
        &P = \frac 12d^2\sin\gamma
      \end{align}
    \subsection{Romb}
      \begin{align}
        &P = ah\\
        &P = \frac{ef}{2}\\
        &P = a^2\sin\alpha
      \end{align}
    \subsection{Równoległobok}
      \begin{align}
        &P = ah_a = bh_b\\
        &P = ab\sin\alpha\\
        &P = \frac 12ef\sin\gamma
      \end{align}
    \subsection{Trapez}
      \begin{equation}
        P = \frac{(a+b)\cdot h}{2}
      \end{equation}

\end{document}
