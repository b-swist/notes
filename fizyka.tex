\documentclass{report}
\title{Fizyka}
\date{2025-02-12}
\author{Bartosz Świst}

\usepackage[utf8]{inputenc}
\usepackage[T1]{fontenc}
\usepackage[polish]{babel}
\usepackage{amsmath, amssymb, amsthm, tikz}

\DeclareMathOperator{\tg}{tg}
\DeclareMathOperator{\const}{const.}
\newcommand{\unit}[1]{\,\left[\mathrm{#1}\right]}
\newcommand{\tab}{\hspace*{3mm}}
\newcommand{\bi}[1]{\textbf{\textit{#1}}}
\renewcommand{\chaptername}{Rozdział}

\newtheoremstyle{colon}{\topsep}{\topsep}{\itshape}{}{\bfseries}{:}{5pt plus 1pt minus 1pt}{}
\theoremstyle{colon}
\newtheorem*{inner}{\innerheader}
\newcommand{\innerheader}{}
\newenvironment{law}[1]{\renewcommand\innerheader{#1}\begin{inner}}{\end{inner}}
\theoremstyle{definition}
\newtheorem*{definition}{Definicja}

\begin{document}
  \maketitle
  \chapter{Kinematyka}

\section{Wektory}

\subsection{Iloczyn skalarny}
\begin{equation}
  c = \vec a \cdot \vec b = |\vec a| \cdot |\vec b| \cdot \cos\measuredangle(\vec a, \vec b)
\end{equation}

\subsection{Iloczyn wektorowy}
\begin{equation}
  \vec c = \vec a \times \vec b =|\vec a| \cdot |\vec b| \cdot \sin\measuredangle(\vec a, \vec b)
\end{equation}

\section{Opis ruchu}
\begin{gather*}
  v_\text{śr} = \frac s t \unit{\frac m s}\\
  \vec v_\text{śr} = \frac{\Delta \vec x}{\Delta t} \unit{\frac m s}\\
  \vec a = \frac{\Delta \vec v}{\Delta t} \unit{\frac{m}{s^2}}
\end{gather*}

\section{Ruch jednostajny prostoliniowy}
\begin{gather*}
  v = \const\\
  v = \frac s t = \tg \alpha\\
  x(t) = x_0 \pm vt
\end{gather*}

\section{Ruch jednostajnie przyspieszony}
\begin{gather*}
  a = \const\\
  a = \frac{\Delta v}{\Delta t}\\
  v_k = v_0 + at\\
  s = v_0 t + \frac{a t^2}{2}
\end{gather*}
jeżeli $v_0 = 0$:
\begin{gather*}
  v = at\\
  s = \frac{a t^2}{2}
\end{gather*}

\subsection{Stosunek przebytych dróg do odcinków czasu}
\begin{equation*}
  s_1:s_2:s_3:s_4:s_5:\ldots = 1:3:5:7:9:\ldots
\end{equation*}

\section{Ruch jednostajnie opóźniony}
\begin{gather*}
  v_k = v_0 - at\\[0.8em]
  \begin{aligned}
    s &= v_0 t - \frac{a t^2}{2}\\
    &= v_0 t - \frac 1 2 \Delta vt
  \end{aligned}
\end{gather*}
jeżeli $v_k = 0$:
\begin{gather*}
  v_0 = at\\
  s = \frac 1 2 v_0 t
\end{gather*}

\section{Rzut pionowy}

\subsection{Wznoszenie się}
\begin{gather*}
  h = v_0 t - \frac{g t^2}{2}\\
  v = v_0 - gt
\end{gather*}

\subsection{Opadanie}
\begin{gather*}
  h = v_0t + \frac{gt^2}{2}\\
  v = v_0 + gt
\end{gather*}
jeżeli $v_0 = 0$:
\begin{equation*}
  v = gt
\end{equation*}

\section{Rzut poziomy}
\begin{gather*}
  h = \frac{g t^2}{2}\\
  x = v_0 t = v_0 \sqrt{\frac{2h}{g}}\\
  v = \sqrt{v_0^2 + v_y^2} = \sqrt{v_0^2 + (gt)^2}
\end{gather*}

\section{Rzut ukośny}
% \begin{center}
%   \begin{tikzpicture}
%     \draw[gray, very thin] (0, 0) grid (9.8, 3.8);
%     \draw[->, thin] (-0.3, 0) -- (10, 0) node[below] {$x$};
%     \draw[->, thin] (0, -0.3) -- (0, 4) node[left] {$y$};
%     \draw (4, 3) parabola (0, 0);
%     \draw (4, 3) parabola (8, 0);
%
%     \draw[->, thick] (0, 0) -- (0.8, 0) node[midway, below] {$\vec v_{0_x}$};
%     \draw[->, thick] (0, 0) -- (0, 1) node[midway, left] {$\vec v_{0_y}$};
%     \draw[->, thick] (0, 0) -- (0.8, 1) node[midway, above, xshift=-1] {$\vec v_0$};
%     \draw (0.6, 0) arc (0:50:0.6) node[below, xshift=-0.8, yshift=-2.5] {$\alpha$};
%
%     \draw[->, thick] (4, 3) -- (4.8, 3) node[midway, below] {$\vec v_{0_x}$};
%
%     \draw[->, thick] (8, 0) -- (8.8, 0) node[midway, above] {$\vec v_{0_x}$};
%     \draw[->, thick] (8, 0) -- (8, -1) node[midway, left] {$\vec v_y$};
%     \draw[->, thick] (8, 0) -- (8.8, -1) node[midway, right] {$\vec v$};
%
%     \draw[dotted, thick] (4, 0) -- (4, 3) node[midway, right] {$h_{max}$};
%   \end{tikzpicture}
% \end{center}
\begin{align*}
  v_{0_x} &= v_0 \cos \alpha\\
  v_{0_y} &= v_0 \sin \alpha
\end{align*}

\begin{equation}
  y(x) = x \tg \alpha - x^2 \cdot \frac{g}{2 v_0^2 \cos^2 \alpha}
\end{equation}

\begin{gather*}
  t_c = \frac{2 v_0 \sin \alpha}{g}\\
  h_\text{max} = \frac{v_0^2 \sin^2 \alpha}{2g}\\
  z = \frac{v_0^2 \sin 2\alpha}{g}
\end{gather*}
dla $\alpha = 45^\circ$:
\begin{equation*}
  z = \frac{v_0^2}{2g}
\end{equation*}

\section{Ruch jednostajny po okręgu}

\begin{gather*}
  \alpha = \frac Lr \unit{rad}\\
  f = \frac nt \unit{Hz}\\
  \omega = \frac{\Delta \alpha}{\Delta t} \unit{\frac{rad}{s}}
\end{gather*}
dla jednego obrotu:
\begin{gather*}
  f = \frac 1 T\\
  \omega = \frac{2\pi}{T} = 2\pi f
\end{gather*}

\begin{gather*}
  v = \omega r\\
  a_r = \frac{v^2}{r}
\end{gather*}

\begin{gather*}
  \vec v = \vec \omega \times \vec r\\
  v = \omega r \sin\measuredangle(\vec \omega, \vec r)
\end{gather*}
dla $\vec \omega \perp \vec r$:
\begin{equation*}
  v = \omega r
\end{equation*}

\section{Przyspieszenie w ruchu po okręgu}
\begin{gather*}
  \vec a_s = \frac{\Delta \vec v}{\Delta t}\\[0.5em]
  a_w = \sqrt{a_r^2 + a_s^2}
\end{gather*}

  \chapter{Dynamika}

\section{Zasady dynamiki Newtona}

\subsection{Pierwsza zasada}
\begin{subequations}
  \begin{equation}
    \vec F_w = 0 \Rightarrow \vec v = 0 \lor \vec v = \const
  \end{equation}

  \subsection{Druga zasada}
  \begin{gather}
    \begin{aligned}
      F_w \ne 0 &\Rightarrow a = \const\\
      \vec a = \frac{\vec F_w}{m} &\Rightarrow \vec F_w = m \vec a \unit N
    \end{aligned}
  \end{gather}

  \subsection{Trzecia zasada}
  \begin{equation}
    \begin{aligned}
      \vec F_{AB} &= -\vec F_{BA}\\
      F_{AB} &= F_{BA}
    \end{aligned}
  \end{equation}
\end{subequations}

\section{Ruch na równi pochyłej}
\begin{align*}
  \frac{\vec F_Z}{\vec F_g} = \sin \alpha &\Rightarrow \vec F_Z = \vec F_g \sin \alpha = mg \sin \alpha\\
  \frac{\vec F_N}{\vec F_g} = \cos \alpha &\Rightarrow \vec F_N = \vec F_g \cos \alpha = mg \cos \alpha
\end{align*}

%% TODO: add a section

\section{Pęd ciała}
\begin{equation}
  \vec p = m \vec v \unit{\frac{kg\cdot m}{s}}
\end{equation}

\begin{equation*}
  \Delta p = F \Delta t
\end{equation*}

\subsection{Zasada zachowania pędu}
\begin{equation*}
  \Delta \vec p = 0 \Leftrightarrow \vec p = \const
\end{equation*}

\section{Środek masy}
\begin{equation*}
  x_c = \frac{\sum_{i=1}^n m_i x_i}{\sum_{i=1}^n m_i}
\end{equation*}

\section{Tarcie}
\begin{align*}
  T_s &= \mu_s F_N \unit N\\
  T_k &\leqslant \mu_k F_N \unit N
\end{align*}

\section{Siła dośrodkowa}
\begin{equation*}
  F_{do} = \frac{mv^2}{r} \unit N
\end{equation*}

\section{Siła bezwładności}
\begin{equation*}
  \vec F_b = -m \vec a \unit N
\end{equation*}

% vim:spell:spl=pl

  \chapter{Praca, moc, energia}

\section{Praca}
\begin{equation}
  \begin{gathered}
    W = \vec F \Delta \vec r \unit J\\
    W = F \Delta r \cos\measuredangle(\vec F, \Delta \vec r)
  \end{gathered}
\end{equation}
dla $\alpha = 0^\circ$:
\begin{equation*}
  W = F \Delta r = Fs
\end{equation*}
dla $\alpha = 90^\circ$:
\begin{equation*}
  W = 0
\end{equation*}

\section{Moc}
\begin{equation}
  P = \frac W t \unit W
\end{equation}
dla $v = \const$:
\begin{equation*}
  P = Fs
\end{equation*}

\section{Energia mechaniczna}

\subsection{Energia kinetyczna}
\begin{equation}
  \begin{gathered}
    E_k = \frac{mv^2}{2} \unit J\\
    \Delta E_k = W
  \end{gathered}
\end{equation}

\subsection{Energia potencjalna}
\begin{equation}
  \begin{gathered}
    E_p = mgh \unit J\\
    \Delta E_p = W
  \end{gathered}
\end{equation}

\subsection{Zasada zachowania energii}
\begin{gather*}
  E_c = \const\\
  E_c = E_k + E_p\\
  \Delta E_c = \Delta E_p + \Delta E_k
\end{gather*}

\section{Sprawność}
\begin{equation}
  \eta = \frac{E_\text{użyt.}}{E_\text{pob.}} = \frac{W_\text{użyt.}}{E_\text{pob.}}
\end{equation}

\begin{equation*}
  \eta_\text{użyt.} = \prod_{i=1}^n \eta_i
\end{equation*}

% vim:spell:spl=pl

  \chapter{Hydrostatyka}

\section{Ciśnienie i parcie}

\subsection{Ciśnienie}
\begin{equation}
  p = \frac{F_N}{S} \unit{Pa}
\end{equation}
dla $F_N = m g$:
\begin{equation}
  p = \frac{m g}{S}
\end{equation}

\subsection{Parcie}
\begin{equation*}
  P = pS \unit N
\end{equation*}

\subsection{Ciśnienie hydrostatyczne}
\begin{equation*}
  p_h = \frac P S = \varrho_c gh \unit{Pa}
\end{equation*}

\subsection{Paradoks hydrostatyczny}

\section{Prawo Pascala}
\begin{equation}
  p_1 = p_2 \Rightarrow \frac{F_1}{S_1} = \frac{F_2}{S_2}
\end{equation}

\subsection{Naczynia połączone}
\begin{equation*}
  \varrho_1 h_1 = \varrho_2 h_2
\end{equation*}

\section{Prawo Archimedesa}
\begin{equation}
  F_W = \varrho_c g V_z \unit N
\end{equation}

\subsection{Warunki wypływania}
\begin{align*}
  F_W > F_g &\Rightarrow \text{ciało wypływa}\\
  F_W = F_g &\Rightarrow \text{ciało pływa}\\
  F_W < F_g &\Rightarrow \text{ciało tonie}
\end{align*}

  \chapter{Bryła sztywna}
  \section{Ruch obrotowy}
    \subsection{Prędkość kątowa}
      \begin{equation}
        \omega = \frac{\Delta\alpha}{\Delta t} \unit{\frac{rad}{s},\;\frac{1}{s}}
      \end{equation}
    \subsection{przyśpieszenie kątowe}
      \begin{equation}
        \varepsilon = \frac{\Delta\omega}{\Delta t} \unit{\frac{rad}{s^2},\;\frac{1}{s^2}}
      \end{equation}
    \subsection{Prędkość liniowa (styczna)}
      \begin{gather}
        \vec v = \vec\omega \times \vec r\unit{\frac{m}{s}}\\
        v = \omega r\sin\measuredangle(\vec\omega, \vec r)
      \end{gather}
      dla $\vec\omega \perp \vec r$:
      \begin{equation}
        v =\omega r
      \end{equation}
    \subsection{przyśpieszenie liniowe}
      \begin{equation}
        a_r = \varepsilon r \unit{\frac{m}{s^2}}
      \end{equation}
    \section{Równania obrotu}
      \begin{gather}
        \omega = \omega_0 \pm \varepsilon t\\
        \alpha = \omega_0t \pm \frac{\varepsilon t^2}{2}\\
      \end{gather}
      dla $\omega_0 = 0$:
      \begin{equation}
        \alpha = \frac{1}{2} \omega t
      \end{equation}
  \section{Moment bezwładności}
    \begin{equation}
      I = \sum_{i=1}^n m_ir_i^2 \unit{kg\cdot m^2}
    \end{equation}
      \subsection{Momenty bezwładności wybranych brył}
      kula: $I_0 = \frac{2}{5}mr^2$\\
      walec: $I_0 = \frac{1}{2}mr^2$\\
      pręt: $I_0 = \frac{1}{12}ml^2$\\
      rura grubościenna: $I_0 = \frac{1}{2}m(r_1^2 + r_2^2)$
    \subsection{Twierdzenie Steinera}
      \begin{equation}
        I = I_0 +mx^2
      \end{equation}
  \section{Energia kinetyczna}
    \begin{gather}
      E_{k_o} = \sum_{i=1}^n \frac{m_iv_i}{2} \unit{J}\\
      E_{k_o} = \frac{I\omega^2}{2}
    \end{gather}
  \section{Moment siły}
    \begin{gather}
      \vec M = \vec r\times\vec F \unit{N\cdot m}\\
      M = rF\sin\measuredangle(\vec r, \vec F)
    \end{gather}
    dla $\vec r \perp \vec F$:
    \begin{equation}
      M = rF
    \end{equation}
    dla $\vec r \parallel \vec F$:
    \begin{equation}
      M = 0
    \end{equation}
    \subsection{Wypadkowy moment siły}
      \begin{gather}
        M_w = \sum_{i=1}^n M_i\\
        M_w = \varepsilon I
      \end{gather}
    \subsection{Równowaga bryły sztywnej}
      \begin{gather}
        F_w = 0\\
        M_w = 0
      \end{gather}
  \section{Moment pędu}
    \begin{gather}
      \vec L = \vec r \times\vec p \unit{\frac{kg\cdot m^2}{s}}\\
      L = rp\sin\measuredangle(\vec r,\vec p)\\
      L = mrv\sin\measuredangle(\vec r,\vec v)
    \end{gather}
    dla $\vec p \perp \vec r$:
    \begin{equation}
      L = rp = mrv
    \end{equation}
    \begin{equation}
      L = \sum_{i=1}^n m_ir_iv_i\sin\measuredangle(\vec r,\vec v)
    \end{equation}
    dla $\vec r \perp \vec v$:
    \begin{equation}
      L = \omega I
    \end{equation}

  \chapter{Grawitacja}
  \section{Prawa Keplera}
    \begin{law}{Pierwsze prawo Keplera}
      Planety poruszają się po elipsach, a jednym z jej ognisk znajduje się Słońce.
    \end{law}
    \begin{law}{Drugie prawo Keplera}
      Promień wodzący planety w jednolitych odstępach czasu zakreśla jednolite pola.
      \begin{align*}
        s_1 &= s_2\\
        L_1 = L_2 &\Rightarrow r_1v_1 = r_2v_2
      \end{align*}
    \end{law}
    \begin{law}{Trzecie prawo Keplera}
      Iloraz kwadratu okresu obiegu planety wokół Słońca i sześcianu średniej odległości planety od Słońca jest jednakowy dla wszystkich planet Układu Słonecznego.
      \begin{equation*}
        \frac{T^2}{r^3} = \const
      \end{equation*}
    \end{law}
  \section{Prawo powszechnego ciążenia}
    \begin{law}{Prawo powszechnego ciążenia}
      Siła wzajemnego oddziaływania dwóch ciał jest wprost proporcjonalna do iloczynu mas tych ciał i odwrotnie proporcjonalna do kwadratu odległości między ich średkami mas.
      \begin{equation}
        F = G\frac{m_1m_2}{r^2} \unit{N}
      \end{equation}
      \begin{symbols}
        \item $G$ -- stała grawitacji
      \end{symbols}
      \begin{equation*}
        G = 6,67\cdot 10^{-11} \unit{\frac{N\cdot m^2}{kg^2}}
      \end{equation*}
    \end{law}
    \subsection{Siły grawitacji wewnątrz lub na powierzchni planety}
      \begin{gather*}
        F = \frac 43\pi RGdm\\
        F \sim dR
      \end{gather*}
  \section{Pole grawitacyjne}
    \begin{definition}
      \bi{Pole grawitacyjne} to przestrzeń, w której na umieszczone tam ciała gdziała siła grawitacji. 
    \end{definition}
    \subsection{Natężenie pola grawitacyjnego}
      \begin{definition}
        \bi{Natężenie pola grawitacyjnego} to stosunek siły grawitacji działającej na ciało próbne do masy tego ciała.
        \begin{subequations}
          \begin{equation}
            \vec\gamma = \frac{\vec F_g}{m} \units{\frac{N}{kg}}{\frac{m}{s^2}}
          \end{equation}
          dla pola centralnego:
          \begin{equation}
            \gamma = \frac{GM}{r^2}
          \end{equation}
        \end{subequations}
      \end{definition}
    \subsection{Praca w polu grawitacyjnym}
      \begin{equation}
        \begin{gathered}
          W_{Z_{(A\rightarrow B)}} = GMm\left(\frac{1}{r_A} - \frac{1}{r_B}\right)\\
          W_{g_{(A\rightarrow B)}} = -W_{Z_{(A\rightarrow B)}}
        \end{gathered}
      \end{equation}
    \subsection{Energia w polu grawitacyjnym}
      \begin{equation}
        E_p = -\frac{GMm}{r}
      \end{equation}
    \subsection{Potencjał pola grawitacyjnego}
      \begin{definition}
        \bi{Potencjał pola grawitacyjnego} to stosunek energii potencjalnej ciała próbnego do jego masy.
        \begin{subequations}
          \begin{gather}
            V = \frac{E_p}{m} \unit{\frac{J}{kg}}\\
            \Delta V = \frac{\Delta E_p}{m}
          \end{gather}
        \end{subequations}
      \end{definition}
  \section{Prędkości kosmiczne}
    \begin{definition}
      \bi{Pierwsza prędkość kosmiczna} to prędkość, jaką trzeba nadać ciału by oderwało się z powierzchni źródła pola grawitacyjnego i krążyło wokół niego po orbicie kołowej.
      \begin{equation*}
        v_{\scriptscriptstyle\mathrm I} = \sqrt\frac{GM}{r}
      \end{equation*}
    \end{definition}
    \begin{definition}
      \bi{Druga prędkość kosmiczna} to prędkość, jaką trzeba nadać ciału by oddaliło się nieskończenie daleko od źródła pola grawitacyjnego.
      \begin{equation*}
        v_{\scriptscriptstyle\mathrm{II}} = \sqrt{\frac{2GM}{r}} = v_{\scriptscriptstyle\mathrm I}\sqrt2
      \end{equation*}
    \end{definition}
    \subsection{Satelita geostacjonarny}
      \begin{definition}
        \bi{Satelita geostacjonarny} to taki satelita, który krąży nad powierzchnią w płaszczyźnie równika nad tym samym punktem (jego okres obiegu jest równy okresowi obrotu planety).
        \begin{equation*}
          r = \sqrt[3]{\frac{GMT^2}{4\pi^2}}
        \end{equation*}
      \end{definition}

  \chapter{Ruch drgający}
\begin{gather}
  F_z = kx\\
  F_s = -kx
\end{gather}

\begin{equation*}
  k = \left|\frac{F_s}{x}\right| \unit{\frac N m}
\end{equation*}

\section{Ruch harmoniczny}
\begin{gather*}
  x = r \sin \alpha\\
  T = 2\pi \sqrt{\frac m k} \unit s
\end{gather*}

\subsection{Równania ruchu harmonicznego}
\begin{subequations}
  \begin{align}
    x(t) &= A \sin(\omega t + \varphi_0)\\
    v(t) &= \omega A\ cos(\omega t + \varphi_0)\\
    a(t) &= -\omega^2 A \sin(\omega t + \varphi_0)
  \end{align}
\end{subequations}

\begin{align*}
  x_\text{max} &= A\ \text{dla}\ \sin90^\circ = 1\\
  v_\text{max} &= \omega A\ \text{dla}\ \cos0^\circ = 1\\
  a_\text{max} &= -\omega^2 A\ \text{dla}\ \sin90^\circ = 1\\
\end{align*}

\subsection{Łączenie sprężyn}
\begin{align*}
  F &= \const & x &= \const\\
  x &= \sum_{i=1}^n x_i & F_c &= \sum_{i=1}^n F_i\\
  \frac 1 k &= \sum_{i=1}^n \frac{1}{k_i} & k &= \sum_{i=1}^n k_i
\end{align*}

\section{Energia w ruchu harmonicznym}
\begin{equation*}
  W = \frac 1 2 Fx \Rightarrow E_{p_s} = \frac 1 2 kx^2
\end{equation*}

\begin{align*}
  E_c &= E_{p_s} + E_k = \frac 1 2 kA^2\\
  E_k &= \frac 1 2 k(A^2 - x^2)
\end{align*}

\section{Wahadło matematyczne}
\begin{equation*}
  F = F_g \sin \alpha
\end{equation*}
dla małych kątów $\sin \alpha \approx \alpha$:
\begin{gather*}
  F = mg\alpha\\
  T = 2\pi\sqrt{\frac lg}
\end{gather*}

  \chapter{Termodynamika}
  \section{Zerowa zasada dynamiki}
    \begin{equation}
      p = \frac{2}{3}\cdot\frac{NE_{k_{\acute sr.}}}{V} \unit{Pa}
    \end{equation}
    gdzie $N$ - liczba cząstek gazu
    \begin{equation}
      E_{k_{\acute sr.}} = \frac{1}{2}mv_{\acute sr.}^2 \unit{J}
    \end{equation}
  \section{Równanie gazu doskonałego}
    \begin{equation}
      \frac{p_1V_1}{T_1} = \frac{p_2V_2}{T_2} \Rightarrow \frac{pV}{T} = \const
    \end{equation}
    \subsection{Równanie Clapeyrona}
      \begin{equation}
        pV = nRT = NkT
      \end{equation}
      gdzie:
      \begin{gather}
        R =  8,31 \unit{\frac{J}{mol\cdot K}}\\
        k = \frac{R}{N_A} = 1,38\cdot 10^{-23} \unit{\frac{J}{K}}
      \end{gather}
  \section{Przemiany gazu doskonałego}
    \subsection{Przemiana izotermiczna}
      \begin{gather}
        T = \const\\
        \frac{p_1V_1}{T} = \frac{p_2V_2}{T} \Rightarrow p_1V_1 = p_2V_2\\
        pV = \const \Rightarrow p = \frac{const.}{V}\\
        \text{(Prawo Boyle'a)}
      \end{gather}
    \subsection{Przemiana izochoryczna}
      \begin{gather}
        V = \const\\
        \frac{p_1V}{T_1} = \frac{p_2V}{T_2} \Rightarrow \frac{p_1}{T_1} = \frac{p_2}{T_2}\\
        \frac{p}{T} = \const \Rightarrow p = T\cdot const.\\\text{(Prawo Charles'a)}
      \end{gather}
      % \begin{tikzpicture}
      %   \draw[->] (-0.3, 0) -- (4, 0) node[below] {$T$};
      %   \draw[->] (0, -0.3) -- (0, 3.5) node[left] {$p$};
      %   \draw[thick] (0.58, 0.5) -- (3.5, 3);
      %   \draw[dotted] (0, 0) -- (0.58, 0.5);
      % \end{tikzpicture}
    \subsection{Przemiana izobaryczna}
      \begin{gather}
        p = \const\\
        \frac{pV_1}{T_1} = \frac{pV_2}{T_2} \Rightarrow \frac{V_1}{T_1} = \frac{V_2}{T_2}\\
        \frac{V}{T} = \const \Rightarrow V = T\cdot const.\\
        \text{(Prawo Gay-Lussaca)}
      \end{gather}
      % \begin{tikzpicture}
      %   \draw[->] (-0.3, 0) -- (4, 0) node[below] {$V$};
      %   \draw[->] (0, -0.3) -- (0, 3.5) node[left] {$T$};
      %   \draw[thick] (0.58, 0.5) -- (3.5, 3);
      %   \draw[dotted] (0, 0) -- (0.58, 0.5);
      % \end{tikzpicture}
  \section{Pierwsza zasada termodynamiki}
    \begin{equation}
      \Delta U = Q + W_z \unit{J}
    \end{equation}
    dla $Q > 0$ ciepło zostało pobrane\\
    dla $Q < 0$ ciepło zostało oddane
    \begin{gather}
      W_z = F_z\Delta x\cos\measuredangle(\vec F_z, \Delta\vec x)\\
      W_z = -W_{gazu}
    \end{gather}
    dla $W_z > 0$:
    \begin{equation}
      W_z = F_z\Delta x
    \end{equation}
    dla $W_z < 0$:
    \begin{equation}
      W_z = -F_z\Delta x
    \end{equation}
    \begin{equation}
      |W| = p|\Delta V|
    \end{equation}
  \section{Energia wewnętrzna gazu doskonałego}
    \begin{gather}
      U = N\cdot\frac{i}{2}kT\\
      \Delta U = N\cdot\frac{i}{2}k\Delta T
    \end{gather}
    gdzie $i$ - stopnie swobody cząstek
    \subsection{Przemiana izotermiczna}
      \begin{gather}
        T = \const \Leftrightarrow U = const.\\
        \Delta U = 0 \Rightarrow Q + W = 0
      \end{gather}
    \subsection{Przemiana izochoryczna}
      \begin{gather}
        V = \const \Rightarrow \Delta V = 0\\
        W = 0 \Rightarrow \Delta U = Q
      \end{gather}
    \subsection{Przemiana adiabatyczna}
      \begin{gather}
        Q = 0 \Rightarrow \Delta U = W\\
        pV^\kappa = \const
      \end{gather}
      gdzie:
      \begin{equation}
        \kappa = \frac{C_p}{C_V}
      \end{equation}
  \section{Ciepło molowe i właściwe}
    \subsection{Ciepło właściwe}
      \begin{gather}
        C_w = \frac{Q}{m\Delta T} \unit{\frac{J}{kg\cdot K}}\\
        Q = mC_w\Delta T
      \end{gather}
    \subsection{Ciepło molowe}
      \begin{equation}
        C = \frac{Q}{n\Delta T} \unit{\frac{J}{mol\cdot K}}
      \end{equation}
      ciepło molowe przy stałym ciśnieniu: $C_p$\\
      ciepło molowe przy stałej objętości: $C_V$
      \begin{align}
        &Q_p = Q_V + p\Delta V\\
        &C_p = C_V + R
      \end{align}
  \section{Energia wewnętrzna jako funkcja stanu}
    \begin{equation}
      \Delta U = Q_V = nC_V\Delta T
    \end{equation}
    \section{Silnik cieplny}
      % \begin{center}
      %   \begin{tikzpicture}[scale=1.5]
      %     \draw[->] (-0.3, 0) -- (4, 0) node[below] {$V$};
      %     \draw[->] (0, -0.3) -- (0, 3.5) node[left] {$p$};
      %     \draw[thick, ->] (1, 3) -- (2.5, 2.6);
      %     \draw[thick, ->] (2.5, 2.6) -- (3, 1);
      %     \draw[thick, <-] (1, 3) arc (180:270:2);
      %     \draw (2.05, 2.05) node {\LARGE $W$};
      %   \end{tikzpicture}
      % \end{center}
    \begin{equation}
      \eta = \frac{|Q_1|-|Q_2|}{Q_1} = \frac{T_1 - T_2}{T_1}
    \end{equation}
  \section{Przejścia fazowe}
    \begin{equation}
      Q = mC_w\Delta T
    \end{equation}
    woda - lód: $T_T = T_K = 0^\circ C$\\
    woda - para wodna: $T_W = T_S = 100^\circ C$
    \begin{equation}
      Q = mL
    \end{equation}
    \begin{equation}
      Q = mR
    \end{equation}
  \section{Rozszerzalność temperaturowa ciał}
    \subsection{Rozszerzalność obiętościowa}
      \begin{equation}
        \Delta V = V_0\alpha\Delta T
      \end{equation}
    \subsection{Rozszerzalność liniowa}
      \begin{equation}
        \Delta l = l_0\lambda\Delta T
      \end{equation}

  \chapter{Elektrostatyka}
  \section{Ładunek elektryczny}
    \begin{gather*}
      |e| = 1,6\cdot 10^{-19} \unit C\\
      q = ne \unit{C}
    \end{gather*}
    \begin{law}{Zasada zachowania ładunku}
      W izolowanym układzie całkowity ładunek elektryczny nie ulega zmianie.
      \begin{equation}
        \sum_{i=1}^n q_i = \const
      \end{equation}
    \end{law}

  \section{Prawo Coulomba}
    \begin{law}{Prawo Coulomba}
      Siła wzajemnego oddziaływania dwóch ładunków jest wprost proporcjonalna do iloczynu tych ładunków i odwrotnie proporcjonalna do kwadratu odległości między nimi.
      \begin{equation}\label{coulomb}
        \boxed{F = k\frac{Qq}{r^2} \unit N}
      \end{equation}
      \begin{symbols}
        \item $k$ -- współczynnik proporcjonalności (stała eletrostatyczna)
        \item $\varepsilon_0$ -- stała przenikalności elektrycznej próżni
      \end{symbols}
      \begin{gather*}
        k = \frac{1}{4\pi\varepsilon_0} \approx 8,99\cdot 10^9 \unit{\frac{N\cdot m^2}{C^2}}\\[0.5em]
        \varepsilon_0 = 8,85\cdot 10^{-12} \unit{\frac{C^2}{N\cdot m^2}}
      \end{gather*}
    \end{law}

  \section{Natężenie pola elektrostatycznego}
    \begin{definition}
      \bi{Natężenie pola elektrostatycznego} to stosunek siły elektrostatycznej działającej na \emph{dodatni} ładunek próbny $q$ w danym punkcie pola do~wartości tego ładunku.
      \begin{subequations}
        \begin{equation}\label{natężenie es}
          \vec E = \frac{\vec F}{q} \unit{\frac NC}\\[0.5em]
        \end{equation}
        Wstawiając równanie \ref{coulomb} do równania \ref{natężenie es} otrzymujemy:
        \begin{equation}
          \boxed{E = \frac{kQ}{r^2}}
        \end{equation}
      \end{subequations}
    \end{definition}

  \section{Rozmieszczenie ładunku na przewodniku}
    Po namagnesowaniu ciała cały dostarczony ładunek rozmieszcza się na jego zewnętrznej powierzchni. Pole elektromagnetyczne wewnątrz zanika.

    Rozmieszczenie ładunku na powierzchni zależy od jego kształu. Rozkład ładunku opisuje gęstość powierzchniowa ładunku --- iloraz ładunku i pole tej powierzchni.
    \begin{equation*}
      \sigma = \frac QS \unit{\frac{C}{m^2}}
    \end{equation*}

  \section{Praca w polu centralnym}
    \begin{equation}\label{praca es}
      W_{A\rightarrow B} = -kQq\left(\frac{1}{r_A} - \frac{1}{r_B}\right) \unit J
    \end{equation}

  \section{Energia w polu centralnym}
  \begin{equation}\label{energia es}
      E_p = \frac{kQq}{r} \unit J
    \end{equation}

  \section{Potencjał w polu centralnym}
    \begin{definition}
      Potencjał pola elektrostatycznego to stosunek energii potencjalnej punktowego ciała do wartości ładunku próbnego umieszczonego w tym polu.
      \begin{subequations}
        \begin{equation}\label{potencjał es}
          V = \frac{E_p}{q} \unit V\\
        \end{equation}
        Podstawiajac równanie \ref{energia es} do równania \ref{potencjał es} otrzymujemy:
        \begin{equation}
          V = \frac{kQ}{r^2}
        \end{equation}
      \end{subequations}
    \end{definition}
    \begin{equation*}
        W = \Delta E_p = q\Delta V = qU
    \end{equation*}

  \section{Pojemność elektryczna przewodnika}
    \begin{definition}
      \bi{Pojemność elektryczna przewodnika} to stosunek ilości ładunku zgromadzonego na przewodniku do uzyskanego potencjału.
      \begin{equation}
        C = \frac QV \unit F
      \end{equation}
    \end{definition}

  \section{Kondensator}
    \begin{definition}
      \bi{Kondensator} to element elektroniczny służący do gromadzenia ładunku elektrycznego.
      \begin{equation*}
        C = \frac{Q}{U}
      \end{equation*}
    \end{definition}

    \subsection{Łączenie kondensatorów}
      \begin{align*}
        U &= \const & Q &= \const\\
        Q &= \sum_{i=1}^n Q_i & U &= \sum_{i=1}^n U_i\\
        C_z &= \sum_{i=1}^n C_i & \frac{1}{C_z} &= \sum_{i=1}^n \frac{1}{C_i}
      \end{align*}

    \subsection{Kondensator płaski}
      \begin{definition}
        Kondensator płaski składa się z dwóch równoległych, metalowych okładek, między którymi znajduje się dielektryk.\\[1.5em]
        Bez dielektryka:
        \begin{equation*}
          C = \frac{\varepsilon_0S}{d}
        \end{equation*}
        Uwzględniając dielektryk:
        \begin{equation*}
          C = \frac{\varepsilon_0\varepsilon_rS}{d}
        \end{equation*}
        Natężenie pola elektrycznego kondensatora:
        \begin{equation*}
          E = \frac Ud
        \end{equation*}
        \begin{symbols}
          \item $d$ -- odległość między okładkami
          \item $S$ -- pole powierzchni okładek
          \item $\varepsilon_r$ -- stała przenikalność dielektryka
        \end{symbols}
      \end{definition}

    \subsection{Energia naładowaniego kondensatora}
      \begin{definition}
        \bi{energia naładowaniego kondensatora} to praca potrzebna do jego naładowania.
        \begin{gather*}
          E = \frac{1}{2}QU = \frac{1}{2}CU^2 = \frac{Q^2}{2C} \units{eV}{J}\\[0.5em]
          (1\mathrm{eV} = 1,6\cdot 10^{-19}\mathrm{J})
        \end{gather*}
      \end{definition}
  \section{Ruch ładunków w polu elektrostatycznym}
    Na naładowane cząstki w polu elektrostatycznym centralnym działa siła elektrostatyczna, przez którą cząstka zaczyna przyspieszać.
    \begin{equation*}
      F = qE
    \end{equation*}
    jeżeli $F = F_w$, to:
    \begin{equation*}
      a = \frac{qE}{m} = \frac{qU}{md} \unit{\frac{m}{s^2}}
    \end{equation*}

    \chapter{Prąd elektryczny}
    \section{Natężenie prądu elektrycznego}
      \begin{equation}
        I = \frac{\Delta Q}{\Delta t} \unit{A}
      \end{equation}
    \section{Pierwsze prawo Kirchoffa}
      \begin{equation}
        \sum_{i=1}^n I_i = 0
      \end{equation}
    \section{Prawo Ohma}
      \begin{gather}
        \frac{U}{I} = const.\\
        R = \frac{U}{I} \unit{\Omega}
      \end{gather}
      \subsection{Wybrane charakterystyki prądowo-napięciowe}
      \subsection{Łączenie rezystorów}
        \begin{gather}
          I = const.\\
          U = \sum_{i=1}^n U_i\\
          R_z = \sum_{i=1}^n R_i
        \end{gather}
        \begin{gather}
          U = const.\\
          I = \sum_{i=1}^n I_i\\
          \frac{1}{R_z} = \sum_{i=1}^n \frac{1}{R_i}
        \end{gather}
      \subsection{Opór przewodnika}
        \begin{equation}
          R = \frac{\varrho l}{S} \unit{\Omega}
        \end{equation}
      gdzie $\varrho$ - opór właściwy materiału przewodnika
    \section{Praca prądu elektrycznego}
      \begin{equation}
        W = UIt = \frac{U^2}{R}t = I^2Rt
      \end{equation}
      \subsection{Emisja ciepła}
        \begin{equation}
          Q = W = I^2Rt \quad\text{(Ciepło Joule'a)}
        \end{equation}
      \subsection{Energia elektryczna}
        \begin{equation}
          E_{el} = W = UIt \unit{kWh}
        \end{equation}
    \section{Moc prądu elektrycznego}
      \begin{gather}
        P = \frac{W}{t} \unit{W}\\
        P = UI = I^2R = \frac{U^2}{R}
      \end{gather}
    \section{Ogniwo galwaniczne}
      \subsection{Prawo Ohma dla obwodu}
        \begin{gather}
          \mathcal E = U + U_w = U + IR_w\\
          U = \mathcal E - IR_w
        \end{gather}
        \begin{equation}
          I = \frac{\mathcal E}{R_z} = \frac{\mathcal E}{R + R_w} \unit{A}
        \end{equation}
    \section{Drugie prawo Kirchoffa}
      \begin{equation}
        \sum_{i=1}^n \mathcal E_i + \sum_{i=1}^n I_iR_i = 0
      \end{equation}
    \section{Przewodnictwo ciał stałych}
    \section{Dioda półprzewodnikowa}

  \chapter{Magnetyzm}
  \section{Pole magnetyczne}
    \begin{definition}
      \textbf{Pole magnetyczne} to przestrzeń, w której na umieszczone w niej naładowane cząstki oraz ciała o~właściwościach magnetycznych działają siły magnetyczne. Ciała wykazujące właściwości magnetyczne posiadają \textbf{domeny magnetyczne} --- obszary o~stałym namagnesowaniu. Źródłem pola magnetycznego może być np. magnes trwały lub~przewodnik z~prądem (doświadczenie Ørsteda)
      \begin{center}
        \begin{tikzpicture}
          % magnes
        \end{tikzpicture}
      \end{center}
    \end{definition}

    \begin{definition}
      \textbf{Indukcja pola magnetycznego} to wielkość wektorowa wyrażająca natężenie pola magnetycznego w~danym punkcie przestrzeni.
      \begin{equation}
        B = \frac{F}{qv} \unit{T}
      \end{equation}
    \end{definition}

    \subsection{Pole magnetyczne prostoliniowego przewodnika}
      \begin{equation*}
        B = \frac{\mu_0 I}{2\pi r}
      \end{equation*}
      \begin{symbols}
        \item $\mu_0$ -- stała przenikalności magnetycznej próżni
      \end{symbols}
      \begin{equation*}
        \mu_0 = 4\pi\cdot 10^{-7} \units{\frac{Tm}{A}}{\frac{N}{A^2}}
      \end{equation*}

    \subsection{Pole magnetyczne gęstej zwojnicy}
      \begin{equation*}
        B = \frac{\mu_0 nI}{l}
      \end{equation*}
      \begin{symbols}
        \item $n$ -- ilość nawiniętych zwojów
        \item $l$ -- długość zwojnicy
      \end{symbols}

    \subsection{Pole magnetyczne pętli (pojedynczego zwoju)}
      \begin{equation*}
        B = \frac{\mu_0 I}{2r}
      \end{equation*}

  \section{Siła elektrodynamiczna}
    \begin{definition}
      \textbf{Siła elektrodynamiczna} to siła działająca na~umieszczony w~polu magnetycznym przewodnik, przez który przepływa prąd elektryczny, powodując jego ruch.
      \begin{gather*}
          \vec F = I\Delta\vec l\times\vec B \unit{N}\\
          F = I\Delta l\cdot B\sin\measuredangle(\Delta\vec l, \vec B)
      \end{gather*}
      jeżeli $\Delta\vec l \perp \vec B$, to:
      \begin{equation}
        \boxed{F = BI\Delta l}
      \end{equation}
    \end{definition}

  \section{Siła Lorenza}
    \begin{definition}
      \textbf{Siła Lorenza} to siła działająca na~naładowane cząstki poruszające się w~polu magnetycznym powodująca odchylenie toru ruchu tych cząstek.
    \begin{gather*}
      \vec F_L = q\vec v\times\vec B \unit{N}\\
      F_L = qvB\sin\measuredangle(\vec v, \vec B)
    \end{gather*}
    jeżeli $\vec v\perp\vec B$, to:
    \begin{equation}
      \boxed{F_L = qvB}
    \end{equation}
    \end{definition}

  \section{Cyklotron}
    Cyklotron służy do przyspieszania cząstek obdarzonych ładunkiem. Składa się z dwóch duantów w kształcie puszki.

\end{document}
