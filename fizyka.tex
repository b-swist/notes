\documentclass{report}
\title{Fizyka}
\date{2025-02-12}
\author{Bartosz Świst}

\usepackage[utf8]{inputenc}
\usepackage[T1]{fontenc}
\usepackage[polish]{babel}
\usepackage{amsmath, amssymb, amsthm, tikz}

\DeclareMathOperator{\tg}{tg}
\DeclareMathOperator{\const}{const.}
\newcommand{\unit}[1]{\,\left[\mathrm{#1}\right]}
\newcommand{\tab}{\hspace*{3mm}}
\newcommand{\bi}[1]{\textbf{\textit{#1}}}
\renewcommand{\chaptername}{Rozdział}

\newtheoremstyle{colon}{\topsep}{\topsep}{\itshape}{}{\bfseries}{:}{5pt plus 1pt minus 1pt}{}
\theoremstyle{colon}
\newtheorem*{inner}{\innerheader}
\newcommand{\innerheader}{}
\newenvironment{law}[1]{\renewcommand\innerheader{#1}\begin{inner}}{\end{inner}}
\theoremstyle{definition}
\newtheorem*{definition}{Definicja}

\begin{document}
  \maketitle
  \chapter{Kinematyka}

\section{Wektory}

\subsection{Iloczyn skalarny}
\begin{equation}
  c = \vec a \cdot \vec b = |\vec a| \cdot |\vec b| \cdot \cos\measuredangle(\vec a, \vec b)
\end{equation}

\subsection{Iloczyn wektorowy}
\begin{equation}
  \vec c = \vec a \times \vec b =|\vec a| \cdot |\vec b| \cdot \sin\measuredangle(\vec a, \vec b)
\end{equation}

\section{Opis ruchu}
\begin{gather*}
  v_\text{śr} = \frac s t \unit{\frac m s}\\
  \vec v_\text{śr} = \frac{\Delta \vec x}{\Delta t} \unit{\frac m s}\\
  \vec a = \frac{\Delta \vec v}{\Delta t} \unit{\frac{m}{s^2}}
\end{gather*}

\section{Ruch jednostajny prostoliniowy}
\begin{gather*}
  v = \const\\
  v = \frac s t = \tg \alpha\\
  x(t) = x_0 \pm vt
\end{gather*}

\section{Ruch jednostajnie przyspieszony}
\begin{gather*}
  a = \const\\
  a = \frac{\Delta v}{\Delta t}\\
  v_k = v_0 + at\\
  s = v_0 t + \frac{a t^2}{2}
\end{gather*}
jeżeli $v_0 = 0$:
\begin{gather*}
  v = at\\
  s = \frac{a t^2}{2}
\end{gather*}

\subsection{Stosunek przebytych dróg do odcinków czasu}
\begin{equation*}
  s_1:s_2:s_3:s_4:s_5:\ldots = 1:3:5:7:9:\ldots
\end{equation*}

\section{Ruch jednostajnie opóźniony}
\begin{gather*}
  v_k = v_0 - at\\[0.8em]
  \begin{aligned}
    s &= v_0 t - \frac{a t^2}{2}\\
    &= v_0 t - \frac 1 2 \Delta vt
  \end{aligned}
\end{gather*}
jeżeli $v_k = 0$:
\begin{gather*}
  v_0 = at\\
  s = \frac 1 2 v_0 t
\end{gather*}

\section{Rzut pionowy}

\subsection{Wznoszenie się}
\begin{gather*}
  h = v_0 t - \frac{g t^2}{2}\\
  v = v_0 - gt
\end{gather*}

\subsection{Opadanie}
\begin{gather*}
  h = v_0t + \frac{gt^2}{2}\\
  v = v_0 + gt
\end{gather*}
jeżeli $v_0 = 0$:
\begin{equation*}
  v = gt
\end{equation*}

\section{Rzut poziomy}
\begin{gather*}
  h = \frac{g t^2}{2}\\
  x = v_0 t = v_0 \sqrt{\frac{2h}{g}}\\
  v = \sqrt{v_0^2 + v_y^2} = \sqrt{v_0^2 + (gt)^2}
\end{gather*}

\section{Rzut ukośny}
% \begin{center}
%   \begin{tikzpicture}
%     \draw[gray, very thin] (0, 0) grid (9.8, 3.8);
%     \draw[->, thin] (-0.3, 0) -- (10, 0) node[below] {$x$};
%     \draw[->, thin] (0, -0.3) -- (0, 4) node[left] {$y$};
%     \draw (4, 3) parabola (0, 0);
%     \draw (4, 3) parabola (8, 0);
%
%     \draw[->, thick] (0, 0) -- (0.8, 0) node[midway, below] {$\vec v_{0_x}$};
%     \draw[->, thick] (0, 0) -- (0, 1) node[midway, left] {$\vec v_{0_y}$};
%     \draw[->, thick] (0, 0) -- (0.8, 1) node[midway, above, xshift=-1] {$\vec v_0$};
%     \draw (0.6, 0) arc (0:50:0.6) node[below, xshift=-0.8, yshift=-2.5] {$\alpha$};
%
%     \draw[->, thick] (4, 3) -- (4.8, 3) node[midway, below] {$\vec v_{0_x}$};
%
%     \draw[->, thick] (8, 0) -- (8.8, 0) node[midway, above] {$\vec v_{0_x}$};
%     \draw[->, thick] (8, 0) -- (8, -1) node[midway, left] {$\vec v_y$};
%     \draw[->, thick] (8, 0) -- (8.8, -1) node[midway, right] {$\vec v$};
%
%     \draw[dotted, thick] (4, 0) -- (4, 3) node[midway, right] {$h_{max}$};
%   \end{tikzpicture}
% \end{center}
\begin{align*}
  v_{0_x} &= v_0 \cos \alpha\\
  v_{0_y} &= v_0 \sin \alpha
\end{align*}

\begin{equation}
  y(x) = x \tg \alpha - x^2 \cdot \frac{g}{2 v_0^2 \cos^2 \alpha}
\end{equation}

\begin{gather*}
  t_c = \frac{2 v_0 \sin \alpha}{g}\\
  h_\text{max} = \frac{v_0^2 \sin^2 \alpha}{2g}\\
  z = \frac{v_0^2 \sin 2\alpha}{g}
\end{gather*}
dla $\alpha = 45^\circ$:
\begin{equation*}
  z = \frac{v_0^2}{2g}
\end{equation*}

\section{Ruch jednostajny po okręgu}

\begin{gather*}
  \alpha = \frac Lr \unit{rad}\\
  f = \frac nt \unit{Hz}\\
  \omega = \frac{\Delta \alpha}{\Delta t} \unit{\frac{rad}{s}}
\end{gather*}
dla jednego obrotu:
\begin{gather*}
  f = \frac 1 T\\
  \omega = \frac{2\pi}{T} = 2\pi f
\end{gather*}

\begin{gather*}
  v = \omega r\\
  a_r = \frac{v^2}{r}
\end{gather*}

\begin{gather*}
  \vec v = \vec \omega \times \vec r\\
  v = \omega r \sin\measuredangle(\vec \omega, \vec r)
\end{gather*}
dla $\vec \omega \perp \vec r$:
\begin{equation*}
  v = \omega r
\end{equation*}

\section{Przyspieszenie w ruchu po okręgu}
\begin{gather*}
  \vec a_s = \frac{\Delta \vec v}{\Delta t}\\[0.5em]
  a_w = \sqrt{a_r^2 + a_s^2}
\end{gather*}

% vim:spell:spl=pl

  \chapter{Dynamika}

\section{Zasady dynamiki Newtona}

\subsection{Pierwsza zasada}
\begin{subequations}
  \begin{equation}
    \vec F_w = 0 \Rightarrow \vec v = 0 \lor \vec v = \const
  \end{equation}

  \subsection{Druga zasada}
  \begin{gather}
    \begin{aligned}
      F_w \ne 0 &\Rightarrow a = \const\\
      \vec a = \frac{\vec F_w}{m} &\Rightarrow \vec F_w = m \vec a \unit N
    \end{aligned}
  \end{gather}

  \subsection{Trzecia zasada}
  \begin{equation}
    \begin{aligned}
      \vec F_{AB} &= -\vec F_{BA}\\
      F_{AB} &= F_{BA}
    \end{aligned}
  \end{equation}
\end{subequations}

\section{Ruch na równi pochyłej}
\begin{align*}
  \frac{\vec F_Z}{\vec F_g} = \sin \alpha &\Rightarrow \vec F_Z = \vec F_g \sin \alpha = mg \sin \alpha\\
  \frac{\vec F_N}{\vec F_g} = \cos \alpha &\Rightarrow \vec F_N = \vec F_g \cos \alpha = mg \cos \alpha
\end{align*}

%% TODO: add a section

\section{Pęd ciała}
\begin{equation}
  \vec p = m \vec v \unit{\frac{kg\cdot m}{s}}
\end{equation}

\begin{equation*}
  \Delta p = F \Delta t
\end{equation*}

\subsection{Zasada zachowania pędu}
\begin{equation*}
  \Delta \vec p = 0 \Leftrightarrow \vec p = \const
\end{equation*}

\section{Środek masy}
\begin{equation*}
  x_c = \frac{\sum_{i=1}^n m_i x_i}{\sum_{i=1}^n m_i}
\end{equation*}

\section{Tarcie}
\begin{align*}
  T_s &= \mu_s F_N \unit N\\
  T_k &\leqslant \mu_k F_N \unit N
\end{align*}

\section{Siła dośrodkowa}
\begin{equation*}
  F_{do} = \frac{mv^2}{r} \unit N
\end{equation*}

\section{Siła bezwładności}
\begin{equation*}
  \vec F_b = -m \vec a \unit N
\end{equation*}

% vim:spell:spl=pl

    \chapter{Praca, moc, energia}
    \section{Praca}
      \begin{gather}
        W = \vec F\Delta\vec r \unit{J}\\
        W = F\Delta r \cos\measuredangle(\vec F, \Delta\vec r)
      \end{gather}
      dla $\alpha = 0^\circ$:
      \begin{equation}
        W = F\Delta r = Fs
      \end{equation}
      dla $\alpha = 90^\circ$:
      \begin{equation}
        W = 0
      \end{equation}
    \section{Moc}
      \begin{equation}
        P = \frac{W}{t} \unit{W}
      \end{equation}
      dla $v = const.$:
      \begin{equation}
        P = Fs
      \end{equation}
    \section{Energia mechaniczna}
      \subsection{Energia kinetyczna}
        \begin{align}
          &E_k = \frac{mv^2}{2} \unit{J}\\
          &\Delta E_k = W
        \end{align}
      \subsection{Energia potencjalna}
        \begin{align}
          &E_p = mgh \unit{J}\\
          &\Delta E_p = W
        \end{align}
      \subsection{Zasada zachowania energii}
        \begin{align}
          &E_c = E_k + E_p\\
          &E_c = const. \Rightarrow \Delta E_c = 0\\
          &\Delta E_c = \Delta E_p + \Delta E_k
        \end{align}
    \section{Sprawność}
      \begin{equation}
        \eta = \frac{E_{u\dot{z}yt.}}{E_{pob.}}\: (\cdot 100\%) = \frac{W_{u\dot{z}yt.}}{E_{pob.}}\: (\cdot 100\%)
      \end{equation}
      \begin{equation}
        \eta_{u\dot{z}yt.} = \prod_{i=1}^n \eta_i
      \end{equation}

    \chapter{Hydrostatyka}
    \section{Ciśnienie i parcie}
      \subsection{Ciśnienie}
        \begin{equation}
          p = \frac{F_N}{S} \unit{Pa}
        \end{equation}
        dla $F_N = mg$:
        \begin{equation}
          p = \frac{mg}{S}
        \end{equation}
      \subsection{Parcie}
        \begin{equation}
          P = pS \unit{N}
        \end{equation}
      \subsection{Ciśnienie hydrostatyczne}
        \begin{equation}
          p_h = \frac{P}{S} = \varrho_cgh \unit{Pa}
        \end{equation}
      \subsection{Paradoks hydrostatyczny}
    \section{Prawo Pascala}
      \begin{equation}
        p_1 = p_2 \Rightarrow \frac{F_1}{S_1} = \frac{F_2}{S_2}
      \end{equation}
      \subsection{Naczynia połączone}
        \begin{equation}
          p_1 = p_2 \Rightarrow \varrho_1h_1 = \varrho_2h_2
        \end{equation}
    \section{Prawo Archimedesa}
      \begin{equation}
        F_W = P_2 - P_1 = \varrho_cgV_z \unit{N}
      \end{equation}
      \subsection{Warunki wypływania}
        \begin{align*}
          F_W > F_g &\Rightarrow \text{ciało wypływa}\\
          F_W = F_g &\Rightarrow \text{ciało pływa}\\
          F_W < F_g &\Rightarrow \text{ciało tonie}
        \end{align*}

  \chapter{Bryła sztywna}

\section{Ruch obrotowy}

\subsection{Prędkość kątowa}
\begin{equation}
  \omega = \frac{\Delta \alpha}{\Delta t} \units{\frac{rad}{s}}{\frac 1 s}
\end{equation}

\subsection{przyspieszenie kątowe}
\begin{equation}
  \varepsilon = \frac{\Delta \omega}{\Delta t} \units{\frac{rad}{s^2}}{\frac{1}{s^2}}
\end{equation}

\subsection{Prędkość liniowa (styczna)}
\begin{gather*}
  \vec v = \vec \omega \times \vec r \unit{\frac m s}\\
  v = \omega r \sin\measuredangle(\vec \omega, \vec r)
\end{gather*}
dla $\vec \omega \perp \vec r$:
\begin{equation*}
  v = \omega r
\end{equation*}

\subsection{przyspieszenie liniowe}
\begin{equation}
  a_r = \varepsilon r \unit{\frac{m}{s^2}}
\end{equation}

\section{Równania obrotu}

\begin{gather*}
  \omega = \omega_0 \pm \varepsilon t\\
  \alpha = \omega_0 t \pm \frac{\varepsilon t^2}{2}\\
\end{gather*}
dla $\omega_0 = 0$:
\begin{equation*}
  \alpha = \frac 1 2 \omega t
\end{equation*}

\section{Moment bezwładności}
\begin{equation}
  I = \sum_{i=1}^n m_i r_i^2 \unit{kg \cdot m^2}
\end{equation}

\subsection{Momenty bezwładności wybranych brył}
kula: $I_0 = \frac 2 5 mr^2$\\
walec: $I_0 = \frac 1 2 mr^2$\\
pręt: $I_0 = \frac{1}{12} ml^2$\\
rura grubościenna: $I_0 = \frac 1 2 m(r_1^2 + r_2^2)$

\subsection{Twierdzenie Steinera}
\begin{equation*}
  I = I_0 + mx^2
\end{equation*}

\section{Energia kinetyczna}
\begin{gather*}
  E_{k_o} = \sum_{i=1}^n \frac{m_i v_i}{2} \unit J\\
  E_{k_o} = \frac{I \omega^2}{2}
\end{gather*}

\section{Moment siły}
\begin{equation}
  \begin{gathered}
    \vec M = \vec r \times \vec F \unit{N \cdot m}\\
    M = rF \sin\measuredangle(\vec r, \vec F)
  \end{gathered}
\end{equation}
dla $\vec r \perp \vec F$:
\begin{equation*}
  M = rF
\end{equation*}
dla $\vec r \parallel \vec F$:
\begin{equation*}
  M = 0
\end{equation*}

\subsection{Wypadkowy moment siły}
\begin{equation*}
  M_w = \sum_{i=1}^n M_i\\
  M_w = \varepsilon I
\end{equation*}

\subsection{Równowaga bryły sztywnej}
\begin{gather*}
  F_w = 0\\
  M_w = 0
\end{gather*}

\section{Moment pędu}
\begin{subequations}
  \begin{equation}
    \begin{aligned}
      \vec L &= \vec r \times \vec p \unit{\frac{kg \cdot m^2}{s}}\\[1em]
      L &= rp \sin\measuredangle(\vec r, \vec p)\\
      &= mrv \sin\measuredangle(\vec r, \vec v)
    \end{aligned}
  \end{equation}
  dla $\vec p \perp \vec r$:
  \begin{equation*}
    L = rp = mrv
  \end{equation*}

  \begin{equation}
    L = \sum_{i=1}^n m_i r_i v_i \sin\measuredangle(\vec r,\vec v)
  \end{equation}
  dla $\vec r \perp \vec v$:
  \begin{equation*}
    L = \omega I
  \end{equation*}
\end{subequations}

% vim:spell:spl=pl

    \chapter{Grawitacja}
    \section{Prawa Keplera}
      \subsection{Pierwsze prawo}
      \subsection{Drugie prawo}
        \begin{align}
          s_1 &= s_2\\
          L_1 = L_2 &\Rightarrow r_1v_1 = r_2v_2
        \end{align}
      \subsection{Trzecie prawo}
        \begin{equation}
          \frac{T^2}{r^3} = const.
        \end{equation}
    \section{Prawo powszechnego ciążenia}
      \begin{equation}
        F = G\frac{m_1m_2}{r^2} \unit{N}
      \end{equation}
      gdzie:
      \begin{equation}
        G = 6,67\cdot 10^{-11} \unit{\frac{N\cdot m^2}{kg^2}}
      \end{equation}
      \begin{gather}
        F = \frac{4}{3}\pi RGdm\\
        F \sim dR
      \end{gather}
    \section{Natężenie pola grawitacyjnego}
      \begin{equation}
        \vec\gamma = \frac{\vec F_g}{m} \unit{\frac{N}{kg},\;\frac{m}{s^2}}
      \end{equation}
      dla pola centralnego:
      \begin{equation}
        \gamma = \frac{GM}{r^2}
      \end{equation}
    \section{Praca w polu grawitacyjnym}
      \begin{gather}
        W = mgh\\
        \Delta E_p = W
      \end{gather}
      \begin{gather}
        W_{Z_{(A\rightarrow B)}} = GMm\left(\frac{1}{r_A} - \frac{1}{r_B}\right)\\
        W_{g_{(A\rightarrow B)}} = -W_{Z_{(A\rightarrow B)}}
      \end{gather}
    \section{Energia w polu grawitacyjnym}
      \begin{equation}
        E_p = -\frac{GMm}{r}
      \end{equation}
    \section{Potencjał pola grawitacyjnego}
      \begin{gather}
        V = \frac{E_p}{m} \unit{\frac{J}{kg}}\\
        \Delta V = \frac{\Delta E_p}{m}
      \end{gather}
    \section{Prędkości kosmiczne}
      \subsection{Pierwsza prędkość kosmiczna}
        \begin{equation}
          v_{{}_\mathrm{I}} = \sqrt{\frac{GM}{r}}
        \end{equation}
      \subsection{Satelita geostacjonarny}
        \begin{equation}
          r = \sqrt[3]{\frac{GMT^2}{4\pi^4}}
        \end{equation}
      \subsection{Druga prędkość kosmiczna}
        \begin{equation}
          v_{{}_\mathrm{II}} = \sqrt{\frac{2GM}{r}} = v_{{}_\mathrm{I}}\sqrt{2}
        \end{equation}

  \chapter{Ruch drgający}
\begin{gather}
  F_z = kx\\
  F_s = -kx
\end{gather}

\begin{equation*}
  k = \left|\frac{F_s}{x}\right| \unit{\frac N m}
\end{equation*}

\section{Ruch harmoniczny}
\begin{gather*}
  x = r \sin \alpha\\
  T = 2\pi \sqrt{\frac m k} \unit s
\end{gather*}

\subsection{Równania ruchu harmonicznego}
\begin{subequations}
  \begin{align}
    x(t) &= A \sin(\omega t + \varphi_0)\\
    v(t) &= \omega A\ cos(\omega t + \varphi_0)\\
    a(t) &= -\omega^2 A \sin(\omega t + \varphi_0)
  \end{align}
\end{subequations}

\begin{align*}
  x_\text{max} &= A\ \text{dla}\ \sin90^\circ = 1\\
  v_\text{max} &= \omega A\ \text{dla}\ \cos0^\circ = 1\\
  a_\text{max} &= -\omega^2 A\ \text{dla}\ \sin90^\circ = 1\\
\end{align*}

\subsection{Łączenie sprężyn}
\begin{align*}
  F &= \const & x &= \const\\
  x &= \sum_{i=1}^n x_i & F_c &= \sum_{i=1}^n F_i\\
  \frac 1 k &= \sum_{i=1}^n \frac{1}{k_i} & k &= \sum_{i=1}^n k_i
\end{align*}

\section{Energia w ruchu harmonicznym}
\begin{equation*}
  W = \frac 1 2 Fx \Rightarrow E_{p_s} = \frac 1 2 kx^2
\end{equation*}

\begin{align*}
  E_c &= E_{p_s} + E_k = \frac 1 2 kA^2\\
  E_k &= \frac 1 2 k(A^2 - x^2)
\end{align*}

\section{Wahadło matematyczne}
\begin{equation*}
  F = F_g \sin \alpha
\end{equation*}
dla małych kątów $\sin \alpha \approx \alpha$:
\begin{gather*}
  F = mg\alpha\\
  T = 2\pi\sqrt{\frac lg}
\end{gather*}

  \chapter{Termodynamika}

\section{Zasady dynamiki}

\begin{law}{Zerowa zasada dynamiki}
  Jeżeli ciało A jest w równowadze termodynamicznej z ciałem B, a ciało B jest w równowadze
  termodynamicznej z ciałem C, to ciała A i C są również w równowadze termodynamicznej.
\end{law}

\begin{law}{Pierwsza zasada termodynamiki}
  Zmiana energii wewnętrznej ciała jest równa sumie ilości ciepła wymienionego z otoczeniem i pracy
  wykonanej nad ciałem przez siłę zewnętrzną.
\end{law}

\begin{equation*}
  p = \frac 2 3 \cdot \frac{NE_{k_\text{śr}}}{V} \unit{Pa}
\end{equation*}
gdzie $N$ - liczba cząstek gazu
\begin{equation*}
  E_{k_\text{śr}} = \frac 1 2 mv_\text{śr}^2 \unit J
\end{equation*}

\section{Równanie gazu doskonałego}
\begin{equation*}
  \frac{p_1 V_1}{T_1} = \frac{p_2 V_2}{T_2} \Rightarrow \frac{pV}{T} = \const
\end{equation*}

\subsection{Równanie Clapeyrona}
\begin{equation}
  pV = nRT = NkT
\end{equation}
gdzie:
\begin{gather*}
  R =  8,31 \unit{\frac{J}{mol \cdot K}}\\
  k = \frac{R}{N_A} = 1,38 \cdot 10^{-23} \unit{\frac J K}
\end{gather*}

\section{Przemiany gazu doskonałego}

\subsection{Przemiana izotermiczna}
\begin{gather*}
  T = \const\\
  \frac{p_1 V_1}{T} = \frac{p_2 V_2}{T} \Rightarrow p_1 V_1 = p_2 V_2\\
  pV = \const \Rightarrow p = \frac{\const}{V}\\ \text{(Prawo Boyle'a)}
\end{gather*}

\subsection{Przemiana izochoryczna}
\begin{gather*}
  V = \const\\
  \frac{p_1 V}{T_1} = \frac{p_2 V}{T_2} \Rightarrow \frac{p_1}{T_1} = \frac{p_2}{T_2}\\
  \frac p T = \const \Rightarrow p = T \cdot \const\\ \text{(Prawo Charles'a)}
\end{gather*}

\subsection{Przemiana izobaryczna}
\begin{gather*}
  p = \const\\
  \frac{p V_1}{T_1} = \frac{p V_2}{T_2} \Rightarrow \frac{V_1}{T_1} = \frac{V_2}{T_2}\\
  \frac V T = \const \Rightarrow V = T \cdot \const\\ \text{(Prawo Gay-Lussaca)}
\end{gather*}

\section{Pierwsza zasada termodynamiki}
\begin{equation}
  \Delta U = Q + W_z \unit J
\end{equation}
dla $Q > 0$ ciepło zostało pobrane\\
dla $Q < 0$ ciepło zostało oddane

\begin{gather*}
  W_z = F_z \Delta x \cos\measuredangle(\vec F_z, \Delta \vec x)\\
  W_z = -W_\text{gazu}
\end{gather*}
dla $W_z > 0$:
\begin{equation*}
  W_z = F_z \Delta x
\end{equation*}
dla $W_z < 0$:
\begin{equation*}
  W_z = -F_z \Delta x
\end{equation*}
\begin{equation*}
  |W| = p|\Delta V|
\end{equation*}

\section{Energia wewnętrzna gazu doskonałego}
\begin{gather*}
  U = N \cdot\frac i 2 kT\\
  \Delta U = N \cdot \frac i 2 k \Delta T
\end{gather*}
gdzie $i$ -- stopnie swobody cząstek

\subsection{Przemiana izotermiczna}
\begin{gather*}
  T = \const \Leftrightarrow U = \const\\
  \Delta U = 0 \Rightarrow Q + W = 0
\end{gather*}

\subsection{Przemiana izochoryczna}
\begin{gather*}
  V = \const \Rightarrow \Delta V = 0\\
  W = 0 \Rightarrow \Delta U = Q
\end{gather*}

\subsection{Przemiana adiabatyczna}
\begin{gather*}
  Q = 0 \Rightarrow \Delta U = W\\
  p V^\kappa = \const
\end{gather*}
gdzie:
\begin{equation*}
  \kappa = \frac{C_p}{C_V}
\end{equation*}

\section{Ciepło właściwe i molowe }

\subsection{Ciepło właściwe}
\begin{gather*}
  C_w = \frac{Q}{m \Delta T} \unit{\frac{J}{kg \cdot K}}\\
  Q = m C_w \Delta T
\end{gather*}

\subsection{Ciepło molowe}
\begin{equation*}
  C = \frac{Q}{n \Delta T} \unit{\frac{J}{mol \cdot K}}
\end{equation*}
ciepło molowe przy stałym ciśnieniu: $C_p$\\
ciepło molowe przy stałej objętości: $C_V$

\begin{gather*}
  Q_p = Q_V + p \Delta V\\
  C_p = C_V + R
\end{gather*}

\section{Energia wewnętrzna jako funkcja stanu}
\begin{equation}
  \Delta U = Q_V = n C_V \Delta T
\end{equation}

\section{Silnik cieplny}
\begin{equation*}
  \eta = \frac{|Q_1|-|Q_2|}{Q_1} = \frac{T_1 - T_2}{T_1}
\end{equation*}

\section{Przejścia fazowe}
\begin{equation*}
  Q = m C_w \Delta T
\end{equation*}
woda - lód: $T_T = T_K = 0^\circ C$\\
woda - para wodna: $T_W = T_S = 100^\circ C$

\begin{equation*}
  Q = mL
\end{equation*}

\begin{equation*}
  Q = mR
\end{equation*}

\section{Rozszerzalność temperaturowa ciał}

\subsection{Rozszerzalność objętościowa}
\begin{equation*}
  \Delta V = V_0 \alpha \Delta T
\end{equation*}

\subsection{Rozszerzalność liniowa}
\begin{equation*}
  \Delta l = l_0 \lambda \Delta T
\end{equation*}

% vim:spell:spl=pl

  \chapter{Elektrostatyka}
  \section{Ładunek elektryczny}
    \begin{gather*}
      |e| = 1,6\cdot 10^{-19} \unit{C}\\
      q = ne \unit{C}
    \end{gather*}
    \begin{law}{Zasada zachowania ładunku}
      W izolowanym układzie całkowity ładunek elektryczny nie ulega zmianie.
      \begin{equation}
        \sum_{i=1}^n q_i = \const
      \end{equation}
    \end{law}

  \section{Prawo Coulomba}
    \begin{law}{Prawo Coulomba}
      Siła wzajemnego oddziaływania dwóch ładunków jest wprost proporcjonalna do iloczynu tych ładunków i odwrotnie proporcjonalna do kwadratu odległości między nimi.
      \begin{equation}\label{coulomb}
        \boxed{F = k\frac{Qq}{r^2} \unit{N}}
      \end{equation}
      \begin{symbols}
        \item $k$ -- współczynnik proporcjonalności (stała eletrostatyczna)
        \item $\varepsilon_0$ -- stała przenikalności elektrycznej próżni
      \end{symbols}
      \begin{gather*}
        k = \frac{1}{4\pi\varepsilon_0} \approx 8,99\cdot 10^9 \unit{\frac{N\cdot m^2}{C^2}}\\[0.5em]
        \varepsilon_0 = 8,85\cdot 10^{-12} \unit{\frac{C^2}{N\cdot m^2}}
      \end{gather*}
    \end{law}

  \section{Natężenie pola elektrostatycznego}
    \begin{definition}
      \bi{Natężenie pola elektrostatycznego} to stosunek siły elektrostatycznej działającej na \emph{dodatni} ładunek próbny $q$ w danym punkcie pola do~wartości tego ładunku.
      \begin{subequations}
        \begin{equation}\label{natężenie es}
          \vec E = \frac{\vec F}{q} \unit{\frac{N}{C}}\\[0.5em]
        \end{equation}
        Wstawiając równanie \ref{coulomb} do równania \ref{natężenie es} otrzymujemy:
        \begin{equation}
          \boxed{E = \frac{kQ}{r^2}}
        \end{equation}
      \end{subequations}
    \end{definition}

  \section{Rozmieszczenie ładunku na przewodniku}
    Po namagnesowaniu ciała cały dostarczony ładunek rozmieszcza się na jego zewnętrznej powierzchni. Pole elektromagnetyczne wewnątrz zanika.

    Rozmieszczenie ładunku na powierzchni zależy od jego kształu. Rozkład ładunku opisuje gęstość powierzchniowa ładunku --- iloraz ładunku i pole tej powierzchni.
    \begin{equation*}
      \sigma = \frac{Q}{S} \unit{\frac{C}{m^2}}
    \end{equation*}

  \section{Praca w polu centralnym}
    \begin{equation}\label{praca es}
      W_{A\rightarrow B} = -kQq\left(\frac{1}{r_A} - \frac{1}{r_B}\right) \unit{J}
    \end{equation}

  \section{Energia w polu centralnym}
  \begin{equation}\label{energia es}
      E_p = \frac{kQq}{r} \unit{J}
    \end{equation}

  \section{Potencjał w polu centralnym}
    \begin{definition}
      Potencjał pola elektrostatycznego to stosunek energii potencjalnej punktowego ciała do wartości ładunku próbnego umieszczonego w tym polu.
      \begin{subequations}
        \begin{equation}\label{potencjał es}
          V = \frac{E_p}{q} \unit{V}\\
        \end{equation}
        Podstawiajac równanie \ref{energia es} do równania \ref{potencjał es} otrzymujemy:
        \begin{equation}
          V = \frac{kQ}{r^2}
        \end{equation}
      \end{subequations}
    \end{definition}
    \begin{equation*}
        W = \Delta E_p = q\Delta V = qU
    \end{equation*}

  \section{Pojemność elektryczna przewodnika}
    \begin{definition}
      \bi{Pojemność elektryczna przewodnika} to stosunek ilości ładunku zgromadzonego na przewodniku do uzyskanego potencjału.
      \begin{equation}
        C = \frac{Q}{V} \unit{F}
      \end{equation}
    \end{definition}

  \section{Kondensator}
    \begin{definition}
      \bi{Kondensator} to element elektroniczny służący do gromadzenia ładunku elektrycznego.
      \begin{equation*}
        C = \frac{Q}{U}
      \end{equation*}
    \end{definition}

    \subsection{Łączenie kondensatorów}
      \begin{align*}
        U &= \const & Q &= \const\\
        Q &= \sum_{i=1}^n Q_i & U &= \sum_{i=1}^n U_i\\
        C_z &= \sum_{i=1}^n C_i & \frac{1}{C_z} &= \sum_{i=1}^n \frac{1}{C_i}
      \end{align*}

    \subsection{Kondensator płaski}
      \begin{definition}
        Kondensator płaski składa się z dwóch równoległych, metalowych okładek, między którymi znajduje się dielektryk.\\[1.5em]
        Bez dielektryka:
        \begin{equation*}
          C = \frac{\varepsilon_0S}{d} 
        \end{equation*}
        Uwzględniając dielektryk:
        \begin{equation*}
          C = \frac{\varepsilon_0\varepsilon_rS}{d}
        \end{equation*}
        Natężenie pola elektrycznego kondensatora:
        \begin{equation*}
          E = \frac{U}{d}
        \end{equation*}
        \begin{symbols}
          \item $d$ -- odległość między okładkami
          \item $S$ -- pole powierzchni okładek
          \item $\varepsilon_r$ -- stała przenikalność dielektryka
        \end{symbols}
      \end{definition}

    \subsection{Energia naładowaniego kondensatora}
      \begin{definition}
        \bi{energia naładowaniego kondensatora} to praca potrzebna do jego naładowania.
        \begin{gather*}
          E = \frac{1}{2}QU = \frac{1}{2}CU^2 = \frac{Q^2}{2C} \units{eV}{J}\\[0.5em]
          (1\mathrm{eV} = 1,6\cdot 10^{-19}\mathrm{J})
        \end{gather*}
      \end{definition}
  \section{Ruch ładunków w polu elektrostatycznym}
    Na naładowane cząstki w polu elektrostatycznym centralnym działa siła elektrostatyczna, przez którą cząstka zaczyna przyspieszać.
    \begin{equation*}
      F = qE
    \end{equation*}
    jeżeli $F = F_w$, to:
    \begin{equation*}
      a = \frac{qE}{m} = \frac{qU}{md} \unit{\frac{m}{s^2}}
    \end{equation*}

  \chapter{Prąd elektryczny}

\begin{definition}
  \bi{Prąd elektryczny} to uporządkowany ruch ładunków elektrycznych, których nośnikami w metalach
  są \textbf{elektrony}.
\end{definition}

\begin{definition}
  \bi{Napięcie elektryczne} to różnica potencjałów między dwoma punktami obwodu elektrycznego
  powodująca przepływ ładunków.
  \begin{equation}
    U = \Delta V \unit V
  \end{equation}
\end{definition}

\begin{definition}
  \bi{Natężenie prądu elektrycznego} to stosunek ilości ładunków przepływających przez przekrój
  poprzeczny przewodu do czasu, w którym ten ładunek przepłynął.
  \begin{equation}
    I = \frac{\Delta Q}{\Delta t} \unit A
  \end{equation}
\end{definition}

\section{Prawo Ohma}
\begin{law}{Prawo Ohma}
  Natężenie prądu płynącego przez przewodnik jest wprost proporcjonalne do napięcia pomiędzy
  końcami tego przewodnika.
  \begin{equation}
    \boxed{R = \frac U I \unit \Omega}
  \end{equation}
\end{law}

\subsection{Łączenie rezystorów}
\begin{align*}
  I &= \const & U &= \const\\
  U &= \sum_{i=1}^n U_i & I &= \sum_{i=1}^n I_i\\
  R_z &= \sum_{i=1}^n R_i & \frac{1}{R_z} &= \sum_{i=1}^n \frac{1}{R_i}
\end{align*}

\subsection{Opór elektryczny przewodnika}
\begin{definition}
  \bi{Opór elektryczny} to zdolność ciała do przeciwstawiania się przepływowi prądu elektrycznego.
  \begin{equation*}
    R = \frac{\varrho l}{S} \unit \Omega
  \end{equation*}
  \begin{symbols}
    \item $\varrho$ -- opór właściwy materiału przewodnika
    \item $l$ -- długość przewodu
    \item $S$ -- pole przekroju przewodnika
  \end{symbols}
\end{definition}

\section{Praca i moc prądu elektrycznego}
Przepływ prądu elektrycznego wiąże się z wykonywaniem przez elektrony pracy.
\begin{equation}\label{praca el}
  \boxed{W = UIt = \frac{U^2}{R} t = I^2 Rt}
\end{equation}

\subsection{Emisja ciepła (ciepło Joule'a)}
\begin{equation*}
  Q = W = I^2 Rt
\end{equation*}

\subsection{Energia elektryczna}
\begin{gather*}
  E_{el} = W = UIt \unit{kWh}\\
  (1\mathrm{kWh} = 3,6\mathrm{MJ})
\end{gather*}

\subsection{Moc prądu elektrycznego}
\begin{definition}
  \bi{Moc prądu elektrycznego} to stosunek pracy wykonanej przez przepływające elektrony do czasu,
  w którym tą pracę wykonywały.
  \begin{subequations}
    \begin{equation}\label{moc el}
      P = \frac{W}{t} \unit W
    \end{equation}
    Wstawiając równanie \ref{praca el} do równania \ref{moc el} otrzymujemy:
    \begin{equation}
      P = UI = I^2R = \frac{U^2}{R}
    \end{equation}
  \end{subequations}
\end{definition}

\section{Ogniwo galwaniczne}
\begin{definition}
  \bi{Ogniwo galwaniczne} to elektrolit kwasu, soli lub zasady, w którym zanurzono dwie elektrody
  wykonane np. z miedzi albo cynku. Na skutek dysocjacji elektrolitycznej między biegunami ogniwa
  powstaje różnica potencjałów, którą nazywamy \textbf{siłą elektromotoryczną} (SEM).
\end{definition}

\subsection{Prawo Ohma dla obwodu}
\begin{gather*}
  \mathcal E = U + U_w = U + IR_w \unit V\\
  I = \frac{\mathcal E}{R_z} = \frac{\mathcal E}{R + R_w} \unit A
\end{gather*}
\begin{symbols}
  \item $\mathcal E$ -- siła elektromotoryczna
  \item $R_w$ -- opór wewnętrzny ogniwa
\end{symbols}

\section{Prawa Kirchhoffa}
\begin{law}{Pierwsze prawo Kirchhoffa}
  Suma natężeń wpływających do węzła obwodu elektrycznego jest równa sumie natężeń wypływających z węzła.
  \begin{equation}
    \sum_{i=1}^n I_i = 0
  \end{equation}
\end{law}

\begin{law}{Drugie Prawo Kirchhoffa}
  Suma sił elektromotorycznych i spadków napięć w~obwodzie zamkniętym (oczku) jest równa zero.
  \begin{equation}
    \sum_{i=1}^n \mathcal E_i + \sum_{i=1}^n I_i R_i = 0
  \end{equation}
\end{law}

\section{Przewodnictwo ciał stałych}
Wyróżnia się trzy grupy ciał stałych ze względu na właściwości elektryczne:
\begin{itemize}
  \item przewodniki
  \item izolatory
  \item półprzewodniki
\end{itemize}

\textbf{Przewodnikami} są przede wszystkim metale takie jak miedź i żelazo. Dobre przewodzą prąd,
bo posiadają wolne elektrony. Wraz ze wzrostem temperatury opór elektryczny przewodników wzrasta
wskutek drgań sieci krystalicznej, w~której poruszają się elektrony.

\textbf{Izolatory} nie przewodzą prądu elektrycznego lub robią to bardzo słabo z~powodu braku
wolnych elektronów.

\textbf{Półprzewodniki} to materiały które mogą wykazywać właściwości zarówno izolatorów, jak i
przewodników. Wraz ze wzrostem temperatury ich opór elektryczny maleje, gdyż część elektronów
przeskakuje z poziomu podstawowego do poziomu przewodnictwa, stając się nośnikami prądu
elektrycznego. Poprzez domieszkowanie półprzewodnika pierwiastkami z grupy 13 bądź 15 układu
okresowego uzyskuje się odpowiednio \emph{półprzewodnik dziurowy} (typ „p”) oraz \emph{półprzewodnik
elektronowy} (typ „n”).

\section{Dioda półprzewodnikowa}
Dioda półprzewodnikowa jest złożona z dwóch złączonych półprzewodników --- jeden typu „p”, a drugi
typu „n” --- tworzących złączę p-n/n-p. Dioda półprzewodnikowa przepuszcza prąd tylko w jednym kierunku.

    \chapter{Magnetyzm}
    \section{Indukcja magnetyczna}
      \subsection{Pole magnetyczne prostoliniowego przewodnika}
        \begin{equation}
          B = \frac{\mu_0 I}{2\pi r} \unit{T}
        \end{equation}
        gdzie:
        \begin{equation}
          \mu_0 = 4\pi\cdot 10^{-7} \unit{\frac{Tm}{A},\;\frac{N}{A^2}}
        \end{equation}
      \subsection{Pole magnetyczne gęstej zwojnicy}
        \begin{equation}
          B = \frac{\mu_0 nI}{l} \unit{T}
        \end{equation}
      \subsection{Pole magnetyczne pętli}
        \begin{equation}
          B = \frac{\mu_0 I}{2r} \unit{T}
        \end{equation}
    \section{Siła elektrodynamiczna}
      \begin{equation}
        F = BI\Delta l\sin\measuredangle(\Delta\vec l, \vec B) \unit{N}
      \end{equation}
      dla $\Delta\vec l \perp \vec B$:
      \begin{equation}
        F = BI\Delta l
      \end{equation}
  \section{Siła Lorenza}
    \begin{gather}
      \vec F_L = q\vec v\times\vec B \unit{N}\\
      F_L = qvB\sin\measuredangle(\vec v, \vec B)
    \end{gather}
    dla $\vec v\perp\vec B$:
    \begin{equation}
      F_L = qvB
    \end{equation}

\end{document}
