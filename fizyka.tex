\documentclass{article}

\title{Fizyka}
\date{2025-02-12}
\author{Bartosz Świst}

\usepackage[utf8]{inputenc}
\usepackage[T1]{fontenc}

\usepackage{amsmath, amssymb}

\numberwithin{equation}{section}

\newcommand{\unit}[1]{\, \left[\mathrm{#1}\right]}

\begin{document}
  \maketitle
  \newpage

  \section{Kinematyka}

  \newpage
  \section{Dynamika}
    \subsection{Zasady dynamiki Newtona}
      \subsubsection{Pierwsza zasada}
        \begin{equation}
          \vec F_w = 0 \Rightarrow \vec v = 0 \lor \vec v = const.
        \end{equation}
      \subsubsection{Druga zasada}
        \begin{align}
          \vec F_w \ne 0 &\Rightarrow a = const.\\
          \vec a = \frac{\vec F_w}{m} &\Rightarrow \vec F_w = m\vec a \unit{N}
        \end{align}
      \subsubsection{Trzecia zasada}
        \begin{align}
          \vec F_{AB} &= -\vec F_{BA}\\
          F_{AB} &= F_{BA}
        \end{align}
    \subsection{Ruch na równi pochyłej}
      \begin{align}
        \frac{\vec F_Z}{\vec F_g} = \sin\alpha &\Rightarrow \vec F_Z = \vec F_g\sin\alpha = mg\sin\alpha\\
        \frac{\vec F_N}{\vec F_g} = \cos\alpha &\Rightarrow \vec F_N = \vec F_g\cos\alpha = mg\cos\alpha
      \end{align}
    \subsection{tbd}
    \subsection{Pęd ciała}
      \begin{equation}
        \vec p = m\vec v \unit{\frac{kg\cdot m}{s}}
      \end{equation}
      \begin{equation}
        \Delta p = F\Delta t
      \end{equation}
      \subsubsection{Zasada zachowania pędu}
        \begin{equation}
          \Delta \vec p = 0 \Leftrightarrow \vec p = const.
        \end{equation}
      \subsection{Środek masy}
        \begin{equation}
          x_c = \frac{m_1x_1+m_2x_2+\dots+m_nx_n}{m_1+m_2+\dots+m_n} =
          \frac{\sum\limits_{i=1}^n m_ix_i}{\sum\limits_{i=1}^n m_i}
        \end{equation}
      \subsection{Tarcie}
        \begin{align}
          T_s &= \mu_sF_N \unit{N}\\
          T_k &\leqslant \mu_kF_N \unit{N}
        \end{align}
      \subsection{Siła dośrodkowa}
        \begin{equation}
          F_{do} = \frac{mv^2}{r} \unit{N}
        \end{equation}
        \subsection{Siła bezwładności}
        \begin{equation}
          \vec F_b = -m\vec a \unit{N}
        \end{equation}

  \newpage
  \section{Praca, moc, energia}
    \subsection{Praca}
      \begin{align}
        W &= \vec F\Delta\vec r \unit{J}\\
        W &= F\Delta r \cos\measuredangle\left(\vec F, \Delta\vec r\right)
      \end{align}
      dla $\alpha = 0^\circ$:
      \begin{equation}
        W = F\Delta r = Fs
      \end{equation}
      dla $\alpha = 90^\circ$:
      \begin{equation}
        W = 0
      \end{equation}
    \subsection{Moc}
      \begin{equation}
        P = \frac{W}{t} \unit{W}
      \end{equation}
      dla $v = const.$:
      \begin{equation}
        P = Fs
      \end{equation}
    \subsection{Energia mechaniczna}
      \subsubsection{Energia kinetyczna}
        \begin{align}
          &E_k = \frac{mv^2}{2} \unit{J}\\
          &\Delta E_k = W
        \end{align}
      \subsubsection{Energia potencjalna}
        \begin{align}
          &E_p = mgh \unit{J}\\
          &\Delta E_p = W
        \end{align}
      \subsubsection{Zasada zachowania energii}
        \begin{align}
          &E_c = E_k + E_p\\
          &E_c = const. \Rightarrow \Delta E_c = 0\\
          &\Delta E_c = \Delta E_p + \Delta E_k
        \end{align}
    \subsection{Sprawność}
      \begin{equation}
        \eta = \frac{E_{u\dot{z}yt.}}{E_{pob.}}\: (\cdot 100\%) = \frac{W_{u\dot{z}yt.}}{E_{pob.}}\: (\cdot 100\%)
      \end{equation}
      \begin{equation}
        \eta_{u\dot{z}yt.} = \prod_{i=1}^n \eta_i
      \end{equation}

  \newpage
  \section{Hydrostatyka}
    \subsection{Ciśnienie i parcie}
      \subsubsection{Ciśnienie}
        \begin{equation}
          p = \frac{F_N}{S} \unit{Pa}
        \end{equation}
        dla $F_N = mg$:
        \begin{equation}
          p = \frac{mg}{S}
        \end{equation}
      \subsubsection{Parcie}
        \begin{equation}
          P = pS \unit{N}
        \end{equation}
      \subsubsection{Ciśnienie hydrostatyczne}
        \begin{equation}
          p_h = \frac{P}{S} = \varrho_cgh \unit{Pa}
        \end{equation}
      \subsubsection{Paradoks hydrostatyczny}
    \subsection{Prawo Pascala}
      \begin{equation}
        p_1 = p_2 \Rightarrow \frac{F_1}{S_1} = \frac{F_2}{S_2}
      \end{equation}
      \subsubsection{Naczynia połączone}
        \begin{equation}
          p_1 = p_2 \Rightarrow \varrho_1h_1 = \varrho_2h_2
        \end{equation}
    \subsection{Prawo Archimedesa}
      \begin{equation}
        F_W = P_2 - P_1 = \varrho_cgV_z \unit{N}
      \end{equation}
      \subsubsection{Warunki wypływania}
        \begin{align*}
          F_W > F_g &\Rightarrow \text{ciało wypływa}\\
          F_W = F_g &\Rightarrow \text{ciało pływa}\\
          F_W < F_g &\Rightarrow \text{ciało tonie}
        \end{align*}

  \newpage
  \section{Bryła sztywna}
    \subsection{Ruch obrotowy}
      \subsubsection{Prędkość kątowa}
        \begin{equation}
          \omega = \frac{\Delta\alpha}{\Delta t} \unit{\frac{rad}{s},\;\frac{1}{s}}
        \end{equation}
      \subsubsection{Przyspieszenie kątowe}
        \begin{equation}
          \varepsilon = \frac{\Delta\omega}{\Delta t} \unit{\frac{rad}{s^2},\;\frac{1}{s^2}}
        \end{equation}
      \subsubsection{Prędkość liniowa (styczna)}
        \begin{align}
          \vec v &= \vec\omega \times \vec r\unit{\frac{m}{s}}\\
          v &= \omega r\sin\measuredangle\left(\vec\omega, \vec r\right)
        \end{align}
        dla $\vec\omega \perp \vec r$:
        \begin{equation}
          v =\omega r
        \end{equation}
      \subsubsection{Przyspieszenie liniowe}
        \begin{equation}
          a_r = \varepsilon r \unit{\frac{m}{s^2}}
        \end{equation}
      \subsection{Równania obrotu}
        \begin{align}
          \omega &= \omega_0 \pm \varepsilon t\\
          \alpha &= \omega_0t \pm \frac{\varepsilon t^2}{2}\\
        \end{align}
        dla $\omega_0 = 0$:
        \begin{equation}
          \alpha = \frac{1}{2} \omega t
        \end{equation}
    \subsection{Moment bezwładności}
      \begin{equation}
        I = \sum_{i=1}^n m_ir_i^2 \unit{kg\cdot m^2}
      \end{equation}
      \subsubsection{Momenty bezwładności wybranych brył}
        \begin{align}
          \text{kula: } I_0 &= \frac{2}{5}mr^2\\
          \text{walec: } I_0 &= \frac{1}{2}mr^2\\
          \text{pręt: } I_0 &= \frac{1}{12}ml^2\\
          \text{rura grubościenna: } I_0 &= \frac{1}{2}m(r_1^2 + r_2^2)
        \end{align}
      \subsubsection{Twierdzenie Steinera}
        \begin{equation}
          I = I_0 +mx^2
        \end{equation}
    \subsection{Energia kinetyczna}
      \begin{align}
        E_{k_o} &= \sum_{i=1}^n \frac{m_iv_i}{2} \unit{J}\\
        E_{k_o} &= \frac{I\omega^2}{2}
      \end{align}
    \subsection{Moment siły}
      \begin{align}
        \vec M &= \vec r\times\vec F \unit{N\cdot m}\\
        M &= rF\sin\measuredangle\left(\vec r, \vec F\right)
      \end{align}
      dla $\vec r \perp \vec F$:
      \begin{equation}
        M = rF
      \end{equation}
      dla $\vec r \parallel \vec F$:
      \begin{equation}
        M = 0
      \end{equation}
      \subsubsection{Wypadkowy moment siły}
        \begin{align}
          M_w &= \sum_{i=1}^n M_i\\
          M_w &= \varepsilon I
        \end{align}
      \subsubsection{Równowaga bryły sztywnej}
        \begin{align}
          F_w &= 0\\
          M_w &= 0
        \end{align}
    \subsection{Moment pędu}
      \begin{align}
        \vec L &= \vec r \times\vec p \unit{\frac{kg\cdot m^2}{s}}\\
        L &= rp\sin\measuredangle\left(\vec r,\vec p\right)\\
          &= mrv\sin\measuredangle\left(\vec r,\vec v\right)\\
      \end{align}
      dla $\vec p \perp \vec r$:
      \begin{equation}
        L = rp = mrv
      \end{equation}
      \begin{equation}
        L = \sum_{i=1}^n m_ir_iv_i\sin\measuredangle\left(\vec r,\vec v\right)
      \end{equation}
      dla $\vec r \perp \vec v$:
      \begin{equation}
        L = \omega I
      \end{equation}

  \newpage
  \section{Grawitacja}

  \newpage
  \section{Ruch drgający}

  \newpage
  \section{Termodynamika}

  \newpage
  \section{Elektrostatyka}

  \newpage
  \section{Prąd elektryczny}

  \newpage
  \section{Magnetyzm}

\end{document}
