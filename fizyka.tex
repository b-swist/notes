\documentclass{article}

\title{Fizyka}
\date{2025-02-12}
\author{Bartosz Świst}

\usepackage[utf8]{inputenc}
\usepackage[T1]{fontenc}

\usepackage{amsmath, amssymb}
\usepackage{siunitx}

% \newcommand{\Vc}[2]{\vec{#1}_{\mathrm{#2}}}
% \newcommand{\Sc}[2]{#1_{\mathrm{#2}}}
% \newcommand{\F}[1]{\Vc{F}{#1}}
% \newcommand{\E}[1]{\Sc{E}{#1}}
% \newcommand{\W}[1]{\Sc{W}{#1}}
\newcommand{\F}{\vec{F}}

\begin{document}
  \maketitle
  \newpage

  \section{Kinematyka}

  \newpage
  \section{Dynamika}
    \subsection{Zasady dynamiki Newtona}
      \subsubsection{Pierwsza zasada}
        \begin{equation*}
          \F_w = 0 \Rightarrow \vec{v} = 0 \lor \vec{v} = const.
        \end{equation*}
      \subsubsection{Druga zasada}
        \begin{align*}
          \F_w \neq 0 &\Rightarrow a = const.\\
          \vec{a} = \frac{\F_w}{m} &\Rightarrow \F_w = m\vec{a}
        \end{align*}
      \subsubsection{Trzecia zasada}
        \begin{align*}
          \F_{AB} &= -\F_{BA}\\
          F_{AB} &= F_{BA}
        \end{align*}
    \subsection{Ruch na równi pochyłej}
      \begin{align*}
        \frac{\F_Z}{\F_g} = \sin\alpha &\Rightarrow \F_Z = \F_g\sin\alpha = mg\sin\alpha\\
        \frac{\F_N}{\F_g} = \cos\alpha &\Rightarrow \F_N = \F_g\cos\alpha = mg\cos\alpha
      \end{align*}
    \subsection{tbd}
    \subsection{Pęd ciała}
      \begin{equation*}
        \vec{p} = m\vec{v}
      \end{equation*}
      \begin{equation*}
        \Delta p = F\Delta t
      \end{equation*}
      \subsubsection{Zasada zachowania pędu}
        \begin{equation*}
          \Delta \vec{p} = 0 \Leftrightarrow \vec{p} = const.
        \end{equation*}
      \subsection{Środek masy}
        \begin{equation*}
          x_c = \frac{m_1x_1+m_2x_2+\dots+m_nx_n}{m_1+m_2+\dots+m_n}
        \end{equation*}
      \subsection{Tarcie}
        \begin{align*}
          T_s &= \mu_sF_N\\
          T_k &\leqslant \mu_kF_N
        \end{align*}
      \subsection{Siła dośrodkowa}
        \begin{equation*}
          F_{do} = \frac{mv^2}{r}
        \end{equation*}
        \subsection{Siła bezwładności}
        \begin{equation*}
          \F_b = -m\vec{a}
        \end{equation*}
  \newpage
  \section{Praca, moc, energia}
    \subsection{Praca}
      \begin{align*}
        W &= \F\Delta\vec{r} = F\Delta r \cos\measuredangle\left(\F, \Delta\vec{r}\right)\\
        \text{dla } \alpha = \ang{0}\colon\quad W &= F\Delta r\\
        \text{dla } \alpha = \ang{90}\colon\quad W &= 0
      \end{align*}
    \subsection{Moc}
      \begin{align*}
        &P = \frac{W}{t}\quad\\
      \text{dla }v = const.\colon\quad &P = Fv
      \end{align*}
    \subsection{Energia mechaniczna}
      \subsubsection{Energia kinetyczna}
        \begin{align*}
          &E_k = \frac{mv^2}{2}\\
          &\Delta E_k = W
        \end{align*}
      \subsubsection{Energia potencjalna}
        \begin{align*}
          &E_p = mgh\\
          &\Delta E_p = W
        \end{align*}
      \subsubsection{Zasada zachowania energii}
        \begin{align*}
          &E_c = E_k + E_p\\
          &E_c = const. \Rightarrow \Delta E_c = 0\\
          &\Delta E_c = \Delta E_p + \Delta E_k
        \end{align*}
    \subsection{Sprawność}
      \begin{equation*}
        \eta = \frac{E_{u\dot{z}yt.}}{E_{pob.}}\: (\cdot 100\%) = \frac{W_{u\dot{z}yt.}}{E_{pob.}}\: (\cdot 100\%)
      \end{equation*}
      \begin{equation*}
        \eta_{u\dot{z}yt.} = \sum_{i=1}^{n} \eta_i
      \end{equation*}
  \newpage
  \section{Hydrostatyka}

  \newpage
  \section{Bryła sztywna}

  \newpage
  \section{Grawitacja}

  \newpage
  \section{Ruch drgający}

  \newpage
  \section{Termodynamika}

  \newpage
  \section{Elektrostatyka}

  \newpage
  \section{Prąd elektryczny}

  \newpage
  \section{Magnetyzm}

\end{document}
