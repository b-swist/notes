\documentclass{article}

\title{Fizyka}
\date{2025-02-12}
\author{Bartosz Świst}

\usepackage[utf8]{inputenc}
\usepackage[T1]{fontenc}
\usepackage{amsmath, amssymb}

\numberwithin{equation}{section}
\newcommand{\unit}[1]{\, \left[\mathrm{#1}\right]}

\begin{document}
  \maketitle
  \newpage
  \section{Kinematyka}

  \newpage
  \section{Dynamika}
    \subsection{Zasady dynamiki Newtona}
      \subsubsection{Pierwsza zasada}
        \begin{equation}
          \vec F_w = 0 \Rightarrow \vec v = 0 \lor \vec v = const.
        \end{equation}
      \subsubsection{Druga zasada}
        \begin{align}
          \vec F_w \ne 0 &\Rightarrow a = const.\\
          \vec a = \frac{\vec F_w}{m} &\Rightarrow \vec F_w = m\vec a \unit{N}
        \end{align}
      \subsubsection{Trzecia zasada}
        \begin{align}
          \vec F_{AB} &= -\vec F_{BA}\\
          F_{AB} &= F_{BA}
        \end{align}
    \subsection{Ruch na równi pochyłej}
      \begin{align}
        \frac{\vec F_Z}{\vec F_g} = \sin\alpha &\Rightarrow \vec F_Z = \vec F_g\sin\alpha = mg\sin\alpha\\
        \frac{\vec F_N}{\vec F_g} = \cos\alpha &\Rightarrow \vec F_N = \vec F_g\cos\alpha = mg\cos\alpha
      \end{align}
    \subsection{tbd}
    \subsection{Pęd ciała}
      \begin{equation}
        \vec p = m\vec v \unit{\frac{kg\cdot m}{s}}
      \end{equation}
      \begin{equation}
        \Delta p = F\Delta t
      \end{equation}
      \subsubsection{Zasada zachowania pędu}
        \begin{equation}
          \Delta \vec p = 0 \Leftrightarrow \vec p = const.
        \end{equation}
      \subsection{Środek masy}
        \begin{equation}
          x_c = \frac{m_1x_1+m_2x_2+\dots+m_nx_n}{m_1+m_2+\dots+m_n} =
          \frac{\sum\limits_{i=1}^n m_ix_i}{\sum\limits_{i=1}^n m_i}
        \end{equation}
      \subsection{Tarcie}
        \begin{align}
          T_s &= \mu_sF_N \unit{N}\\
          T_k &\leqslant \mu_kF_N \unit{N}
        \end{align}
      \subsection{Siła dośrodkowa}
        \begin{equation}
          F_{do} = \frac{mv^2}{r} \unit{N}
        \end{equation}
        \subsection{Siła bezwładności}
        \begin{equation}
          \vec F_b = -m\vec a \unit{N}
        \end{equation}

  \newpage
  \section{Praca, moc, energia}
    \subsection{Praca}
      \begin{align}
        W &= \vec F\Delta\vec r \unit{J}\\
        W &= F\Delta r \cos\measuredangle\left(\vec F, \Delta\vec r\right)
      \end{align}
      dla $\alpha = 0^\circ$:
      \begin{equation}
        W = F\Delta r = Fs
      \end{equation}
      dla $\alpha = 90^\circ$:
      \begin{equation}
        W = 0
      \end{equation}
    \subsection{Moc}
      \begin{equation}
        P = \frac{W}{t} \unit{W}
      \end{equation}
      dla $v = const.$:
      \begin{equation}
        P = Fs
      \end{equation}
    \subsection{Energia mechaniczna}
      \subsubsection{Energia kinetyczna}
        \begin{align}
          &E_k = \frac{mv^2}{2} \unit{J}\\
          &\Delta E_k = W
        \end{align}
      \subsubsection{Energia potencjalna}
        \begin{align}
          &E_p = mgh \unit{J}\\
          &\Delta E_p = W
        \end{align}
      \subsubsection{Zasada zachowania energii}
        \begin{align}
          &E_c = E_k + E_p\\
          &E_c = const. \Rightarrow \Delta E_c = 0\\
          &\Delta E_c = \Delta E_p + \Delta E_k
        \end{align}
    \subsection{Sprawność}
      \begin{equation}
        \eta = \frac{E_{u\dot{z}yt.}}{E_{pob.}}\: (\cdot 100\%) = \frac{W_{u\dot{z}yt.}}{E_{pob.}}\: (\cdot 100\%)
      \end{equation}
      \begin{equation}
        \eta_{u\dot{z}yt.} = \prod_{i=1}^n \eta_i
      \end{equation}

  \newpage
  \section{Hydrostatyka}
    \subsection{Ciśnienie i parcie}
      \subsubsection{Ciśnienie}
        \begin{equation}
          p = \frac{F_N}{S} \unit{Pa}
        \end{equation}
        dla $F_N = mg$:
        \begin{equation}
          p = \frac{mg}{S}
        \end{equation}
      \subsubsection{Parcie}
        \begin{equation}
          P = pS \unit{N}
        \end{equation}
      \subsubsection{Ciśnienie hydrostatyczne}
        \begin{equation}
          p_h = \frac{P}{S} = \varrho_cgh \unit{Pa}
        \end{equation}
      \subsubsection{Paradoks hydrostatyczny}
    \subsection{Prawo Pascala}
      \begin{equation}
        p_1 = p_2 \Rightarrow \frac{F_1}{S_1} = \frac{F_2}{S_2}
      \end{equation}
      \subsubsection{Naczynia połączone}
        \begin{equation}
          p_1 = p_2 \Rightarrow \varrho_1h_1 = \varrho_2h_2
        \end{equation}
    \subsection{Prawo Archimedesa}
      \begin{equation}
        F_W = P_2 - P_1 = \varrho_cgV_z \unit{N}
      \end{equation}
      \subsubsection{Warunki wypływania}
        \begin{align*}
          F_W > F_g &\Rightarrow \text{ciało wypływa}\\
          F_W = F_g &\Rightarrow \text{ciało pływa}\\
          F_W < F_g &\Rightarrow \text{ciało tonie}
        \end{align*}

  \newpage
  \section{Bryła sztywna}
    \subsection{Ruch obrotowy}
      \subsubsection{Prędkość kątowa}
        \begin{equation}
          \omega = \frac{\Delta\alpha}{\Delta t} \unit{\frac{rad}{s},\;\frac{1}{s}}
        \end{equation}
      \subsubsection{Przyspieszenie kątowe}
        \begin{equation}
          \varepsilon = \frac{\Delta\omega}{\Delta t} \unit{\frac{rad}{s^2},\;\frac{1}{s^2}}
        \end{equation}
      \subsubsection{Prędkość liniowa (styczna)}
        \begin{align}
          \vec v &= \vec\omega \times \vec r\unit{\frac{m}{s}}\\
          v &= \omega r\sin\measuredangle\left(\vec\omega, \vec r\right)
        \end{align}
        dla $\vec\omega \perp \vec r$:
        \begin{equation}
          v =\omega r
        \end{equation}
      \subsubsection{Przyspieszenie liniowe}
        \begin{equation}
          a_r = \varepsilon r \unit{\frac{m}{s^2}}
        \end{equation}
      \subsection{Równania obrotu}
        \begin{align}
          \omega &= \omega_0 \pm \varepsilon t\\
          \alpha &= \omega_0t \pm \frac{\varepsilon t^2}{2}\\
        \end{align}
        dla $\omega_0 = 0$:
        \begin{equation}
          \alpha = \frac{1}{2} \omega t
        \end{equation}
    \subsection{Moment bezwładności}
      \begin{equation}
        I = \sum_{i=1}^n m_ir_i^2 \unit{kg\cdot m^2}
      \end{equation}
      \subsubsection{Momenty bezwładności wybranych brył}
        \begin{align}
          \text{kula: } I_0 &= \frac{2}{5}mr^2\\
          \text{walec: } I_0 &= \frac{1}{2}mr^2\\
          \text{pręt: } I_0 &= \frac{1}{12}ml^2\\
          \text{rura grubościenna: } I_0 &= \frac{1}{2}m(r_1^2 + r_2^2)
        \end{align}
      \subsubsection{Twierdzenie Steinera}
        \begin{equation}
          I = I_0 +mx^2
        \end{equation}
    \subsection{Energia kinetyczna}
      \begin{align}
        E_{k_o} &= \sum_{i=1}^n \frac{m_iv_i}{2} \unit{J}\\
        E_{k_o} &= \frac{I\omega^2}{2}
      \end{align}
    \subsection{Moment siły}
      \begin{align}
        \vec M &= \vec r\times\vec F \unit{N\cdot m}\\
        M &= rF\sin\measuredangle\left(\vec r, \vec F\right)
      \end{align}
      dla $\vec r \perp \vec F$:
      \begin{equation}
        M = rF
      \end{equation}
      dla $\vec r \parallel \vec F$:
      \begin{equation}
        M = 0
      \end{equation}
      \subsubsection{Wypadkowy moment siły}
        \begin{align}
          M_w &= \sum_{i=1}^n M_i\\
          M_w &= \varepsilon I
        \end{align}
      \subsubsection{Równowaga bryły sztywnej}
        \begin{align}
          F_w &= 0\\
          M_w &= 0
        \end{align}
    \subsection{Moment pędu}
      \begin{align}
        \vec L &= \vec r \times\vec p \unit{\frac{kg\cdot m^2}{s}}\\
        L &= rp\sin\measuredangle\left(\vec r,\vec p\right)\\
          &= mrv\sin\measuredangle\left(\vec r,\vec v\right)\\
      \end{align}
      dla $\vec p \perp \vec r$:
      \begin{equation}
        L = rp = mrv
      \end{equation}
      \begin{equation}
        L = \sum_{i=1}^n m_ir_iv_i\sin\measuredangle\left(\vec r,\vec v\right)
      \end{equation}
      dla $\vec r \perp \vec v$:
      \begin{equation}
        L = \omega I
      \end{equation}

  \newpage
  \section{Grawitacja}
    \subsection{Prawa Keplera}
      \subsubsection{Pierwsze prawo}
      \subsubsection{Drugie prawo}
        \begin{align}
          s_1 &= s_2\\
          L_1 &= L_2 \Rightarrow r_1v_1 = r_2v_2
        \end{align}
      \subsubsection{Trzecie prawo}
        \begin{equation}
          \frac{T^2}{r^3} = const.
        \end{equation}
    \subsection{Prawo powszechnego ciążenia}
      \begin{equation}
        F = G\frac{m_1m_2}{r^2} \unit{N}
      \end{equation}
      gdzie:
      \begin{equation}
        G = 6,67\cdot 10^{-11} \unit{\frac{N\cdot m^2}{kg^2}}
      \end{equation}
      \begin{align}
        F &= \frac{4}{3}\pi RGdm\\
        F &\sim dR
      \end{align}
    \subsection{Natężenie pola grawitacyjnego}
      \begin{equation}
        \vec\gamma = \frac{\vec F_g}{m} \unit{\frac{N}{kg},\;\frac{m}{s^2}}
      \end{equation}
      dla pola centralnego:
      \begin{equation}
        \gamma = \frac{GM}{r^2}
      \end{equation}
    \subsection{Praca w polu grawitacyjnym}
      \begin{align}
        &W = mgh\\
        &\Delta E_p = W
      \end{align}
      \begin{align}
        W_{Z_{(A\rightarrow B)}} &= GMm\left(\frac{1}{r_A} - \frac{1}{r_B}\right)\\
        W_{g_{(A\rightarrow B)}} &= -W_{Z_{(A\rightarrow B)}}
      \end{align}
    \subsection{Energia w polu grawitacyjnym}
      \begin{equation}
        E_p = -\frac{GMm}{r}
      \end{equation}
    \subsection{Potencjał pola grawitacyjnego}
      \begin{align}
        &V = \frac{E_p}{m} \unit{\frac{J}{kg}}\\
        &\Delta V = \frac{\Delta E_p}{m}
      \end{align}
    \subsection{Prędkości kosmiczne}
      \subsubsection{Pierwsza prędkość kosmiczna}
        \begin{equation}
          v_{{}_\mathrm{I}} = \sqrt{\frac{GM}{r}}
        \end{equation}
      \subsubsection{Satelita geostacjonarny}
        \begin{equation}
          r = \sqrt[3]{\frac{GMT^2}{4\pi^4}}
        \end{equation}
      \subsubsection{Druga prędkość kosmiczna}
        \begin{align}
          v_{{}_\mathrm{II}} &= \sqrt{\frac{2GM}{r}}\\
          v_{{}_\mathrm{II}} &= v_{{}_\mathrm{I}}\sqrt{2}
        \end{align}

  \newpage
  \section{Ruch drgający}

  \newpage
  \section{Termodynamika}
    \subsection{Zerowa zasada dynamiki}
      \begin{equation}
        p = \frac{2}{3}\cdot\frac{NE_{k_{\acute sr.}}}{V} \unit{Pa}
        % N - liczba cząstek gazu
      \end{equation}
      \begin{equation}
        E_{k_{\acute sr.}} = \frac{1}{2}mv_{\acute sr.}^2 \unit{J}
      \end{equation}
    \subsection{Równanie gazu doskonałego}
      \begin{equation}
        \frac{p_1V_1}{T_1} = \frac{p_2V_2}{T_2} \Rightarrow \frac{pV}{T} = const.
      \end{equation}
      \subsubsection{Równanie Clapeyrona}
        \begin{equation}
          pV = nRT = NkT
        \end{equation}
        gdzie:
        \begin{align}
          &R =  8,31 \unit{\frac{J}{mol\cdot K}}\\
          &k = \frac{R}{N_A} = 1,38\cdot 10^{-23} \unit{\frac{J}{K}}
        \end{align}
    \subsection{Przemiany gazu doskonałego}
      \subsubsection{Przemiana izotermiczna}
        \begin{align}
          &T = const.\\
          &\frac{p_1V_1}{T} = \frac{p_2V_2}{T} \Rightarrow p_1V_1 = p_2V_2\\
          &pV = const. \Rightarrow p = \frac{const.}{V} \quad\text{(Prawo Boyle'a)}
        \end{align}
      \subsubsection{Przemiana izochoryczna}
        \begin{align}
          &V = const.\\
          &\frac{p_1V}{T_1} = \frac{p_2V}{T_2} \Rightarrow \frac{p_1}{T_1} = \frac{p_2}{T_2}\\
          &\frac{p}{T} = const. \Rightarrow p = T\cdot const. \quad\text{(Prawo Charles'a)}
        \end{align}
      \subsubsection{Przemiana izochoryczna}
        \begin{align}
          &p = const.\\
          &\frac{pV_1}{T_1} = \frac{pV_2}{T_2} \Rightarrow \frac{V_1}{T_1} = \frac{V_2}{T_2}\\
          &\frac{V}{T} = const. \Rightarrow V = T\cdot const. \quad\text{(Prawo Gay-Lussaca)}
        \end{align}
    \subsection{Pierwsza zasada termodynamiki}
      \begin{equation}
        \Delta U = Q + W_Z \unit{J}
      \end{equation}
      dla $Q > 0$ ciepło zostało pobrane\\
      dla $Q < 0$ ciepło zostało oddane
      \begin{align}
        W_Z &= F_Z\Delta x\cos\measuredangle\left(\vec F_Z, \Delta\vec x\right)\\
        W_Z &= -W_{gazu}
      \end{align}
      dla $W_Z > 0$:
      \begin{equation}
        W_Z = F_Z\Delta x
      \end{equation}
      dla $W_Z < 0$:
      \begin{equation}
        W_Z = -F_Z\Delta x
      \end{equation}
      \begin{equation}
        |W| = p|\Delta V|
      \end{equation}
    \subsection{Energia wewnętrzna gazu doskonałego}
      \begin{align}
        &U = N\cdot\frac{i}{2}kT\\
        &\Delta U = N\cdot\frac{i}{2}k\Delta T
      \end{align}
      \subsubsection{Przemiana izotermiczna}
        \begin{align}
          &T = const. \Leftrightarrow U = const.\\
          &\Delta U = 0 \Rightarrow Q + W = 0
        \end{align}
      \subsubsection{Przemiana izochoryczna}
        \begin{align}
          &V = const. \Rightarrow \Delta V = 0\\
          &W = 0 \Rightarrow \Delta U = Q
        \end{align}
      \subsubsection{Przemiana adiabatyczna}
        \begin{align}
          &Q = 0 \Rightarrow \Delta U = W\\
          &pV^\kappa = const.
        \end{align}
        gdzie:
        \begin{equation}
          \kappa = \frac{C_p}{C_V}
        \end{equation}
    \subsection{Ciepło molowe i właściwe}
      \subsubsection{Ciepło właściwe}
        \begin{align}
          &C_w = \frac{Q}{m\Delta T} \unit{\frac{J}{kg\cdot K}}\\
          &Q = mC_w\Delta T
        \end{align}
      \subsubsection{Ciepło molowe}
        \begin{equation}
          C = \frac{Q}{n\Delta T} \unit{\frac{J}{mol\cdot K}}
        \end{equation}
        ciepło molowe przy stałym ciśnieniu: $C_p$\\
        ciepło molowe przy stałej objętości: $C_V$
        \begin{align}
          &Q_p = Q_V + p\Delta V\\
          &C_p = C_V + R
        \end{align}
    \subsection{Energia wewnętrzna jako funkcja stanu}
      \begin{equation}
        \Delta U = Q_V = nC_V\Delta T
      \end{equation}
    \subsection{Silnik cieplny}
      \begin{equation}
        \eta = \frac{|Q_1|-|Q_2|}{Q_1} = \frac{T_1 - T_2}{T_1}
      \end{equation}
    \subsection{Przejścia fazowe}
      \begin{equation}
        Q - mC_w\Delta T
      \end{equation}
      woda - lód: $T_T = T_K = 0^\circ C$\\
      woda - para wodna: $T_W = T_S = 100^\circ C$
      \begin{equation}
        Q = mL
      \end{equation}
      \begin{equation}
        Q = mR
      \end{equation}
    \subsection{Rozszerzalność temperaturowa ciał}
      \subsubsection{Rozszerzalność obiętościowa}
        \begin{equation}
          \Delta V = V_0\alpha\Delta T
        \end{equation}
      \subsubsection{Rozszerzalność liniowa}
        \begin{equation}
          \Delta l = l_0\lambda\Delta T
        \end{equation}

  \newpage
  \section{Elektrostatyka}
    \subsection{Ładunek}
      \begin{align}
        &e = 1,6\cdot 10^{-19} \unit{C}\\
        &q = ne \unit{C}
      \end{align}
    \subsection{Prawo Culomba}
      \begin{equation}
        F = k\frac{q_1q_2}{r^2} \unit{N}
      \end{equation}
      gdzie:
      \begin{align}
        &k = \frac{1}{4\pi\varepsilon_0} \approx 8,99\cdot 10^9 \unit{\frac{N\cdot m^2}{C^2}}\\
        &\varepsilon_0 = 8,85\cdot 10^{-12} \unit{\frac{C^2}{N\cdot m^2}}
      \end{align}
    \subsection{Natężenie pola elektrostatycznego}
      \begin{align}
        \vec E &= \frac{\vec F}{q} \unit{\frac{N}{C}}\\
        E &= \frac{k|Q|}{r^2}
      \end{align}
    \subsection{Rozmieszczenie ładunku na przewodniku}
      \begin{equation}
        \sigma = \frac{Q}{S} \unit{\frac{C}{m^2}}
      \end{equation}
    \subsection{Praca, energia i potencjał w polu elektrostatycznym centralnym}
      \begin{equation}
        W_{A\rightarrow B} = -kQq\left(\frac{1}{r_A} - \frac{1}{r_B}\right) \unit{J}
      \end{equation}
      \begin{equation}
        E_p = \frac{kQq}{r} \unit{J}
      \end{equation}
      \begin{align}
        V = \frac{E_p}{q} = \frac{kQ}{r} \unit{V}\\
        W = q\Delta V = qU
      \end{align}
    \subsection{Pojemność elektryczna przewodnika}
      \begin{equation}
        C = \frac{Q}{V} \unit{F}
      \end{equation}
    \subsection{Kondensator}
      \begin{equation}
        C= \frac{Q}{U}
      \end{equation}
      dla kondensatora płaskiego:
      \begin{equation}
        C= \frac{\varepsilon_0S}{d}
      \end{equation}
      \begin{equation}
        E = \frac{U}{d}
      \end{equation}
      \subsubsection{Łączenie kondensatorów}
        \begin{align}
          &Q = const.\\
          &U = \sum_{i=1}^n U_i\\
          &\frac{1}{C_z} = \sum_{i=1}^n \frac{1}{C_i}
        \end{align}
        \begin{align}
          &U = const.\\
          &Q = \sum_{i=1}^n Q_i\\
          &C_z = \sum_{i=1}^n C_i
        \end{align}
      \subsubsection{Kondensator z dielektrykiem}
      \begin{equation}
        C= \frac{\varepsilon_0\varepsilon_rS}{d} \unit{F}
      \end{equation}
      gdzie $\varepsilon_r$ - stała przenikalności dielektryka
      \subsubsection{Energia naładowaniego kondensatora}
        \begin{equation}
          E = \frac{1}{2}QU = \frac{1}{2}CU^2 = \frac{Q^2}{2C}
        \end{equation}
    \subsection{Ruch ładunków w polu elektrostatycznym}
      \begin{equation}
        F = qE
      \end{equation}
      jeżeli $F = F_w$:
      \begin{equation}
        a = \frac{qE}{m} = \frac{qU}{md} \unit{\frac{m}{s^2}}
      \end{equation}

  \newpage
  \section{Prąd elektryczny}
    \subsection{Natężenie prądu elektrycznego}
      \begin{equation}
        I = \frac{\Delta Q}{\Delta t} \unit{A}
      \end{equation}
    \subsection{Pierwsze prawo Kirchoffa}
      \begin{equation}
        \sum_{i=1}^n I_i = 0
      \end{equation}
    \subsection{Prawo Ohma}
      \begin{align}
        \frac{U}{I} &= const.\\
        R &= \frac{U}{I} \unit{\Omega}
      \end{align}
      \subsubsection{Wybrane charakterystyki prądowo-napięciowe}
      \subsubsection{Łączenie rezystorów}
        \begin{align}
          &I = const.\\
          &U = \sum_{i=1}^n U_i\\
          &R_z = \sum_{i=1}^n R_i
        \end{align}
        \begin{align}
          &U = const.\\
          &I = \sum_{i=1}^n I_i\\
          &\frac{1}{R_z} = \sum_{i=1}^n \frac{1}{R_i}
        \end{align}
      \subsubsection{Opór przewodnika}
        \begin{equation}
          R = \frac{\varrho l}{S} \unit{\Omega}
        \end{equation}
      gdzie $\varrho$ - opór właściwy materiału przewodnika
    \subsection{Praca prądu elektrycznego}
      \begin{equation}
        W = UIt = \frac{U^2}{R}t = I^2Rt
      \end{equation}
      \subsubsection{Emisja ciepła}
        \begin{equation}
          Q = W = I^2Rt \quad\text{(Ciepło Joule'a)}
        \end{equation}
      \subsubsection{Energia elektryczna}
        \begin{equation}
          E_{el} = W = UIt \unit{kWh}
        \end{equation}
    \subsection{Moc prądu elektrycznego}
      \begin{align}
        &P = \frac{W}{t} \unit{W}\\
        &P = UI = I^2R = \frac{U^2}{R}
      \end{align}
    \subsection{Ogniwo galwaniczne}
      \subsubsection{Prawo Ohma dla obwodu}
        \begin{align}
          &\mathcal E = U + U_w = U + IR_w\\
          &U = \mathcal E - IR_w
        \end{align}
        \begin{equation}
          I = \frac{\mathcal E}{R_z} = \frac{\mathcal E}{R + R_w} \unit{A}
        \end{equation}
    \subsection{Drugie prawo Kirchoffa}
      \begin{equation}
        \sum_{i=1}^n \mathcal E_i + \sum_{i=1}^n I_iR_i = 0
      \end{equation}
    \subsection{Przewodnictwo ciał stałych}
    \subsection{Dioda półprzewodnikowa}

  \newpage
  \section{Magnetyzm}

\end{document}
