\chapter{Magnetyzm}
  \section{Pole magnetyczne}
    \begin{definition}
      \textbf{Pole magnetyczne} to przestrzeń, w której na umieszczone w niej naładowane cząstki oraz ciała o~właściwościach magnetycznych działają siły magnetyczne. Ciała wykazujące właściwości magnetyczne posiadają \textbf{domeny magnetyczne} --- obszary o~stałym namagnesowaniu. Źródłem pola magnetycznego może być np. magnes trwały lub~przewodnik z~prądem (doświadczenie Ørsteda)
      \begin{center}
        \begin{tikzpicture}
          % magnes
        \end{tikzpicture}
      \end{center}
    \end{definition}

    \begin{definition}
      \textbf{Indukcja pola magnetycznego} to wielkość wektorowa wyrażająca natężenie pola magnetycznego w~danym punkcie przestrzeni.
      \begin{equation}
        B = \frac{F}{qv} \unit{T}
      \end{equation}
    \end{definition}

    \subsection{Pole magnetyczne prostoliniowego przewodnika}
      \begin{equation*}
        B = \frac{\mu_0 I}{2\pi r}
      \end{equation*}
      \begin{symbols}
        \item $\mu_0$ -- stała przenikalności magnetycznej próżni
      \end{symbols}
      \begin{equation*}
        \mu_0 = 4\pi\cdot 10^{-7} \units{\frac{Tm}{A}}{\frac{N}{A^2}}
      \end{equation*}

    \subsection{Pole magnetyczne gęstej zwojnicy}
      \begin{equation*}
        B = \frac{\mu_0 nI}{l}
      \end{equation*}
      \begin{symbols}
        \item $n$ -- ilość nawiniętych zwojów
        \item $l$ -- długość zwojnicy
      \end{symbols}

    \subsection{Pole magnetyczne pętli (pojedynczego zwoju)}
      \begin{equation*}
        B = \frac{\mu_0 I}{2r}
      \end{equation*}

  \section{Siła elektrodynamiczna}
    \begin{definition}
      \textbf{Siła elektrodynamiczna} to siła działająca na~umieszczony w~polu magnetycznym przewodnik, przez który przepływa prąd elektryczny, powodując jego ruch.
      \begin{gather*}
          \vec F = I\Delta\vec l\times\vec B \unit{N}\\
          F = I\Delta l\cdot B\sin\measuredangle(\Delta\vec l, \vec B)
      \end{gather*}
      jeżeli $\Delta\vec l \perp \vec B$, to:
      \begin{equation}
        \boxed{F = BI\Delta l}
      \end{equation}
    \end{definition}

  \section{Siła Lorenza}
    \begin{definition}
      \textbf{Siła Lorenza} to siła działająca na~naładowane cząstki poruszające się w~polu magnetycznym powodująca odchylenie toru ruchu tych cząstek.
    \begin{gather*}
      \vec F_L = q\vec v\times\vec B \unit{N}\\
      F_L = qvB\sin\measuredangle(\vec v, \vec B)
    \end{gather*}
    jeżeli $\vec v\perp\vec B$, to:
    \begin{equation}
      \boxed{F_L = qvB}
    \end{equation}
    \end{definition}

  \section{Cyklotron}
    Cyklotron służy do przyspieszania cząstek obdarzonych ładunkiem. Składa się z dwóch duantów w kształcie puszki.
