\chapter{Magnetyzm}
  \section{Pole magnetyczne}
    \begin{definition}
      \textbf{Pole magnetyczne} to przestrzeń, w której na umieszczone w niej naładowane cząstki oraz ciała o~właściwościach magnetycznych działają siły magnetyczne. Ciała wykazujące właściwości magnetyczne posiadają \textbf{domeny magnetyczne} --- obszary o~stałym namagnesowaniu. Źródłem pola magnetycznego może być np. magnes trwały lub~przewodnik z~prądem (doświadczenie Ørsteda).
    \end{definition}

    \begin{subequations}
    \subsection{Pole magnetyczne prostoliniowego przewodnika}
      \begin{equation}
        B = \frac{\mu_0 I}{2\pi r}
      \end{equation}
      \begin{symbols}
        \item $\mu_0$ -- stała przenikalności magnetycznej próżni
      \end{symbols}
      \begin{equation*}
        \mu_0 = 4\pi\cdot 10^{-7} \units{\frac{Tm}{A}}{\frac{N}{A^2}}
      \end{equation*}
      \begin{law}{Reguła prawej prawej ręki dla prostoliniowego przewodnika}
        Jeżeli kciuk zaciśniętej wokół prewodnika prawej ręki wskazuje kierunek przepływu prądu, to reszta palców tej ręki wskazuje zwrot linii pola magrnetyczego.
      \end{law}

    \subsection{Pole magnetyczne gęstej zwojnicy}
      \begin{equation}
        B = \frac{\mu_0 nI}{l}
      \end{equation}
      \begin{symbols}
        \item $n$ -- ilość nawiniętych zwojów
        \item $l$ -- długość zwojnicy
      \end{symbols}
      \begin{law}{Reguła prawej ręki dla zwojnicy}
        Jeżeli palce prawej ręki (oprócz kciuka) wskazują kierunek przepływu prądu w zwojach zwojnicy, to kciuk wskazuje zwrot, w którym znajduje się północny biegun magnesu.
      \end{law}

    \subsection{Pole magnetyczne pętli (pojedynczego zwoju)}
      \begin{equation}
        B = \frac{\mu_0 I}{2r}
      \end{equation}
    \end{subequations}

  \section{Siła elektrodynamiczna}
    \begin{definition}
      \textbf{Siła elektrodynamiczna} to siła działająca na~umieszczony w~polu magnetycznym przewodnik, przez który przepływa prąd elektryczny, powodując jego ruch.
      \begin{gather*}
          \vec F = I\Delta\vec l\times\vec B \unit N\\
          F = I\Delta l\cdot B\sin\measuredangle(\Delta\vec l, \vec B)
      \end{gather*}
      jeżeli $\Delta\vec l \perp \vec B$, to:
      \begin{equation}
        \boxed{F = BI\Delta l}
      \end{equation}
    \end{definition}
    \begin{law}{Zasada lewej ręki dla siły elektrodynamicznej}
      Jeżeli palce lewej ręki (oprócz kciuka) wskazują kierunek przepływu prądu, a wewnętrzna część dłoni jest skierowana tak, że wbijają się w nią linie pola magnetycznego, to kciuk wskazuje zwrot działania siły elektrodynamicznej.
    \end{law}

  \section{Siła Lorenza}
    \begin{definition}
      \bi{Siła Lorenza} to siła działająca na~naładowane cząstki poruszające się w~polu magnetycznym powodująca odchylenie toru ruchu tych cząstek.
    \begin{gather*}
      \vec F_L = q\vec v\times\vec B \unit N\\
      F_L = qvB\sin\measuredangle(\vec v, \vec B)
    \end{gather*}
    jeżeli $\vec v\perp\vec B$, to:
    \begin{equation}
      \boxed{F_L = qvB}
    \end{equation}
    \end{definition}
    \begin{law}{Zasada lewej ręki dla siły Lorenza}
      Jeżeli palce lewej ręki (oprócz kciuka) wskazują zwrot wektora prędkości cząstki, a wewnętrzna część dłoni jest skierowana tak, że wbijają się w nią linie pola magnetycznego, to kciuk wskazuje zwrot działania siły Lorenza dla cząstek naładowanych dodatnie, a przeciwny zwrot dla cząstek naładowanych ujemnie.
    \end{law}

  \section{Cyklotron}
    Cyklotron służy do przyspieszania cząstek obdarzonych ładunkiem. Składa się z dwóch duantów w kształcie puszki.

  \section{Wzajemne oddziaływanie przewodników z prądem}
    Jeżeli w równoległych przewodnikach prąd płynie w tym samym kierunku, to te przewodniki się przyciągają, a jeżeli prąd płynie w przeciwnych kierunkach, to się odpychają.
    \begin{center}
      \begin{tikzpicture}
      \end{tikzpicture}
    \end{center}
    \begin{equation*}
      F = \frac{\mu_0I_1I_2\Delta l}{2\pi r} \unit N
    \end{equation*}
    jeżeli $I_1 = I_2$, to:
    \begin{equation*}
      F = \frac{\mu_0I^2\Delta l}{2\pi r}
    \end{equation*}

  \section{Właściwości magnetyczne substancji}
    Ze względu na właściości magnetyczne substancje dzielimy na:
    \begin{itemize}
      \item ferromagnetyki
      \item paramagnetyki
      \item diamagnetyki
    \end{itemize}


    \bi{Ferromagnetyki} posiadają silne właściwości magnetyczne, np. żelazo, kobalt, nikiel. ($\mu_r \gg 1$)

    \bi{Paramagnetyki} wykazują słabe właściwości magnetyczne, np. aluminium, cyna, magnez, hemoglobina, ciekły tlen. ($\mu_r > 1$)

    \bi{Diamagnetyki} nie wykazują żadnych właściwości magnetycznych, osłabiają pola magnetyczne zewnętrzne, wypychają pole mangetyczne ze swojego wnętrza, np. miedź, złoto, ołów, cynk, gazy szlachetne. ($\mu_r <1$)

    \begin{equation*}
      \mu_r = \frac{B}{B_0}
    \end{equation*}
    \begin{symbols}
     \item $\mu_r$ -- względna przenikalność magnetyczna substancji
     \item $B$ -- indukcja magnetyczna wewnątrz substancji
     \item $B_0$  -- indukcja magnetyczna w próżni
    \end{symbols}

    \subsection{Pętla histerezy ferromagnetyka}

  \section{Silnik elektryczny}
    Silnik elektryczny zamienia energię elektryczną w energię mechaniczną. Zbudowany jest ze:
    \begin{itemize}
      \item stojana, na którym umieszczono elektromagnesy
      \item wirnika
      \item komutatora, który zamienia kierunek przepływu prądu
      \item szczotek, np. grafitowych
    \end{itemize}

    Prąd elektryczny, przepływając przez wirnik składający się z wielu uzwojeń, wytwarza pole magnetyczne, które oddziałuje z polem magnetycznym elektektromagnesów. Pojawia się siła elektrodynamiczna, która powoduje obrót wirnika.
