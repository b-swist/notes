  \chapter{Magnetyzm}
    \section{Pole magnetyczne}
      \begin{definition}
        \textbf{Pole magnetyczne} to przestrzeń, w~której na~umieszczone w niej naładowane cząstki oraz ciała o~właściwościach magnetycznych działają siły magnetyczne. Ciała wykazujące właściwości magnetyczne posiadają \textbf{domeny magnetyczne} --- obszary o~stałym namagnesowaniu. Źródłem pola magnetycznego może być np. magnes trwały lub~przewodnik z~prądem (doświadczenie Ørsteda)
      \end{definition}
      \begin{center}
        \begin{tikzpicture}
          % magnes
        \end{tikzpicture}
      \end{center}
      \begin{definition}
        \textbf{Indukcja pola magnetycznego} to~wielkość wektorowa wyrażająca natężenie pola magnetycznego w~danym punkcie przestrzeni.
      \end{definition}
      \begin{subequations}
        \begin{equation}
          B = \frac{F}{qv} \unit{T}
        \end{equation}
        \subsection{Pole magnetyczne prostoliniowego przewodnika}
          \begin{equation}
            B = \frac{\mu_0 I}{2\pi r}
          \end{equation}
          gdzie: $\mu_0$ -- stała przenikalności magnetycznej próżni
          \begin{equation*}
            \mu_0 = 4\pi\cdot 10^{-7} \unit{\frac{Tm}{A},\;\frac{N}{A^2}}
          \end{equation*}
        \subsection{Pole magnetyczne gęstej zwojnicy}
          \begin{equation}
            B = \frac{\mu_0 nI}{l}
          \end{equation}
          gdzie:
          $n$ -- ilość nawiniętych zwojów\\
          \tab $l$ -- długość zwojnicy
        \subsection{Pole magnetyczne pętli (pojedynczego zwoju)}
          \begin{equation}
            B = \frac{\mu_0 I}{2r}
          \end{equation}
      \end{subequations}
    \section{Siła elektrodynamiczna}
      \begin{definition}
        \textbf{Siła elektrodynamiczna} to~siła działająca na~umieszczony w~polu magnetycznym przewodnik, przez który przepływa prąd elektryczny, powodując jego ruch.
      \end{definition}
      \begin{subequations}
        \begin{equation}
          \begin{gathered}
            \vec F = I\Delta\vec l\times\vec B \unit{N}\\
            F = I\Delta l\cdot B\sin\measuredangle(\Delta\vec l, \vec B)
          \end{gathered}
        \end{equation}
        jeżeli $\Delta\vec l \perp \vec B$, to:
        \begin{equation}
          \boxed{F = BI\Delta l}
        \end{equation}
      \end{subequations}
  \section{Siła Lorenza}
    \begin{definition}
      \textbf{Siła Lorenza} to~siła działająca na~naładowane cząstki poruszające się w~polu magnetycznym powodująca odchylenie toru ruchu tych cząstek.
    \end{definition}
    \begin{subequations}
      \begin{equation}
        \begin{gathered}
          \vec F_L = q\vec v\times\vec B \unit{N}\\
          F_L = qvB\sin\measuredangle(\vec v, \vec B)
        \end{gathered}
      \end{equation}
      jeżeli $\vec v\perp\vec B$, to:
      \begin{equation}
        \boxed{F_L = qvB}
      \end{equation}
    \end{subequations}
