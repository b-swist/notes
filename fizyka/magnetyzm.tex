  \chapter{Magnetyzm}
    \section{Indukcja magnetyczna}
      \subsection{Pole magnetyczne prostoliniowego przewodnika}
        \begin{equation}
          B = \frac{\mu_0 I}{2\pi r} \unit{T}
        \end{equation}
        gdzie:
        \begin{equation}
          \mu_0 = 4\pi\cdot 10^{-7} \unit{\frac{Tm}{A},\;\frac{N}{A^2}}
        \end{equation}
      \subsection{Pole magnetyczne gęstej zwojnicy}
        \begin{equation}
          B = \frac{\mu_0 nI}{l} \unit{T}
        \end{equation}
      \subsection{Pole magnetyczne pętli}
        \begin{equation}
          B = \frac{\mu_0 I}{2r} \unit{T}
        \end{equation}
    \section{Siła elektrodynamiczna}
      \begin{equation}
        F = BI\Delta l\sin\measuredangle(\Delta\vec l, \vec B) \unit{N}
      \end{equation}
      dla $\Delta\vec l \perp \vec B$:
      \begin{equation}
        F = BI\Delta l
      \end{equation}
  \section{Siła Lorenza}
    \begin{gather}
      \vec F_L = q\vec v\times\vec B \unit{N}\\
      F_L = qvB\sin\measuredangle(\vec v, \vec B)
    \end{gather}
    dla $\vec v\perp\vec B$:
    \begin{equation}
      F_L = qvB
    \end{equation}
