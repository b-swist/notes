  \chapter{Prąd elektryczny}
    \section{Natężenie prądu elektrycznego}
      \begin{equation}
        I = \frac{\Delta Q}{\Delta t} \unit{A}
      \end{equation}
    \section{Pierwsze prawo Kirchoffa}
      \begin{equation}
        \sum_{i=1}^n I_i = 0
      \end{equation}
    \section{Prawo Ohma}
      \begin{gather}
        \frac{U}{I} = const.\\
        R = \frac{U}{I} \unit{\Omega}
      \end{gather}
      \subsection{Wybrane charakterystyki prądowo-napięciowe}
      \subsection{Łączenie rezystorów}
        \begin{gather}
          I = const.\\
          U = \sum_{i=1}^n U_i\\
          R_z = \sum_{i=1}^n R_i
        \end{gather}
        \begin{gather}
          U = const.\\
          I = \sum_{i=1}^n I_i\\
          \frac{1}{R_z} = \sum_{i=1}^n \frac{1}{R_i}
        \end{gather}
      \subsection{Opór przewodnika}
        \begin{equation}
          R = \frac{\varrho l}{S} \unit{\Omega}
        \end{equation}
      gdzie $\varrho$ - opór właściwy materiału przewodnika
    \section{Praca prądu elektrycznego}
      \begin{equation}
        W = UIt = \frac{U^2}{R}t = I^2Rt
      \end{equation}
      \subsection{Emisja ciepła}
        \begin{equation}
          Q = W = I^2Rt \quad\text{(Ciepło Joule'a)}
        \end{equation}
      \subsection{Energia elektryczna}
        \begin{equation}
          E_{el} = W = UIt \unit{kWh}
        \end{equation}
    \section{Moc prądu elektrycznego}
      \begin{gather}
        P = \frac{W}{t} \unit{W}\\
        P = UI = I^2R = \frac{U^2}{R}
      \end{gather}
    \section{Ogniwo galwaniczne}
      \subsection{Prawo Ohma dla obwodu}
        \begin{gather}
          \mathcal E = U + U_w = U + IR_w\\
          U = \mathcal E - IR_w
        \end{gather}
        \begin{equation}
          I = \frac{\mathcal E}{R_z} = \frac{\mathcal E}{R + R_w} \unit{A}
        \end{equation}
    \section{Drugie prawo Kirchoffa}
      \begin{equation}
        \sum_{i=1}^n \mathcal E_i + \sum_{i=1}^n I_iR_i = 0
      \end{equation}
    \section{Przewodnictwo ciał stałych}
    \section{Dioda półprzewodnikowa}
