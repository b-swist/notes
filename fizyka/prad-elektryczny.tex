\chapter{Prąd elektryczny}

\begin{definition}
  \bi{Prąd elektryczny} to uporządkowany ruch ładunków elektrycznych, których nośnikami w metalach
  są \textbf{elektrony}.
\end{definition}

\begin{definition}
  \bi{Napięcie elektryczne} to różnica potencjałów między dwoma punktami obwodu elektrycznego
  powodująca przepływ ładunków.
  \begin{equation}
    U = \Delta V \unit V
  \end{equation}
\end{definition}

\begin{definition}
  \bi{Natężenie prądu elektrycznego} to stosunek ilości ładunków przepływających przez przekrój
  poprzeczny przewodu do czasu, w którym ten ładunek przepłynął.
  \begin{equation}
    I = \frac{\Delta Q}{\Delta t} \unit A
  \end{equation}
\end{definition}

\section{Prawo Ohma}
\begin{law}{Prawo Ohma}
  Natężenie prądu płynącego przez przewodnik jest wprost proporcjonalne do napięcia pomiędzy
  końcami tego przewodnika.
  \begin{equation}
    \boxed{R = \frac U I \unit \Omega}
  \end{equation}
\end{law}

\subsection{Łączenie rezystorów}
\begin{align*}
  I &= \const & U &= \const\\
  U &= \sum_{i=1}^n U_i & I &= \sum_{i=1}^n I_i\\
  R_z &= \sum_{i=1}^n R_i & \frac{1}{R_z} &= \sum_{i=1}^n \frac{1}{R_i}
\end{align*}

\subsection{Opór elektryczny przewodnika}
\begin{definition}
  \bi{Opór elektryczny} to zdolność ciała do przeciwstawiania się przepływowi prądu elektrycznego.
  \begin{equation*}
    R = \frac{\varrho l}{S} \unit \Omega
  \end{equation*}
  \begin{symbols}
    \item $\varrho$ -- opór właściwy materiału przewodnika
    \item $l$ -- długość przewodu
    \item $S$ -- pole przekroju przewodnika
  \end{symbols}
\end{definition}

\section{Praca i moc prądu elektrycznego}
Przepływ prądu elektrycznego wiąże się z wykonywaniem przez elektrony pracy.
\begin{equation}\label{praca el}
  \boxed{W = UIt = \frac{U^2}{R} t = I^2 Rt}
\end{equation}

\subsection{Emisja ciepła (ciepło Joule'a)}
\begin{equation*}
  Q = W = I^2 Rt
\end{equation*}

\subsection{Energia elektryczna}
\begin{gather*}
  E_{el} = W = UIt \unit{kWh}\\
  (1\mathrm{kWh} = 3,6\mathrm{MJ})
\end{gather*}

\subsection{Moc prądu elektrycznego}
\begin{definition}
  \bi{Moc prądu elektrycznego} to stosunek pracy wykonanej przez przepływające elektrony do czasu,
  w którym tą pracę wykonywały.
  \begin{subequations}
    \begin{equation}\label{moc el}
      P = \frac{W}{t} \unit W
    \end{equation}
    Wstawiając równanie \ref{praca el} do równania \ref{moc el} otrzymujemy:
    \begin{equation}
      P = UI = I^2R = \frac{U^2}{R}
    \end{equation}
  \end{subequations}
\end{definition}

\section{Ogniwo galwaniczne}
\begin{definition}
  \bi{Ogniwo galwaniczne} to elektrolit kwasu, soli lub zasady, w którym zanurzono dwie elektrody
  wykonane np. z miedzi albo cynku. Na skutek dysocjacji elektrolitycznej między biegunami ogniwa
  powstaje różnica potencjałów, którą nazywamy \textbf{siłą elektromotoryczną} (SEM).
\end{definition}

\subsection{Prawo Ohma dla obwodu}
\begin{gather*}
  \mathcal E = U + U_w = U + IR_w \unit V\\
  I = \frac{\mathcal E}{R_z} = \frac{\mathcal E}{R + R_w} \unit A
\end{gather*}
\begin{symbols}
  \item $\mathcal E$ -- siła elektromotoryczna
  \item $R_w$ -- opór wewnętrzny ogniwa
\end{symbols}

\section{Prawa Kirchhoffa}
\begin{law}{Pierwsze prawo Kirchhoffa}
  Suma natężeń wpływających do węzła obwodu elektrycznego jest równa sumie natężeń wypływających z węzła.
  \begin{equation}
    \sum_{i=1}^n I_i = 0
  \end{equation}
\end{law}

\begin{law}{Drugie Prawo Kirchhoffa}
  Suma sił elektromotorycznych i spadków napięć w~obwodzie zamkniętym (oczku) jest równa zero.
  \begin{equation}
    \sum_{i=1}^n \mathcal E_i + \sum_{i=1}^n I_i R_i = 0
  \end{equation}
\end{law}

\section{Przewodnictwo ciał stałych}
Wyróżnia się trzy grupy ciał stałych ze względu na właściwości elektryczne:
\begin{itemize}
  \item przewodniki
  \item izolatory
  \item półprzewodniki
\end{itemize}

\textbf{Przewodnikami} są przede wszystkim metale takie jak miedź i żelazo. Dobre przewodzą prąd,
bo posiadają wolne elektrony. Wraz ze wzrostem temperatury opór elektryczny przewodników wzrasta
wskutek drgań sieci krystalicznej, w~której poruszają się elektrony.

\textbf{Izolatory} nie przewodzą prądu elektrycznego lub robią to bardzo słabo z~powodu braku
wolnych elektronów.

\textbf{Półprzewodniki} to materiały które mogą wykazywać właściwości zarówno izolatorów, jak i
przewodników. Wraz ze wzrostem temperatury ich opór elektryczny maleje, gdyż część elektronów
przeskakuje z poziomu podstawowego do poziomu przewodnictwa, stając się nośnikami prądu
elektrycznego. Poprzez domieszkowanie półprzewodnika pierwiastkami z grupy 13 bądź 15 układu
okresowego uzyskuje się odpowiednio \emph{półprzewodnik dziurowy} (typ „p”) oraz \emph{półprzewodnik
elektronowy} (typ „n”).

\section{Dioda półprzewodnikowa}
Dioda półprzewodnikowa jest złożona z dwóch złączonych półprzewodników --- jeden typu „p”, a drugi
typu „n” --- tworzących złączę p-n/n-p. Dioda półprzewodnikowa przepuszcza prąd tylko w jednym kierunku.
