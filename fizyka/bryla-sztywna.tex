  \chapter{Bryła sztywna}
    \section{Ruch obrotowy}
      \subsection{Prędkość kątowa}
        \begin{equation}
          \omega = \frac{\Delta\alpha}{\Delta t} \unit{\frac{rad}{s},\;\frac{1}{s}}
        \end{equation}
      \subsection{przyśpieszenie kątowe}
        \begin{equation}
          \varepsilon = \frac{\Delta\omega}{\Delta t} \unit{\frac{rad}{s^2},\;\frac{1}{s^2}}
        \end{equation}
      \subsection{Prędkość liniowa (styczna)}
        \begin{gather}
          \vec v = \vec\omega \times \vec r\unit{\frac{m}{s}}\\
          v = \omega r\sin\measuredangle(\vec\omega, \vec r)
        \end{gather}
        dla $\vec\omega \perp \vec r$:
        \begin{equation}
          v =\omega r
        \end{equation}
      \subsection{przyśpieszenie liniowe}
        \begin{equation}
          a_r = \varepsilon r \unit{\frac{m}{s^2}}
        \end{equation}
      \section{Równania obrotu}
        \begin{gather}
          \omega = \omega_0 \pm \varepsilon t\\
          \alpha = \omega_0t \pm \frac{\varepsilon t^2}{2}\\
        \end{gather}
        dla $\omega_0 = 0$:
        \begin{equation}
          \alpha = \frac{1}{2} \omega t
        \end{equation}
    \section{Moment bezwładności}
      \begin{equation}
        I = \sum_{i=1}^n m_ir_i^2 \unit{kg\cdot m^2}
      \end{equation}
        \subsection{Momenty bezwładności wybranych brył}
        kula: $I_0 = \frac{2}{5}mr^2$\\
        walec: $I_0 = \frac{1}{2}mr^2$\\
        pręt: $I_0 = \frac{1}{12}ml^2$\\
        rura grubościenna: $I_0 = \frac{1}{2}m(r_1^2 + r_2^2)$
      \subsection{Twierdzenie Steinera}
        \begin{equation}
          I = I_0 +mx^2
        \end{equation}
    \section{Energia kinetyczna}
      \begin{gather}
        E_{k_o} = \sum_{i=1}^n \frac{m_iv_i}{2} \unit{J}\\
        E_{k_o} = \frac{I\omega^2}{2}
      \end{gather}
    \section{Moment siły}
      \begin{gather}
        \vec M = \vec r\times\vec F \unit{N\cdot m}\\
        M = rF\sin\measuredangle(\vec r, \vec F)
      \end{gather}
      dla $\vec r \perp \vec F$:
      \begin{equation}
        M = rF
      \end{equation}
      dla $\vec r \parallel \vec F$:
      \begin{equation}
        M = 0
      \end{equation}
      \subsection{Wypadkowy moment siły}
        \begin{gather}
          M_w = \sum_{i=1}^n M_i\\
          M_w = \varepsilon I
        \end{gather}
      \subsection{Równowaga bryły sztywnej}
        \begin{gather}
          F_w = 0\\
          M_w = 0
        \end{gather}
    \section{Moment pędu}
      \begin{gather}
        \vec L = \vec r \times\vec p \unit{\frac{kg\cdot m^2}{s}}\\
        L = rp\sin\measuredangle(\vec r,\vec p)\\
        L = mrv\sin\measuredangle(\vec r,\vec v)
      \end{gather}
      dla $\vec p \perp \vec r$:
      \begin{equation}
        L = rp = mrv
      \end{equation}
      \begin{equation}
        L = \sum_{i=1}^n m_ir_iv_i\sin\measuredangle(\vec r,\vec v)
      \end{equation}
      dla $\vec r \perp \vec v$:
      \begin{equation}
        L = \omega I
      \end{equation}
