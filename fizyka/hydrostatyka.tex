\chapter{Hydrostatyka}

\section{Ciśnienie i parcie}

\subsection{Ciśnienie}
\begin{equation}
  p = \frac{F_N}{S} \unit{Pa}
\end{equation}
dla $F_N = m g$:
\begin{equation}
  p = \frac{m g}{S}
\end{equation}

\subsection{Parcie}
\begin{equation*}
  P = pS \unit N
\end{equation*}

\subsection{Ciśnienie hydrostatyczne}
\begin{equation*}
  p_h = \frac P S = \varrho_c gh \unit{Pa}
\end{equation*}

\subsection{Paradoks hydrostatyczny}

\section{Prawo Pascala}
\begin{equation}
  p_1 = p_2 \Rightarrow \frac{F_1}{S_1} = \frac{F_2}{S_2}
\end{equation}

\subsection{Naczynia połączone}
\begin{equation*}
  \varrho_1 h_1 = \varrho_2 h_2
\end{equation*}

\section{Prawo Archimedesa}
\begin{equation}
  F_W = \varrho_c g V_z \unit N
\end{equation}

\subsection{Warunki wypływania}
\begin{align*}
  F_W > F_g &\Rightarrow \text{ciało wypływa}\\
  F_W = F_g &\Rightarrow \text{ciało pływa}\\
  F_W < F_g &\Rightarrow \text{ciało tonie}
\end{align*}

% vim:spell:spl=pl
