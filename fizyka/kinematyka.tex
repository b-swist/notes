\chapter{Kinematyka}

\section{Wektory}

\subsection{Iloczyn skalarny}
\begin{equation}
  c = \vec a \cdot \vec b = |\vec a| \cdot |\vec b| \cdot \cos\measuredangle(\vec a, \vec b)
\end{equation}

\subsection{Iloczyn wektorowy}
\begin{equation}
  \vec c = \vec a \times \vec b =|\vec a| \cdot |\vec b| \cdot \sin\measuredangle(\vec a, \vec b)
\end{equation}

\section{Opis ruchu}
\begin{gather*}
  v_\text{śr} = \frac s t \unit{\frac m s}\\
  \vec v_\text{śr} = \frac{\Delta \vec x}{\Delta t} \unit{\frac m s}\\
  \vec a = \frac{\Delta \vec v}{\Delta t} \unit{\frac{m}{s^2}}
\end{gather*}

\section{Ruch jednostajny prostoliniowy}
\Graph[scale=0.6]{./figures/01-kinematyka}
\begin{gather*}
  v = \const\\
  v = \frac s t\\
  s = vt\\
  x = x_0 \pm vt
\end{gather*}

\section{Ruch jednostajnie przyspieszony}
\Graph[scale=0.52]{./figures/02-kinematyka}
\begin{gather*}
  a = \const\\
  a = \frac{\Delta v}{\Delta t}\\
  v_k = v_0 + at\\
  s = v_0 t + \frac{a t^2}{2}
\end{gather*}
jeżeli $v_0 = 0$:
\begin{gather*}
  v = at\\
  s = \frac{a t^2}{2}
\end{gather*}

\subsection{Stosunek przebytych dróg do odcinków czasu}
\begin{equation*}
  s_1:s_2:s_3:s_4:s_5:\ldots = 1:3:5:7:9:\ldots
\end{equation*}

\section{Ruch jednostajnie opóźniony}
\Graph[scale=0.52]{./figures/03-kinematyka}
\begin{gather*}
  v_k = v_0 - at\\[0.8em]
  \begin{aligned}
    s &= v_0 t - \frac{a t^2}{2}\\
    &= v_0 t - \frac 1 2 \Delta vt
  \end{aligned}
\end{gather*}
jeżeli $v_k = 0$:
\begin{gather*}
  v_0 = at\\
  s = \frac 1 2 v_0 t
\end{gather*}

\section{Rzut pionowy}

\subsection{Wznoszenie się}
\begin{gather*}
  h = v_0 t - \frac{g t^2}{2}\\
  v = v_0 - gt
\end{gather*}

\subsection{Opadanie}
\begin{gather*}
  h = v_0t + \frac{gt^2}{2}\\
  v = v_0 + gt
\end{gather*}
jeżeli $v_0 = 0$:
\begin{equation*}
  v = gt
\end{equation*}

\section{Rzut poziomy}
\Graph[scale=0.65]{./figures/04-kinematyka}
\begin{gather*}
  h = \frac{g t^2}{2}\\
  x = v_0 t = v_0 \sqrt{\frac{2h}{g}}\\
  v = \sqrt{v_0^2 + v_y^2} = \sqrt{v_0^2 + (gt)^2}
\end{gather*}

\section{Rzut ukośny}
\begin{align*}
  v_{0_x} &= v_0 \cos \alpha\\
  v_{0_y} &= v_0 \sin \alpha
\end{align*}

\begin{equation}
  y(x) = x \tg \alpha - x^2 \cdot \frac{g}{2 v_0^2 \cos^2 \alpha}
\end{equation}

\begin{gather*}
  t_c = \frac{2 v_0 \sin \alpha}{g}\\
  h_\text{max} = \frac{v_0^2 \sin^2 \alpha}{2g}\\
  z = \frac{v_0^2 \sin 2\alpha}{g}
\end{gather*}
dla $\alpha = 45^\circ$:
\begin{equation*}
  z = \frac{v_0^2}{2g}
\end{equation*}

\section{Ruch jednostajny po okręgu}

\begin{gather*}
  \alpha = \frac Lr \unit{rad}\\
  f = \frac nt \unit{Hz}\\
  \omega = \frac{\Delta \alpha}{\Delta t} \unit{\frac{rad}{s}}
\end{gather*}
dla jednego obrotu:
\begin{gather*}
  f = \frac 1 T\\
  \omega = \frac{2\pi}{T} = 2\pi f
\end{gather*}

\begin{gather*}
  v = \omega r\\
  a_r = \frac{v^2}{r}
\end{gather*}

\begin{gather*}
  \vec v = \vec \omega \times \vec r\\
  v = \omega r \sin\measuredangle(\vec \omega, \vec r)
\end{gather*}
dla $\vec \omega \perp \vec r$:
\begin{equation*}
  v = \omega r
\end{equation*}

\section{Przyspieszenie w ruchu po okręgu}
\begin{gather*}
  \vec a_s = \frac{\Delta \vec v}{\Delta t}\\[0.5em]
  a_w = \sqrt{a_r^2 + a_s^2}
\end{gather*}

% vim:spell:spl=pl
