\chapter{Praca, moc, energia}
  \section{Praca}
    \begin{gather}
      W = \vec F\Delta\vec r \unit{J}\\
      W = F\Delta r \cos\measuredangle(\vec F, \Delta\vec r)
    \end{gather}
    dla $\alpha = 0^\circ$:
    \begin{equation}
      W = F\Delta r = Fs
    \end{equation}
    dla $\alpha = 90^\circ$:
    \begin{equation}
      W = 0
    \end{equation}
  \section{Moc}
    \begin{equation}
      P = \frac{W}{t} \unit{W}
    \end{equation}
    dla $v = \const$:
    \begin{equation}
      P = Fs
    \end{equation}
  \section{Energia mechaniczna}
    \subsection{Energia kinetyczna}
      \begin{align}
        &E_k = \frac{mv^2}{2} \unit{J}\\
        &\Delta E_k = W
      \end{align}
    \subsection{Energia potencjalna}
      \begin{align}
        &E_p = mgh \unit{J}\\
        &\Delta E_p = W
      \end{align}
    \subsection{Zasada zachowania energii}
      \begin{align}
        &E_c = E_k + E_p\\
        &E_c = \const \Rightarrow \Delta E_c = 0\\
        &\Delta E_c = \Delta E_p + \Delta E_k
      \end{align}
  \section{Sprawność}
    \begin{equation}
      \eta = \frac{E_{u\dot{z}yt.}}{E_{pob.}}\: (\cdot 100\%) = \frac{W_{u\dot{z}yt.}}{E_{pob.}}\: (\cdot 100\%)
    \end{equation}
    \begin{equation}
      \eta_{u\dot{z}yt.} = \prod_{i=1}^n \eta_i
    \end{equation}
