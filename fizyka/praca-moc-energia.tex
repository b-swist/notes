\chapter{Praca, moc, energia}

\section{Praca}
\begin{equation}
  \begin{gathered}
    W = \vec F \Delta \vec r \unit J\\
    W = F \Delta r \cos\measuredangle(\vec F, \Delta \vec r)
  \end{gathered}
\end{equation}
dla $\alpha = 0^\circ$:
\begin{equation*}
  W = F \Delta r = Fs
\end{equation*}
dla $\alpha = 90^\circ$:
\begin{equation*}
  W = 0
\end{equation*}

\section{Moc}
\begin{equation}
  P = \frac W t \unit W
\end{equation}
dla $v = \const$:
\begin{equation*}
  P = Fs
\end{equation*}

\section{Energia mechaniczna}

\subsection{Energia kinetyczna}
\begin{equation}
  \begin{gathered}
    E_k = \frac{mv^2}{2} \unit J\\
    \Delta E_k = W
  \end{gathered}
\end{equation}

\subsection{Energia potencjalna}
\begin{equation}
  \begin{gathered}
    E_p = mgh \unit J\\
    \Delta E_p = W
  \end{gathered}
\end{equation}

\subsection{Zasada zachowania energii}
\begin{gather*}
  E_c = \const\\
  E_c = E_k + E_p\\
  \Delta E_c = \Delta E_p + \Delta E_k
\end{gather*}

\section{Sprawność}
\begin{equation}
  \eta = \frac{E_\text{użyt.}}{E_\text{pob.}} = \frac{W_\text{użyt.}}{E_\text{pob.}}
\end{equation}

\begin{equation*}
  \eta_\text{użyt.} = \prod_{i=1}^n \eta_i
\end{equation*}
