  \chapter{Grawitacja}
    \section{Prawa Keplera}
      \subsection{Pierwsze prawo}
      \subsection{Drugie prawo}
        \begin{align}
          s_1 &= s_2\\
          L_1 = L_2 &\Rightarrow r_1v_1 = r_2v_2
        \end{align}
      \subsection{Trzecie prawo}
        \begin{equation}
          \frac{T^2}{r^3} = const.
        \end{equation}
    \section{Prawo powszechnego ciążenia}
      \begin{equation}
        F = G\frac{m_1m_2}{r^2} \unit{N}
      \end{equation}
      gdzie:
      \begin{equation}
        G = 6,67\cdot 10^{-11} \unit{\frac{N\cdot m^2}{kg^2}}
      \end{equation}
      \begin{gather}
        F = \frac{4}{3}\pi RGdm\\
        F \sim dR
      \end{gather}
    \section{Natężenie pola grawitacyjnego}
      \begin{equation}
        \vec\gamma = \frac{\vec F_g}{m} \unit{\frac{N}{kg},\;\frac{m}{s^2}}
      \end{equation}
      dla pola centralnego:
      \begin{equation}
        \gamma = \frac{GM}{r^2}
      \end{equation}
    \section{Praca w polu grawitacyjnym}
      \begin{gather}
        W = mgh\\
        \Delta E_p = W
      \end{gather}
      \begin{gather}
        W_{Z_{(A\rightarrow B)}} = GMm\left(\frac{1}{r_A} - \frac{1}{r_B}\right)\\
        W_{g_{(A\rightarrow B)}} = -W_{Z_{(A\rightarrow B)}}
      \end{gather}
    \section{Energia w polu grawitacyjnym}
      \begin{equation}
        E_p = -\frac{GMm}{r}
      \end{equation}
    \section{Potencjał pola grawitacyjnego}
      \begin{gather}
        V = \frac{E_p}{m} \unit{\frac{J}{kg}}\\
        \Delta V = \frac{\Delta E_p}{m}
      \end{gather}
    \section{Prędkości kosmiczne}
      \subsection{Pierwsza prędkość kosmiczna}
        \begin{equation}
          v_{{}_\mathrm{I}} = \sqrt{\frac{GM}{r}}
        \end{equation}
      \subsection{Satelita geostacjonarny}
        \begin{equation}
          r = \sqrt[3]{\frac{GMT^2}{4\pi^4}}
        \end{equation}
      \subsection{Druga prędkość kosmiczna}
        \begin{equation}
          v_{{}_\mathrm{II}} = \sqrt{\frac{2GM}{r}} = v_{{}_\mathrm{I}}\sqrt{2}
        \end{equation}
