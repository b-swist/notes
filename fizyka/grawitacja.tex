\chapter{Grawitacja}

\section{Prawa Keplera}

\begin{law}{Pierwsze prawo Keplera}
  Planety poruszają się po elipsach, a jednym z jej ognisk znajduje się Słońce.
\end{law}

\begin{law}{Drugie prawo Keplera}
  Promień wodzący planety w jednolitych odstępach czasu zakreśla jednolite pola.
  \begin{align*}
    s_1 &= s_2\\
    L_1 = L_2 &\Rightarrow r_1 v_1 = r_2 v_2
  \end{align*}
\end{law}

\begin{law}{Trzecie prawo Keplera}
  Iloraz kwadratu okresu obiegu planety wokół Słońca i sześcianu średniej odległości planety od
  Słońca jest jednakowy dla wszystkich planet Układu Słonecznego.
  \begin{equation*}
    \frac{T^2}{r^3} = \const
  \end{equation*}
\end{law}

\section{Prawo powszechnego ciążenia}
\begin{law}{Prawo powszechnego ciążenia}
  Siła wzajemnego oddziaływania dwóch ciał jest wprost proporcjonalna do iloczynu mas tych ciał i
  odwrotnie proporcjonalna do kwadratu odległości między ich środkami mas.
  \begin{equation}
    F = G\frac{m_1 m_2}{r^2} \unit N
  \end{equation}
  \begin{symbols}
    \item $G$ -- stała grawitacji
  \end{symbols}
  \begin{equation*}
    G = 6,67\cdot 10^{-11} \unit{\frac{N \cdot m^2}{kg^2}}
  \end{equation*}
\end{law}

\subsection{Siły grawitacji wewnątrz lub na powierzchni planety}
\begin{gather*}
  F = \frac 4 3 \pi RGdm\\
  F \sim dR
\end{gather*}

\section{Pole grawitacyjne}
\begin{definition}
  \bi{Pole grawitacyjne} to przestrzeń, w której na umieszczone tam ciała działa siła grawitacji.
\end{definition}

\subsection{Natężenie pola grawitacyjnego}
\begin{definition}
  \bi{Natężenie pola grawitacyjnego} to stosunek siły grawitacji działającej na ciało próbne do
  masy tego ciała.
  \begin{subequations}
    \begin{equation}
      \vec \gamma = \frac{\vec F_g}{m} \units{\frac{N}{kg}}{\frac{m}{s^2}}
    \end{equation}
    dla pola centralnego:
    \begin{equation}
      \gamma = \frac{GM}{r^2}
    \end{equation}
  \end{subequations}
\end{definition}

\subsection{Praca w polu grawitacyjnym}
\begin{equation}
  \begin{gathered}
    W_{Z_{(A \rightarrow B)}} = GMm\left(\frac{1}{r_A} - \frac{1}{r_B}\right)\\
    W_{g_{(A \rightarrow B)}} = -W_{Z_{(A \rightarrow B)}}
  \end{gathered}
\end{equation}

\subsection{Energia w polu grawitacyjnym}
\begin{equation}
  E_p = -\frac{GMm}{r}
\end{equation}

\subsection{Potencjał pola grawitacyjnego}
\begin{definition}
  \bi{Potencjał pola grawitacyjnego} to stosunek energii potencjalnej ciała próbnego do jego masy.
  \begin{subequations}
    \begin{gather}
      V = \frac{E_p}{m} \unit{\frac{J}{kg}}\\
      \Delta V = \frac{\Delta E_p}{m}
    \end{gather}
  \end{subequations}
\end{definition}

\section{Prędkości kosmiczne}

\begin{definition}
  \bi{Pierwsza prędkość kosmiczna} to prędkość, jaką trzeba nadać ciału by oderwało się z
  powierzchni źródła pola grawitacyjnego i krążyło wokół niego po orbicie kołowej.
  \begin{equation*}
    v_{\scriptscriptstyle\mathrm I} = \sqrt\frac{GM}{r}
  \end{equation*}
\end{definition}

\begin{definition}
  \bi{Druga prędkość kosmiczna} to prędkość, jaką trzeba nadać ciału by oddaliło się nieskończenie
  daleko od źródła pola grawitacyjnego.
  \begin{equation*}
    v_{\scriptscriptstyle \mathrm{II}} = \sqrt{\frac{2GM}{r}} = v_{\scriptscriptstyle\mathrm I} \sqrt 2
  \end{equation*}
\end{definition}

\subsection{Satelita geostacjonarny}
\begin{definition}
  \bi{Satelita geostacjonarny} to taki satelita, który krąży nad powierzchnią w płaszczyźnie
  równika nad tym samym punktem (jego okres obiegu jest równy okresowi obrotu planety).
  \begin{equation*}
    r = \sqrt[3]{\frac{GMT^2}{4\pi^2}}
  \end{equation*}
\end{definition}
