  \chapter{Ruch drgający}
    \begin{gather}
      F_z = kx\\
      F_s = -kx
    \end{gather}
    \begin{equation}
      k = \left|\frac{F_s}{x}\right| \unit{\frac Nm}
    \end{equation}
    \section{Ruch harmoniczny}
      \begin{gather}
        x = r\sin\alpha\\
        T = 2\pi\sqrt{\frac mk} \unit{s}
      \end{gather}
      \subsection{Równania ruchu harmonicznego}
        \begin{align}
          x(t) &= A\sin(\omega t + \varphi_0)\\
          v(t) &= \omega A\cos(\omega t + \varphi_0)\\
          a(t) &= -\omega^2A\sin(\omega t + \varphi_0)
        \end{align}
        \begin{align}
          x_{max} &= A\ \text{dla}\ \sin90^\circ = 1\\
          v_{max} &= \omega A\ \text{dla}\ \cos0^\circ = 1\\
          a_{max} &= -\omega^2A\ \text{dla}\ \sin90^\circ = 1\\
        \end{align}
      \subsection{Łączenie sprężyn}
        \begin{gather}
          F = const.\\
          x = \sum_{i=1}^n x_i\\
          \frac{1}{k} = \sum_{i=1}^n \frac{1}{k_i}
        \end{gather}
        \begin{gather}
          x = const.\\
          F_c = \sum_{i=1}^n F_i\\
          k = \sum_{i=1}^n k_i
        \end{gather}
    \section{Energia w ruchu harmonicznym}
      \begin{equation}
        W = \frac{1}{2}Fx \Rightarrow E_{p_s} = \frac{1}{2}kx^2
      \end{equation}
      \begin{align}
        E_c &= E_{p_s} + E_k\\
        E_c &= \frac{1}{2}kA^2\\
        E_k &= \frac{1}{2}k(A^2 - x^2)
      \end{align}
    \section{Wahadło matematyczne}
      \begin{equation}
        F = F_g\sin\alpha
      \end{equation}
      dla małych kątów $\sin\alpha\approx\alpha$:
      \begin{gather}
        F = mg\alpha\\
        T = 2\pi\sqrt{\frac lg}
      \end{gather}
