\chapter{Elektromagnetyzm}

\section{Zjawisko indukcji elektromagnetycznej}\label{faraday}
\begin{definition}
  \textbf{Zjawisko indukcji elektromagnetycznej Faradaya} polega na wzbudzeniu w zwojnicy zmiennego
  pola elektrycznego wskutek działania zewnętrznego, zmiennego pola magnetycznego. W układzie
  zwojnicy indukuje się siła elektromagnetyczna, a po podłączeniu odbiornika w układzie płynie prąd
  indukcyjny.

  \vspace{1em}\noindent Jego siła zależy od:
  \begin{itemize}
    \item ilości zwojów,
    \item wartość indukcji magnetycznej zewnętrznego pola (magnesu),
    \item szybkość zmiany pola magnetycznego (szybkość ruchu magnesu).
  \end{itemize}
\end{definition}

\begin{law}{Reguła Lenza}
  Prąd indukcyjny wytworzony przez indukcję elektromagnetyczną przyjmuje taki kierunek, by pole
  magnetyczne wytworzone przez jego przepływ przeciwstawiało się przyczynie wywołania prądu.
\end{law}

\section{Zjawisko samoindukcji}

\subsection{Strumień indukcji magnetycznej}
\begin{definition}
  \textbf{Strumień indukcji magnetycznej} to strumień pola dla indukcji magnetycznej.
  \begin{equation}
    \begin{gathered}
      \Phi = \vec B \cdot \vec S\unit{Wb}\\
      \Phi = BS \cos\measuredangle(\vec B, \vec S)
    \end{gathered}
  \end{equation}
\end{definition}

\newpage %%TODO: delete
\begin{definition}
  \textbf{Współczynnik samoindukcji} (indukcyjność) to stały stosunek wytwarzanego przez prąd
  strumienia indukcji magnetycznej do natężenia tego prądu.
  \begin{gather*}
    L = \frac \Phi I \unit H\\
    L = \const
  \end{gather*}
\end{definition}

\begin{definition}
  \textbf{Zjawisko samoindukcji} polega na wytwarzaniu się \textit{SEM} w zwojnicy pod wpływem
  zjawiska indukcji Faradaya [\ref{faraday}].
  \begin{equation}
    \begin{aligned}
      \mathcal E &= -\frac{\Delta \Phi}{\Delta t}\\
      &= -L\frac{\Delta I}{\Delta t}
    \end{aligned}
  \end{equation}
\end{definition}

\section{Prąd przemienny}

\subsection{Wzory wartości chwilowych}
\begin{subequations}
  \begin{gather}
    \Phi(t) = BS \cos \omega t\\[10pt]
    \begin{aligned}
      \mathcal E(t) &= \omega BS \sin \omega t\\
      &= \mathcal E_0\sin\omega t
    \end{aligned}\\[10pt]
    \begin{aligned}
      I(t) &= \frac{\mathcal E_0}{R}\sin\omega t\\
      &= I_0\sin \omega t
    \end{aligned}\\[10pt]
    U(t) = I_0 R \sin \omega t\\[10pt]
    P(t) = R I_0^2\sin^2 \omega t
  \end{gather}
\end{subequations}

\subsection{Moc średnia}
\begin{equation}
  P_\text{śr} = \frac 1 2 R I_0^2
\end{equation}

\subsection{Wzory wartości skutecznych}
\begin{subequations}
  \setlength\jot{5pt}
  \begin{gather}
    I_s = \frac{I_0}{\sqrt 2}\\
    U_s = \frac{U_0}{\sqrt 2}\\
    P_s = \frac 1 2 U_0 I_0\\
  \end{gather}
\end{subequations}

\section{Transformator}
\begin{definition}
  \textbf{Transformator} to urządzenie służące do zmiany napięcia prądu przemiennego.

  \begin{equation}
    \frac{U_1}{U_2} = \frac{n_1}{n_2} = \frac{I_2}{I_1}
  \end{equation}
  \begin{symbols}
    \item $U_1$ -- napięcie na uzwojeniu pierwotnym
    \item $n_1$ -- ilość zwojów na uzwojeniu pierwotnym
    \item $I_1$ -- natężenie prądu na uzwojeniu pierwotnym
      \vspace{10pt}
    \item $U_2$ -- napięcie na uzwojeniu wtórnym
    \item $n_2$ -- ilość zwojów na uzwojeniu wtórnym
    \item $I_2$ -- natężenie prądu na uzwojeniu wtórnym
  \end{symbols}
\end{definition}

\subsection{Przekładnia transformatora}
\begin{definition}
  \textbf{Przekładnia transformatora} to stosunek ilości zwojów (lub napięcia) na uzwojeniu wtórym
  do ilości zwojów (lub napięcia) uzwojeniu pierwotnym.
  \begin{equation}
    \vartheta = \frac{n_2}{n_1} = \frac{U_2}{U_1}
  \end{equation}
\end{definition}

% vim:spell:spl=pl
