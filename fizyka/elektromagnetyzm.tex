\chapter{Elektromagnetyzm}
  \section{Zjawisko indukcji elektromagnetyczej}

    \begin{definition}
      \textbf{Zjawisko indukcji elektromagnetyczej Faradaya} polega na wzbudzeniu w zwojnicy zmiennego pola elektrycznego wskutek działania zewnętrznego, zmiennego pola magnetycznego. W układzie zwojnicy indukuje się siła elektromagnetyczna, a po podłączeniu odbiornika w układzie płynie prąd indukcyjny.
    \end{definition}

    \begin{law}{Reguła Lenza}
      Prąd indukcyjny wytworzony przez indukcję elektromagnetyczną przyjmuje taki kierunek, by pole magnetyczne wytworzone przez jego przepływ przeciwstawiało się przyczynie wywołania prądu.
    \end{law}

  \section{Zjawisko samoindukcji}

  \section{Prąd przemienny}
    \subsection{Transformator}
