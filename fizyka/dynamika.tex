  \chapter{Dynamika}
    \section{Zasady dynamiki Newtona}
      \subsection{Pierwsza zasada}
        \begin{equation}
          \vec F_w = 0 \Rightarrow \vec v = 0 \lor \vec v = const.
        \end{equation}
      \subsection{Druga zasada}
        \begin{gather}
          F_w \ne 0 \Rightarrow a = const.\\
          \vec a = \frac{\vec F_w}{m} \Rightarrow \vec F_w = m\vec a \unit{N}
        \end{gather}
      \subsection{Trzecia zasada}
        \begin{align}
          \vec F_{AB} &= -\vec F_{BA}\\
          F_{AB} &= F_{BA}
        \end{align}
    \section{Ruch na równi pochyłej}
      \begin{align}
        \frac{\vec F_Z}{\vec F_g} = \sin\alpha &\Rightarrow \vec F_Z = \vec F_g\sin\alpha = mg\sin\alpha\\
        \frac{\vec F_N}{\vec F_g} = \cos\alpha &\Rightarrow \vec F_N = \vec F_g\cos\alpha = mg\cos\alpha
      \end{align}
    \section{tbd}
    \section{Pęd ciała}
      \begin{equation}
        \vec p = m\vec v \unit{\frac{kg\cdot m}{s}}
      \end{equation}
      \begin{equation}
        \Delta p = F\Delta t
      \end{equation}
      \subsection{Zasada zachowania pędu}
        \begin{equation}
          \Delta \vec p = 0 \Leftrightarrow \vec p = const.
        \end{equation}
      \section{Środek masy}
        \begin{equation}
          x_c = \frac{m_1x_1+m_2x_2+\cdots+m_nx_n}{m_1+m_2+\cdots+m_n} =
          \frac{\sum\limits_{i=1}^n m_ix_i}{\sum\limits_{i=1}^n m_i}
        \end{equation}
      \section{Tarcie}
        \begin{align}
          T_s &= \mu_sF_N \unit{N}\\
          T_k &\leqslant \mu_kF_N \unit{N}
        \end{align}
      \section{Siła dośrodkowa}
        \begin{equation}
          F_{do} = \frac{mv^2}{r} \unit{N}
        \end{equation}
        \section{Siła bezwładności}
        \begin{equation}
          \vec F_b = -m\vec a \unit{N}
        \end{equation}
