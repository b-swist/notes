\chapter{Termodynamika}

  \section{Zasady dynamiki}
    \begin{law}{Zerowa zasada dynamiki}
      Jeżeli ciało A jest w równowadze termodynamicznej z ciałem B, a ciało B jest w równowadze termodynamicznej z ciałem C, to ciała A i C są również w równowadze termodynamicznej.
    \end{law}
    \begin{law}{Pierwsza zasada termodynamiki}
      Zmiana energii wewnętrznej ciała jest równa sumie ilości ciepła wymienionego z otoczeniem i pracy wykonanej nad ciałem przez siłę zewnętrzną.
    \end{law}
    \begin{equation}
      p = \frac 23\cdot\frac{NE_{k_{\textit{śr}}}}{V} \unit{Pa}
    \end{equation}
    gdzie $N$ - liczba cząstek gazu
    \begin{equation}
      E_{k_{\acute sr.}} = \frac{1}{2}mv_{\acute sr.}^2 \unit{J}
    \end{equation}
  \section{Równanie gazu doskonałego}
    \begin{equation}
      \frac{p_1V_1}{T_1} = \frac{p_2V_2}{T_2} \Rightarrow \frac{pV}{T} = \const
    \end{equation}
    \subsection{Równanie Clapeyrona}
      \begin{equation}
        pV = nRT = NkT
      \end{equation}
      gdzie:
      \begin{gather}
        R =  8,31 \unit{\frac{J}{mol\cdot K}}\\
        k = \frac{R}{N_A} = 1,38\cdot 10^{-23} \unit{\frac{J}{K}}
      \end{gather}
  \section{Przemiany gazu doskonałego}

    \subsection{Przemiana izotermiczna}
      \begin{gather}
        T = \const\\
        \frac{p_1V_1}{T} = \frac{p_2V_2}{T} \Rightarrow p_1V_1 = p_2V_2\\
        pV = \const \Rightarrow p = \frac{const.}{V}\\
        \text{(Prawo Boyle'a)}
      \end{gather}
    \subsection{Przemiana izochoryczna}
      \begin{gather}
        V = \const\\
        \frac{p_1V}{T_1} = \frac{p_2V}{T_2} \Rightarrow \frac{p_1}{T_1} = \frac{p_2}{T_2}\\
        \frac{p}{T} = \const \Rightarrow p = T\cdot const.\\\text{(Prawo Charles'a)}
      \end{gather}
      % \begin{tikzpicture}
      %   \draw[->] (-0.3, 0) -- (4, 0) node[below] {$T$};
      %   \draw[->] (0, -0.3) -- (0, 3.5) node[left] {$p$};
      %   \draw[thick] (0.58, 0.5) -- (3.5, 3);
      %   \draw[dotted] (0, 0) -- (0.58, 0.5);
      % \end{tikzpicture}

    \subsection{Przemiana izobaryczna}
      \begin{gather}
        p = \const\\
        \frac{pV_1}{T_1} = \frac{pV_2}{T_2} \Rightarrow \frac{V_1}{T_1} = \frac{V_2}{T_2}\\
        \frac{V}{T} = \const \Rightarrow V = T\cdot const.\\
        \text{(Prawo Gay-Lussaca)}
      \end{gather}
      % \begin{tikzpicture}
      %   \draw[->] (-0.3, 0) -- (4, 0) node[below] {$V$};
      %   \draw[->] (0, -0.3) -- (0, 3.5) node[left] {$T$};
      %   \draw[thick] (0.58, 0.5) -- (3.5, 3);
      %   \draw[dotted] (0, 0) -- (0.58, 0.5);
      % \end{tikzpicture}
  \section{Pierwsza zasada termodynamiki}
    \begin{equation}
      \Delta U = Q + W_z \unit{J}
    \end{equation}
    dla $Q > 0$ ciepło zostało pobrane\\
    dla $Q < 0$ ciepło zostało oddane
    \begin{gather}
      W_z = F_z\Delta x\cos\measuredangle(\vec F_z, \Delta\vec x)\\
      W_z = -W_{gazu}
    \end{gather}
    dla $W_z > 0$:
    \begin{equation}
      W_z = F_z\Delta x
    \end{equation}
    dla $W_z < 0$:
    \begin{equation}
      W_z = -F_z\Delta x
    \end{equation}
    \begin{equation}
      |W| = p|\Delta V|
    \end{equation}
  \section{Energia wewnętrzna gazu doskonałego}
    \begin{gather}
      U = N\cdot\frac{i}{2}kT\\
      \Delta U = N\cdot\frac{i}{2}k\Delta T
    \end{gather}
    gdzie $i$ - stopnie swobody cząstek
    \subsection{Przemiana izotermiczna}
      \begin{gather}
        T = \const \Leftrightarrow U = const.\\
        \Delta U = 0 \Rightarrow Q + W = 0
      \end{gather}
    \subsection{Przemiana izochoryczna}
      \begin{gather}
        V = \const \Rightarrow \Delta V = 0\\
        W = 0 \Rightarrow \Delta U = Q
      \end{gather}
    \subsection{Przemiana adiabatyczna}
      \begin{gather}
        Q = 0 \Rightarrow \Delta U = W\\
        pV^\kappa = \const
      \end{gather}
      gdzie:
      \begin{equation}
        \kappa = \frac{C_p}{C_V}
      \end{equation}
  \section{Ciepło molowe i właściwe}
    \subsection{Ciepło właściwe}
      \begin{gather}
        C_w = \frac{Q}{m\Delta T} \unit{\frac{J}{kg\cdot K}}\\
        Q = mC_w\Delta T
      \end{gather}
    \subsection{Ciepło molowe}
      \begin{equation}
        C = \frac{Q}{n\Delta T} \unit{\frac{J}{mol\cdot K}}
      \end{equation}
      ciepło molowe przy stałym ciśnieniu: $C_p$\\
      ciepło molowe przy stałej objętości: $C_V$
      \begin{align}
        &Q_p = Q_V + p\Delta V\\
        &C_p = C_V + R
      \end{align}
  \section{Energia wewnętrzna jako funkcja stanu}
    \begin{equation}
      \Delta U = Q_V = nC_V\Delta T
    \end{equation}
    \section{Silnik cieplny}
      % \begin{center}
      %   \begin{tikzpicture}[scale=1.5]
      %     \draw[->] (-0.3, 0) -- (4, 0) node[below] {$V$};
      %     \draw[->] (0, -0.3) -- (0, 3.5) node[left] {$p$};
      %     \draw[thick, ->] (1, 3) -- (2.5, 2.6);
      %     \draw[thick, ->] (2.5, 2.6) -- (3, 1);
      %     \draw[thick, <-] (1, 3) arc (180:270:2);
      %     \draw (2.05, 2.05) node {\LARGE $W$};
      %   \end{tikzpicture}
      % \end{center}
    \begin{equation}
      \eta = \frac{|Q_1|-|Q_2|}{Q_1} = \frac{T_1 - T_2}{T_1}
    \end{equation}
  \section{Przejścia fazowe}
    \begin{equation}
      Q = mC_w\Delta T
    \end{equation}
    woda - lód: $T_T = T_K = 0^\circ C$\\
    woda - para wodna: $T_W = T_S = 100^\circ C$
    \begin{equation}
      Q = mL
    \end{equation}
    \begin{equation}
      Q = mR
    \end{equation}
  \section{Rozszerzalność temperaturowa ciał}
    \subsection{Rozszerzalność obiętościowa}
      \begin{equation}
        \Delta V = V_0\alpha\Delta T
      \end{equation}
    \subsection{Rozszerzalność liniowa}
      \begin{equation}
        \Delta l = l_0\lambda\Delta T
      \end{equation}
