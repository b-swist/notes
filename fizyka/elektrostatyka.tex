\chapter{Elektrostatyka}
  \section{Ładunek}
    \begin{gather}
      e = 1,6\cdot 10^{-19} \unit{C}\\
      q = ne \unit{C}
    \end{gather}
  \section{Prawo Culomba}
    \begin{equation}
      F = k\frac{q_1q_2}{r^2} \unit{N}
    \end{equation}
    gdzie:
    \begin{gather}
      k = \frac{1}{4\pi\varepsilon_0} \approx 8,99\cdot 10^9 \unit{\frac{N\cdot m^2}{C^2}}\\
      \varepsilon_0 = 8,85\cdot 10^{-12} \unit{\frac{C^2}{N\cdot m^2}}
    \end{gather}
  \section{Natężenie pola elektrostatycznego}
    \begin{gather}
      \vec E = \frac{\vec F}{q} \unit{\frac{N}{C}}\\
      E = \frac{k|Q|}{r^2}
    \end{gather}
  \section{Rozmieszczenie ładunku na przewodniku}
    \begin{equation}
      \sigma = \frac{Q}{S} \unit{\frac{C}{m^2}}
    \end{equation}
  \section{Praca w polu centralnym}
    \begin{equation}
      W_{A\rightarrow B} = -kQq\left(\frac{1}{r_A} - \frac{1}{r_B}\right) \unit{J}
    \end{equation}
  \section{Energia w polu centralnym}
    \begin{equation}
      E_p = \frac{kQq}{r} \unit{J}
    \end{equation}
  \section{Potencjał w polu centralnym}
    \begin{gather}
      V = \frac{E_p}{q} = \frac{kQ}{r} \unit{V}\\
      W = q\Delta V = qU
    \end{gather}
  \section{Pojemność elektryczna przewodnika}
    \begin{equation}
      C = \frac{Q}{V} \unit{F}
    \end{equation}
  \section{Kondensator}
    \begin{equation}
      C= \frac{Q}{U}
    \end{equation}
    dla kondensatora płaskiego:
    \begin{equation}
      C= \frac{\varepsilon_0S}{d}
    \end{equation}
    \begin{equation}
      E = \frac{U}{d}
    \end{equation}
    \subsection{Łączenie kondensatorów}
      \begin{gather}
        Q = \const\\
        U = \sum_{i=1}^n U_i\\
        \frac{1}{C_z} = \sum_{i=1}^n \frac{1}{C_i}
      \end{gather}
      \begin{gather}
        U = \const\\
        Q = \sum_{i=1}^n Q_i\\
        C_z = \sum_{i=1}^n C_i
      \end{gather}
    \subsection{Kondensator z dielektrykiem}
    \begin{equation}
      C= \frac{\varepsilon_0\varepsilon_rS}{d} \unit{F}
    \end{equation}
    gdzie $\varepsilon_r$ - stała przenikalności dielektryka
    \subsection{Energia naładowaniego kondensatora}
      \begin{equation}
        E = \frac{1}{2}QU = \frac{1}{2}CU^2 = \frac{Q^2}{2C}
      \end{equation}
  \section{Ruch ładunków w polu elektrostatycznym}
    \begin{equation}
      F = qE
    \end{equation}
    jeżeli $F = F_w$:
    \begin{equation}
      a = \frac{qE}{m} = \frac{qU}{md} \unit{\frac{m}{s^2}}
    \end{equation}
