\chapter{Elektrostatyka}
  \section{Ładunek elektryczny}
    \begin{gather*}
      |e| = 1,6\cdot 10^{-19} \unit{C}\\
      q = ne \unit{C}
    \end{gather*}
    \begin{law}{Zasada zachowania ładunku}
      W izolowanym układzie całkowity ładunek elektryczny nie ulega zmianie.
      \begin{equation}
        \sum_{i=1}^n q_i = \const
      \end{equation}
    \end{law}

  \section{Prawo Coulomba}
    \begin{law}{Prawo Coulomba}
      Siła wzajemnego oddziaływania dwóch ładunków jest wprost proporcjonalna do iloczynu tych ładunków i odwrotnie proporcjonalna do kwadratu odległości między nimi.
      \begin{equation}\label{coulomb}
        \boxed{F = k\frac{Qq}{r^2} \unit{N}}
      \end{equation}
      \begin{symbols}
        \item $k$ -- współczynnik proporcjonalności (stała eletrostatyczna)
        \item $\varepsilon_0$ -- stała przenikalności elektrycznej próżni
      \end{symbols}
      \begin{gather*}
        k = \frac{1}{4\pi\varepsilon_0} \approx 8,99\cdot 10^9 \unit{\frac{N\cdot m^2}{C^2}}\\[0.5em]
        \varepsilon_0 = 8,85\cdot 10^{-12} \unit{\frac{C^2}{N\cdot m^2}}
      \end{gather*}
    \end{law}

  \section{Natężenie pola elektrostatycznego}
    \begin{definition}
      \bi{Natężenie pola elektrostatycznego} to stosunek siły elektrostatycznej działającej na \emph{dodatni} ładunek próbny $q$ w danym punkcie pola do~wartości tego ładunku.
      \begin{subequations}
        \begin{equation}\label{natężenie es}
          \vec E = \frac{\vec F}{q} \unit{\frac{N}{C}}\\[0.5em]
        \end{equation}
        Wstawiając równanie \ref{coulomb} do równania \ref{natężenie es} otrzymujemy:
        \begin{equation}
          \boxed{E = \frac{kQ}{r^2}}
        \end{equation}
      \end{subequations}
    \end{definition}

  \section{Rozmieszczenie ładunku na przewodniku}
    Po namagnesowaniu ciała cały dostarczony ładunek rozmieszcza się na jego zewnętrznej powierzchni. Pole elektromagnetyczne wewnątrz zanika.

    Rozmieszczenie ładunku na powierzchni zależy od jego kształu. Rozkład ładunku opisuje gęstość powierzchniowa ładunku --- iloraz ładunku i pole tej powierzchni.
    \begin{equation*}
      \sigma = \frac{Q}{S} \unit{\frac{C}{m^2}}
    \end{equation*}

  \section{Praca w polu centralnym}
    \begin{equation}\label{praca es}
      W_{A\rightarrow B} = -kQq\left(\frac{1}{r_A} - \frac{1}{r_B}\right) \unit{J}
    \end{equation}

  \section{Energia w polu centralnym}
  \begin{equation}\label{energia es}
      E_p = \frac{kQq}{r} \unit{J}
    \end{equation}

  \section{Potencjał w polu centralnym}
    \begin{definition}
      Potencjał pola elektrostatycznego to stosunek energii potencjalnej punktowego ciała do wartości ładunku próbnego umieszczonego w tym polu.
      \begin{subequations}
        \begin{equation}\label{potencjał es}
          V = \frac{E_p}{q} \unit{V}\\
        \end{equation}
        Podstawiajac równanie \ref{energia es} do równania \ref{potencjał es} otrzymujemy:
        \begin{equation}
          V = \frac{kQ}{r^2}
        \end{equation}
      \end{subequations}
    \end{definition}
    \begin{equation*}
        W = \Delta E_p = q\Delta V = qU
    \end{equation*}

  \section{Pojemność elektryczna przewodnika}
    \begin{definition}
      \bi{Pojemność elektryczna przewodnika} to stosunek ilości ładunku zgromadzonego na przewodniku do uzyskanego potencjału.
      \begin{equation}
        C = \frac{Q}{V} \unit{F}
      \end{equation}
    \end{definition}

  \section{Kondensator}
    \begin{definition}
      \bi{Kondensator} to element elektroniczny służący do gromadzenia ładunku elektrycznego.
      \begin{equation*}
        C = \frac{Q}{U}
      \end{equation*}
    \end{definition}

    \subsection{Łączenie kondensatorów}
      \begin{align*}
        U &= \const & Q &= \const\\
        Q &= \sum_{i=1}^n Q_i & U &= \sum_{i=1}^n U_i\\
        C_z &= \sum_{i=1}^n C_i & \frac{1}{C_z} &= \sum_{i=1}^n \frac{1}{C_i}
      \end{align*}

    \subsection{Kondensator płaski}
      \begin{definition}
        Kondensator płaski składa się z dwóch równoległych, metalowych okładek, między którymi znajduje się dielektryk.\\[1.5em]
        Bez dielektryka:
        \begin{equation*}
          C = \frac{\varepsilon_0S}{d} 
        \end{equation*}
        Uwzględniając dielektryk:
        \begin{equation*}
          C = \frac{\varepsilon_0\varepsilon_rS}{d}
        \end{equation*}
        Natężenie pola elektrycznego kondensatora:
        \begin{equation*}
          E = \frac{U}{d}
        \end{equation*}
        \begin{symbols}
          \item $d$ -- odległość między okładkami
          \item $S$ -- pole powierzchni okładek
          \item $\varepsilon_r$ -- stała przenikalność dielektryka
        \end{symbols}
      \end{definition}

    \subsection{Energia naładowaniego kondensatora}
      \begin{definition}
        \bi{energia naładowaniego kondensatora} to praca potrzebna do jego naładowania.
        \begin{gather*}
          E = \frac{1}{2}QU = \frac{1}{2}CU^2 = \frac{Q^2}{2C} \units{eV}{J}\\[0.5em]
          (1\mathrm{eV} = 1,6\cdot 10^{-19}\mathrm{J})
        \end{gather*}
      \end{definition}
  \section{Ruch ładunków w polu elektrostatycznym}
    Na naładowane cząstki w polu elektrostatycznym centralnym działa siła elektrostatyczna, przez którą cząstka zaczyna przyspieszać.
    \begin{equation*}
      F = qE
    \end{equation*}
    jeżeli $F = F_w$, to:
    \begin{equation*}
      a = \frac{qE}{m} = \frac{qU}{md} \unit{\frac{m}{s^2}}
    \end{equation*}
