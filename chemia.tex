\documentclass{article}
\title{Chemia}
\author{Bartosz Świst}
\date{2025-03-16}

\usepackage[utf8]{inputenc}
\usepackage[T1]{fontenc}
\usepackage{chemfig}
\usepackage[version=4]{mhchem}

\newcommand\formula[2]{\begin{center}\chemname{\chemfig{#1}}{#2}\end{center}}
\newcommand\reaction[1]{\begin{center}\ce{#1}\end{center}}

\begin{document}
  \maketitle\newpage

  \section{Kwasy karboksylowe}
    % \begin{center}\chemname{\chemfig{R-C(=[1]O)-[7]O-H}}{Wzór ogólny}\end{center}
    \formula{R-C(=[1]O)-[7]O-H}{Wzór ogólny}

    \subsection{Szereg homologiczny}
      \formula{H-C(=[1]O)-[7]O-H}{kwas mrówkowy (metanowy)}
      \formula{CH_3-C(=[1]O)-[7]O-H}{kwas octowy (etanowy)}
      \formula{CH_3-CH_2-C(=[1]O)-[7]O-H}{kwas propinowy (propanowy)}
      \formula{CH_3-CH_2-CH_2-C(=[1]O)-[7]O-H}{kwas maślany (butanowy)}

    \subsection{Właściwości}
      \begin{enumerate}
        \item Kwasy karboksylowe to słabe elektrolity, ulegające dysocjacji:
        \reaction{RCOOH <=> H+ + RCOO-}
        \item Reagują z:
        \begin{itemize}
          \item metalami, np:
          \reaction{2HCOOH + Zn -> (HCOO)2Zn + H2 ^}
          \item tlenkami metali, np:
          \reaction{2CH3COOH + CuO -> (CH3COO)2Cu + H2O}
          \item wodorotlenkami, np:
          \reaction{HCOOH + KOH -> HCOOK + H2O}
        \end{itemize}
        \item Są wypierane przez silniejsze kwasy z roztworów ich soli, np:
        \reaction{CH3COONa + HCl -> CH3COOH + NaCl}
        \reaction{2HCOOK + H2SO4 -> 2HCOOH + K2SO4}
      \end{enumerate}

    \subsection{Wyższe kwasy karboksylowe}
      \begin{enumerate}
        \item Nasycone:
          \begin{itemize}
            \item kwas stearynowy (\chemfig{C_{17}H_{35}COOH})
            \item kwas palmitynowy (\chemfig{C_{15}H_{31}COOH})
          \end{itemize}
        \item Nienasycone:
          \begin{itemize}
            \item kwas oleinowy (\chemfig{C_{17}H_{33}COOH})
          \end{itemize}
      \end{enumerate}

      \subsubsection*{Mydła}
        Mydła to związki wyższych kwasów karboksylowych z metalami:
        \begin{itemize}
          \item sodem (\chemfig{Na})
          \item potasem (\chemfig{K})
          \item magnesem (\chemfig{Mg})
          \item wapniem (\chemfig{Ca})
        \end{itemize}

  \section{Estry}
    \formula{R^1-C(=[6]O)-O-R^2}{Wzór ogólny}
    \textbf{\textit{Estryfikacja}} to odwaraclna reakcja kwasu z alkoholem, w~której powstaje ester i~woda w~obecności katalizatora --- silnego kwasu (zazwyczaj \chemfig{H_2SO_4}) --- w~celu obniżenia granicy energetycznej równania.
    \reaction{R^1COOH + R^2OH <-->[H2SO4] R^1COOR^2 + H2O}

    \subsection*{Hydroliza estrów}
      \begin{enumerate}
        \item Kwasowa, np:
        \reaction{CH3COOCH2CH3 + H2O <-->[H2SO4] CH3COOH + CH3CH2OH}
        \item Zasadowa, np:
        \reaction{HCOOCH3 + NaOH ->[H2O] HCOONa + CH3OH}
      \end{enumerate}

\end{document}
